\nonstopmode
\documentclass[10pt, a4paper]{article}
%\usepackage[top=1.5cm]{geometry}
\usepackage{blindtext}
\usepackage[T1]{fontenc}
\usepackage[latin1]{inputenc}
\usepackage{array}
\usepackage{longtable}
\usepackage{calc}
\usepackage{multirow}
\usepackage{hhline}
\usepackage{ifthen}
\usepackage[spanish]{babel}
%\usepackage{indentfirst}
\usepackage{fancyhdr}
\usepackage{graphicx,import}
\usepackage{latexsym}
\usepackage{wrapfig,lipsum,booktabs}
\usepackage{lastpage}
\usepackage{algorithm}
\usepackage[noend]{algpseudocode}

\usepackage[colorlinks=true, linkcolor=blue]{hyperref}
\def\inputGnumericTable{}
\usepackage{calc}
\usepackage[parfill]{parskip}
\usepackage{listings}
\usepackage{tikz}
\lstset
{
    language=C++,
    basicstyle=\ttfamily,
    keywordstyle=\color{blue}\ttfamily,
    stringstyle=\color{red}\ttfamily,
    commentstyle=\color{green}\ttfamily,
    morecomment=[l][\color{magenta}]{\#}
}

\tikzstyle{vertex}=
[
    auto=left,
    circle,
    draw=black,
    minimum size=20pt,
    inner sep=0pt
]


\sloppy

\hypersetup
{
    pdfstartview=
    {
        FitH \hypercalcbp
        {
            \paperheight-\topmargin-1in-\headheight
        }
    },
}
\parskip=5pt

\let\olditemize\itemize
\def\itemize{\olditemize\itemsep=0pt}

\title{Algoritmos y Estructuras de Datos III\\Resumen}
\author{amildie}
\date{\vspace{-5ex}}

\begin{document}
\renewcommand*\contentsname{\empty}

\maketitle
\tableofcontents

\newpage
\texttt{---- PRUEBA DE GRAFOS ----}

\raggedright
  \bigskip
  \begin{center}
  \begin{tikzpicture}
  \node[vertex] (n1) at (1,1)  {1};
  \node[vertex] (n2) at (0,0)  {2};
  \node[vertex] (n3) at (2,0)  {3};
  \node[vertex] (n4) at (-1,-1)  {4};
  \node[vertex] (n5) at (1,-1)  {5};
  %\node[vertex] (n2) at (4,3)  {1};
  

  \draw(n1) -- (n2);
  \draw(n2) -- (n5);
  \draw(n2) -- (n4);
  \draw(n1) -- (n3);
  
  %\foreach \from/\to in {n1/n2,n1/n3,n1/n5,n2/n1,n2/n3,n2/n4,n2/n5,n2/n6,n3/n1,n3/n2,n3/n5,n3/n6,n4/n2,n4/n5,n4/n6,n5/n1,n5/n2,n5/n3,n5/n4,n5/n6,n6/n2,n6/n3,n6/n4,n6/n5}
  %\draw (\from) -- (\to);

  \end{tikzpicture}
  \end{center}

  \texttt{---- FIN PRUEBA DE GRAFOS ----}


\newpage
\section{Complejidad Computacional}

Dada una funci\'on $g(n)$, denominamos como $\mathcal{O}(g(n)})$ al conjunto de funciones $f(n)$ que cumplen que:

\begin{center}
$\exists$ $c, n_0$ tal que $0 \leq f(n) \leq c * g(n)$, $\forall n \geq n_0$  
\end{center}

\begin{figure}[htb]
    \centering
    \newpage
\section{Complejidad Computacional}

Dada una funci\'on $g(n)$, denominamos como $\mathcal{O}(g(n)})$ al conjunto de funciones $f(n)$ que cumplen que:

\begin{center}
$\exists$ $c, n_0$ tal que $0 \leq f(n) \leq c * g(n)$, $\forall n \geq n_0$  
\end{center}

\begin{figure}[htb]
    \centering
    \newpage
\section{Complejidad Computacional}

Dada una funci\'on $g(n)$, denominamos como $\mathcal{O}(g(n)})$ al conjunto de funciones $f(n)$ que cumplen que:

\begin{center}
$\exists$ $c, n_0$ tal que $0 \leq f(n) \leq c * g(n)$, $\forall n \geq n_0$  
\end{center}

\begin{figure}[htb]
    \centering
    \input{complej.pdf_tex}
\end{figure}

Es decir $\mathcal{O}(g(n)})$ es un conjunto de funciones $f(n)$ que, a partir de cierto valor $n_0$, van a estar acotadas superiormente por $c * g(n)$. M\'as formalmente:

\begin{center}
$f(n) \in \mathcal{O}(g(n)}) \iff \exists$ $c, n_0$ tal que $0 \leq f(n) \leq c * g(n)$, $\forall n \geq n_0$  
\end{center}

Por ejemplo, si $f(n) = n^2 + n + 1$ y $g(n) = n^2$ podemos decir que $f(n) \in \mathcal{O}(g(n)})$, ya que si $c = 10^{999}$ y $n_0 = 10^{20}$ es f\'acil ver que se cumple que $0 \leq n^2 + n + 1 \leq 10^{999} * n^{2}$, $\forall n \geq 10^{20}$. Generalmente esto se describe diciendo que \emph{``f es $n^{2}$''}.
\end{figure}

Es decir $\mathcal{O}(g(n)})$ es un conjunto de funciones $f(n)$ que, a partir de cierto valor $n_0$, van a estar acotadas superiormente por $c * g(n)$. M\'as formalmente:

\begin{center}
$f(n) \in \mathcal{O}(g(n)}) \iff \exists$ $c, n_0$ tal que $0 \leq f(n) \leq c * g(n)$, $\forall n \geq n_0$  
\end{center}

Por ejemplo, si $f(n) = n^2 + n + 1$ y $g(n) = n^2$ podemos decir que $f(n) \in \mathcal{O}(g(n)})$, ya que si $c = 10^{999}$ y $n_0 = 10^{20}$ es f\'acil ver que se cumple que $0 \leq n^2 + n + 1 \leq 10^{999} * n^{2}$, $\forall n \geq 10^{20}$. Generalmente esto se describe diciendo que \emph{``f es $n^{2}$''}.
\end{figure}

Es decir $\mathcal{O}(g(n)})$ es un conjunto de funciones $f(n)$ que, a partir de cierto valor $n_0$, van a estar acotadas superiormente por $c * g(n)$. M\'as formalmente:

\begin{center}
$f(n) \in \mathcal{O}(g(n)}) \iff \exists$ $c, n_0$ tal que $0 \leq f(n) \leq c * g(n)$, $\forall n \geq n_0$  
\end{center}

Por ejemplo, si $f(n) = n^2 + n + 1$ y $g(n) = n^2$ podemos decir que $f(n) \in \mathcal{O}(g(n)})$, ya que si $c = 10^{999}$ y $n_0 = 10^{20}$ es f\'acil ver que se cumple que $0 \leq n^2 + n + 1 \leq 10^{999} * n^{2}$, $\forall n \geq 10^{20}$. Generalmente esto se describe diciendo que \emph{``f es $n^{2}$''}.
\newpage
\section{Programaci\'on Din\'amica}

\subsection{Subestructura Optima}

La \emph{``subestructura \'optima''} es una propiedad que pueden exhibir algunos problemas. Se dice que un problema tiene subestructura \'optima si el mismo cumple que una soluci\'on \'optima del mismo puede ser construida a partir de soluciones \'optimas de sus subproblemas.

Un ejemplo de esto es el problema del camino m\'as corto. Supongamos dos puntos $A$ y $B$, y un camino $w$ que es el m\'as corto entre ellos. Para cualquier par de puntos $A'$  y $B'$ dentro de $w$, el camino m\'as corto $w'$ entre ellos est\'a necesariamente contenido adentro de $w$.

\subsection{Soluciones Sobrepuestas}

Al igual que la subestructura \'optima, un problema tiene esta caracter\'istica cuando sub subproblemas comparten soluciones entre ellos. Un ejemplo cl\'asico de este fen\'omeno es el problema de calcular el $n$-\'esimo n\'umero de Fibonacci.

La ecuaci\'on recursiva para calcularlo es $f(n) = f(n-1) + f(n-2)$. Supongamos entonces que queremos calcular $f(5)$. Voy a tener que calcular $f(4)$ y $f(3)$. Pero para calcular $f(4)$ voy a tener que calcular $f(3)$ y $f(2)$. Es decir, voy a tener que calcular $f(3)$ m\'as de una vez.

Uno podr\'ia pensar que ser\'ia una buena idea cachear cada n\'umero de fibonacci calculado para no tener que recalcularlo m\'as de una vez. Esta t\'ecnica se llama \emph{``memoization''}\footnote{S\'i, \emph{memoization}, sin r.}.  

\subsection{Programaci\'on Din\'amica}

Cuando un problema exhibe tanto subestructura \'optima como soluciones sobrepuestas, es candidato a poseer un algoritmo para solucionarlo que emplee una t\'ecnica de desarrollo de algoritmos llamada \emph{``programaci\'on din\'amica''}.

Si un problema puede ser solucionado combinando soluciones no sobrepuestas de sus subproblemas, esta estrategia se llama \emph{``divide \& conquer''}. Es por eso que mergesort, por ejemplo, no es un problema de programaci\'on din\'amica.

\newpage
\subsection{Ejemplos}

\subsubsection{Subsecuencia continua de suma m\'axima}


\noindent\fbox{%
    \parbox{\textwidth}{%
Dado un arreglo $A = \{-6, 2, -4, 1, 3, -1, 5, -1\}$, dicho arreglo tiene varias subsecuencias $s$ continuas, por ejemplo $s_1 = \{1, 3, -1\}$, $s_2 = \{-6\}$, $s_3 = \{-1, 5, -1\}$, etc. Cada una de estas subsecuencias tiene un valor $sum(s_i)$ que representa la suma de todos los elementos de la misma. Encontrar el valor de la subsecuencia de suma m\'axima.
    }%
}

Notar que este problema s\'olo es interesante cuando hay tanto n\'umeros negativos como positivos en el arreglo; ya que si fuesen todos positivos, la soluci\'on es simplemente devolver $sum(A)$, y si fuesen todos negativos, la soluci\'on es devolver el elemento m\'as chico de $A$.

Un ejemplo elemental de esto es, por ejemplo teniendo el arreglo $A = \{ 1, 2, 3, -100, 4, 5, 6\}$. Como la suma de cualquier secuencia continua que no tenga al $-100$ es bastante mayor a la suma de cualquier secuencia que lo tenga, es l\'ogico asumir que la soluci\'on no va a tener al $-100$. Hay dos secuencias continuas de que no lo tienen: $S_1 = \{ 1, 2, 3\}$ y $S_2 = \{ 4, 5, 6\}$. Ac\'a es trivial ver que $sum(S_2)$ es la respuesta al problema.

Pero en el arreglo $A = \{-2, -3, 4, -1, -2, 1, 5, -3\}$ deja de ser tan evidente que el valor buscado es $7$, la suma de la subsecuencia $\{4, -1, -2, 1, 5\}$.

Definimos el array $S$, donde $S_i$ representa a la suma m\'axima de todas las subsecuencias continuas de $A$ que tienen a $A_i$ como \'ultimo elemento. Es decir, $A_i$ tiene que estar, por lo que en cada paso nos interesa saber si vamos a preservar la suma que ven\'iamos armando desde antes o a arrancar con $A_i$ como el inicio de una una subsecuencia nueva. Es decir:

\[
S(i) = \left\{\begin{array}{lr}
    A[i] & \text{si } i = 0\\
    max\{S(i-1) + A[i], A[i] \} & \text{si } i > 0
    \end{array}\right
\]

Entonces el problema se reduce a armar a $S$ mientras vamos buscando el m\'aximo:

\begin{center}
\begin{minipage}{0.78\textwidth}
\begin{lstlisting}[frame=lrtb]
int sum(int* a, unsigned int n) {
  int max = numeric_limits<int>::min();
  int s[n];
  memset(s, -1, sizeof(s));
  for(int i = 0; i < n; ++i) {
    if (i == 0) {
      s[i] = a[i];
    } else {
      s[i] = std::max(a[i], a[i] + s[i-1]);
    }
    if (s[i] > max) {
      max = s[i];
    }    
  }
  return max;
}
\end{lstlisting}
\end{minipage}
\end{center}



\subsubsection{Subsecuencia no-continua estrictamente creciente m\'as larga}

\noindent\fbox{%
    \parbox{\textwidth}{%
Dado un arreglo $A = \{3, 2, 6, 4, 5, 1\}$, encontrar una subsecuencia del mismo estrictamente creciente de longitud m\'axima.
    }%
}

Por ejemplo, para el array del enunciado la respuesta es $A = \{2, 4, 5\}$.

Similarmente al problema anterior, vamos a definir un vector de vectores $S$, donde $S_i$ es la subsecuencia de $A$ que termina en $A_i$. 

\[
S_i = \left\{\begin{array}{lr}
    \{ A_i \} & \text{si } i = 0\\
    max\{S_j \textnormal{ tal que } j < i \textnormal{ y } A_j < A_i\} + A_i & \text{si } i > 0
    \end{array}\right
\]

%Es decir, para calcular la subsecuencia que va a estar en $S_i$, lo que hacemos es fijarnos en todas las subsecuencias que calculamos antes. De todas esas, buscamos las de longitudes m\'aximas. Y de esas, seleccionamos alguna que tenga como \'ultimo elemento a algo m\'as chico que $A_i$ y se lo appendeamos.

\begin{center}
\begin{minipage}{1.15\textwidth}
\begin{lstlisting}[frame=lrtb]
vector<int> lis(int* a, unsigned int n) {
  std::vector< std::vector<int> > L(n);
  L[0].push_back(a[0]);
  vector<int> res;

  for(int i = 1; i < n; ++i) {
    int maxLength = numeric_limits<int>::min();
    int maxIndex = 0;

    for(int j = i-1; j >= 0; --j) {
      if((int)L[j].size() > maxLength && L[j].back() < a[i]) {
        maxLength = L[j].size();
        maxIndex = j;
      }
    }
    
    std::vector<int> v;
    if(maxLength != numeric_limits<int>::min()) {
      v = L[maxIndex];      
    }
    v.push_back(a[i]);
    L[i] = v;
  }

  unsigned int maxLength = numeric_limits<unsigned int>::min();
  for(int i = 0; i < L.size(); ++i) {
    if(L[i].size() > maxLength) {
      res = L[i];
      maxLength = L[i].size();
    }
  }

  return res;
}
\end{lstlisting}
\end{minipage}
\end{center}


\newpage
\subsubsection{El problema de la mochila}

El \emph{``problema de la mochila''} es un problema c\'asico de optimizaci\'on combinatoria. Se trata de, dado un conjunto de items, cada uno con un determinado peso y valor, y una mochila que puede soportar hasta cierto peso, encontrar cuantas veces tengo que poner cada \'item para garantizar que la mochila contiene el m\'aximo valor posible.

Este problema tiene diferentes variantes, pero cuando s\'olo tengo un \'item de cada peso, se lo conoce como el \emph{``0-1 knapsack problem''}, que se define formalmente como:

\noindent\fbox{
\parbox{\textwidth}{

Dado un conjunto de $n$ items numerados de $1$ hasta $n$, cada uno con un peso $w_i$ y un valor $v_i$, y una m\'axima capacidad para la mochila $W$:
\[
\textrm{maximizar }\sum_{n=1}^{n} v_ix_i
\textrm{ cumpliendo que } \sum_{n=1}^{n} v_ix_i \leq W \textrm{ y con } x_i \in \{0, 1\}
\]

}
}

La siguiente soluci\'on utilizar programaci\'on din\'amica y corre en tiempo pseudopolinomial. Asumimos que los pesos $w[1], w[2], ..., w[n]$ y el peso m\'aximo de la mochila $W$ son enteros positivos.

Primero definimos una matriz $m[i, w]$ que representa el m\'aximo valor que podemos alcanzar usando hasta el \'item del \'indice $i$ teniendo un peso disponible de $w$. Claramente $m[0,w] = 0$, ya que el valor m\'aximo sin poner ning\'un \'item en la mochila es $0$.

Luego definimos la din\'amica para calcular $m[i,w]$ de la siguiente manera:

\[
m[i,w] = \left\{\begin{array}{lr}
    m[i-1, w] \textnormal{ si } w_i > w\\ 
    max\{ m[i-1,w], m[i-1, w-w_i] + v_i\} \textnormal{ si } w_i \leq w
    \end{array}\right
\]

Por ejemplo, si estamos viendo quinto elemento pueden pasar dos cosas:

\begin{itemize}
\item Que $w[5]$ sea m\'as grande que $w$. Esto significa que ya no nos queda espacio para meter a $w_5$, por lo que no lo metemos, dejando el valor de $m[5, w]$ igual al que ten\'iamos en $m[4,w]$.
\item Que $w[5]$ sea menor o igual a $w$. Esto significa que todav\'ia tenemos espacio suficiente para meter al quinto elemento en la mochila.

Pero que \emph{podamos} meter al quinto elemento no significa que \emph{debamos} hacerlo. Tenemos que elegir el m\'aximo entre \textbf{no} poner al quinto elemento ($m[i-1,w]$) y \textbf{s\'i} ponerlo ($m[i-1, w-w_5] + v_i$). 
\end{itemize}

Es decir, esta din\'amica va reduciendo continuamente la capacidad restante en la mochila llam\'andose recursivamente cada vez con menos espacio disponible, hasta que eventualmente se queda sin lugar.

Para un peso m\'aximo $W$, el valor de la soluci\'on est\'a dado en $m[n,W]$.

\newpage

La siguiente es una implementaci\'on en C++ de la din\'amica anteriormente enunciada. Notar que los arrays son indexados desde $0$ en lugar de $1$.

\begin{center}
\begin{minipage}{1.02\textwidth}
\begin{lstlisting}[frame=lrtb]
void knapsack(int v[], int w[], int n, int W)
{
  int m[2000][2000];
  
  for(int j = 0; j <= W; ++j)
  {
    m[0][j] = 0;
  }

  for(int i = 0; i <= n; ++i)
  {
    for(int j = 0; j <= W; ++j)
    {
      if(w[i] > j)
      {
        m[i][j] = m[i-1][j];
      }
      else
      {
        m[i][j] = max(m[i-1][j], m[i-1][j-w[i]] + v[i]);
      }
    }
  }

  std::cout << "Rta: " << m[n-1][W] << std::endl;
}
\end{lstlisting}
\end{minipage}
\end{center}

Ac\'a queda evidente que el tiempo de ejecuci\'on de la funci\'on depende de los dos ciclos anidados, donde uno va a iterar $n$ veces, siendo $n$ la cantidad de elementos y el otro va a iterar $W$ veces, siendo $W$ el peso m\'aximo de la mochila.

Esto nos deja a la complejidad temporal del algoritmo como $O(nW)$. Notar que la matriz construida tiene que ser de $n$ filas por $W$ columnas, por lo que la complejidad espacial tambi\'en es de $O(nW)$. Es decir, esta soluci\'on es pseudopolinomial en tiempo y espacio.

\newpage
\subsubsection{Subconjunto de suma 0}

Este problema es en realidad un caso especial del problema de la mochila:

\noindent\fbox{
\parbox{\textwidth}{

Dado un array $X = \{a_1, a_2, ..., a_n\}$ determinar si existe una subsecuencia no necesariamente continua de $A$ tal que la suma de todos sus elementos sea 0.

}
}

Por ejemplo, si $X = \{−7, 5, −2, 8, -3\}$ la respuesta es \emph{``s\'i''} ya que $5 - 2 - 3 = 0$.

Este problema lo podemos resolver de manera similar al de la mochila mediante memoization.

Primero ordenamos el array de forma ascendente: $X = \{−7, -3, −2, 5, 8\}$. Luego definimos una matriz booleana $Q$, donde $Q[i][s]$ es verdadero o falso si hay un subconjunto no vac\'io de $x_1$ hasta $x_i$ que suma $s$. Entonces para obtener una soluci\'on del problema tenemos que ver cu\'anto vale $Q[n][0]$.

Luego definimos a $A$ como la suma de todos los valores negativos de $X$ y a $B$ como la suma de todos los positivos. Claramente $Q[i][s]$ es falso si $s < A$ o si $s > B$. Simplemente no se pueden alcanzar esos valores, por lo que los valores de $i$ y de $s$ que nos interesan dentro del problema son $1 \leq i \leq n$ y $A \leq s \leq B$. Vamos a completar $Q$:

Primero seteamos el caso base: Cuando $i = 1$ el valor de $Q[i][s]$ es true s\'olo cuando el \'unico elemento que podemos agarrar es efectivamente igual a $s$:

\begin{algorithm}
\begin{algorithmic}[1]
\For{$A \leq s \leq B$}
  \If{$x_1 == s$}
    \State $Q[1][s] \gets true$
  \Else
    \State $Q[1][s] \gets false$
  \EndIf
\EndFor
\end{algorithmic}
\end{algorithm}

Luego completamos el resto de la tabla:

\begin{algorithm}
\begin{algorithmic}[1]
\For{$2 \leq i \leq n$}
  \For{$A \leq s \leq B$}
    \If{$x_1 == s$}
      \State $Q(1,s) \gets true$
    \Else
      \State $Q(1,s) \gets false$
    \EndIf
  \EndFor
\EndFor
\end{algorithmic}
\end{algorithm}
\newpage
\section{Grafos}

Un \emph{``grafo''} $G = (V, X)$ es un par de conjuntos. $V$ es un conjunto de \emph{``v\'ertices''} y $X$ es un conjunto de \emph{``ejes''}, que a su vez son pares no ordenados de los elementos de $V$. Vamos a definir a $n = |V|$ y a $m = |X|$.

Dados dos v\'ertices $v, w \in V$ se dice que $v$ y $w$ son \emph{``adyacentes''} si $\exists$  $e \in X$ tal que $e = (v, w)$. Un \emph{``multigrafo''} es un grafo en el que pueden haber varios ejes entre el mismo par de v\'ertices. Un \emph{``seudografo''} es un multigrafo donde pueden haber ejes que unan a un v\'ertice con si mismo.

El \emph{``grado''} de un v\'ertice $v$ es la cantidad de ejes incidentes a \'el. Se nota con $d(v)$ y da lugar a la siguiente propiedad: 

\begin{figure}[h]
\[ \sum_{i=1}^{n} d(v_i) = 2m \]
\caption{La suma de los grados de todos los v\'ertices de un grafo es igual a dos veces el n\'umero de aristas.}
\end{figure}

Un grafo se dice \emph{``completo''} si todos los v\'ertices son adyacentes entre si. Al grafo completo de $n$ v\'ertices se lo nota $K_n$.

Dado un grafo $G$, se denomina como el grafo \emph{``complemento''} de $G$ al grafo $\overline{G}$ que tiene los mismos v\'ertices que $G$, pero donde dichos v\'ertices s\'olo son adyacentes en $\overline{G}$ si no lo son en $G$.

\subsection{Caminos y distancia}

Un \emph{``camino''} en un grafo es una sucesi\'on de ejes $e_1, e_2, ..., e_k$ tal que los extremos de $e_i$ coinciden con uno de $e_{i-1}$ y con uno de $e_{i+1}$, para todo $i \in \[2, ..., k-1 \]$.

Cuando un camino no pasa dos veces por el mismo v\'ertice, se lo denomina \emph{``camino simple''}. Un \emph{``circuito''} es un camino que empieza y termina en el mismo v\'ertice. Cuando un circuito tiene 3 o m\'as v\'ertices se lo denomina \emph{``circuito simple''}.

La \emph{``longitud''} de un camino es la cantidad de v\'ertices por los que pasa. La \emph{``distancia''} entre dos v\'ertices $v$ y $w$ de un grafo se define como la longitud del camino m\'as corto entre ambos, y se nota con  $d(v,w)$. Para todo v\'ertice $v$, $d(v,v) = 0$. Si no existe camino entre $v$ y $w$, se dice que la distancia entre ambos es infinita.

Si un camino $P$ entre $v$ y $w$ tiene longitud $d(v, w)$ entonces $P$ es un camino simple. Es decir, la distancia m\'as corta entre dos v\'ertices no va a dar vueltas en ning\'un lado.

Notar que si $P$ es un camino entre $u$ y $v$ de longitud $d(u,v)$ y tenemos los puntos $z$ y $w$ que est\'an adentro de $P$, entonces $P_{zw}$ es un camino entre $z$ y $w$ de longitud $d(z,w)$, donde $P_{zw}$ es el subcamino interno de $P$ entre $z$ y $w$.

\subsection{Subgrafos y biparticidad}

Dado un grafo $G = (V,X)$, un \emph{``subgrafo''} del mismo es un grafo $H = (V',X')$ que cumple que $V' \subseteq V$ y que $X' \subseteq X \cap (V' \times V')$. Si se cumple que para todo $u,v \in V', (u,v) \in X \Longleftrightarrow (u,v) \in X'$ entonces $H$ es un subgrafo \emph{``inducido''}. Es decir, para cada par de v\'ertices de $H$, se preservan los mismos ejes que ten\'ian en $G$.


Un grafo se dice \emph{``conexo''} si existe un camino que conecta cada par de v\'ertices. Una \emph{``componente conexa''} de un grafo $G$ es un subgrafo conexo maximal de $G$.

Un grafo $G = (X, V)$ se dice \emph{``bipartito''} si existe una partici\'on $V_1,V_2$ de $V$ tal que todos los ejes de $G$ tienen un extremo en $V_1$ y otro en $V_2$. Si todo v\'ertice de $V_1$ es adyacente a todo v\'ertice de $V_2$ entonces el $G$ es \emph{``bipartito completo''}. Un grafo es bipartito si y s\'olo si no tiene circuitos simples de longitud impar.

\subsection{Isomorfismo}

Dos grafos $G = (V,X)$ y $G' = (V',X')$ se dicen \emph{``isomorfos''} si existe una funci\'on biyectiva $f:V\rightarrow V'$ tal que para todo $v,w \in V$:

\begin{figure}[h]
\[ (v,w) \in X \Longleftrightarrow (f(v), f(w)) \in X' \]
\end{figure}

Si dos grafos son isomorfos, entonces:

\begin{itemize}
\item tienen el mismo n\'umero de v\'ertices
\item tienen el mismo n\'umero de ejes
\item tienen el mismo n\'umero de componentes conexas
\item $\forall$ $k, 0 \leq k \leq n-1$, tienen el mismo n\'umero de v\'ertices de grado $k$
\item $\forall$ $k, 1 \leq k \leq n-1$, tienen el mismo n\'umero de caminos simples de longitud $k$
\end{itemize}

No se conoce un algoritmo de tiempo polinomial para detectar si dos grafos son isomorfos, ni tampoco es NP-completo, pero en la pr\'actica hay maneras de resolverlo eficientemente.

\newpage
\subsection{Digrafos}

Un \emph{``grafo orientado''} (o \emph{``digrafo''}) es un grafo $G=(V,X)$ en el cual los ejes de $X$ tienen una direcci\'on. Para cada v\'ertice $v$ se define a su \emph{``grado de entrada''} ($d_{in}(v)$) como a la cantidad de ejes en $X$ que llegan a $v$. Es decir, que lo tienen como segundo elemento. El \emph{``grado de salida''} es lo mismo, pero con las aristas que salen del mismo.

Un \emph{``camino orientado''} es un digrafo es una sucesi\'on de ejes $e_1, e_2, ..., e_k$ ta que el primer elemento del par $e_i$ coincide con el segundo de $e_{i-1}$ y el segundo elemento de $e_i$ coincide con el primero de $e_{i+1}$, para todo $i = 2, ..., k-1$.

Un \emph{``circuito orientado''} en un digrafo es un camino orientado que comienza y termina en el mismo v\'ertice. Un digrafo se dice \emph{``fuertemente conexo''} si para todo par de v\'ertices $v,w$ existe un camino orientado de $v$ a $w$ y otro de $w$ a $v$.

\subsection{Grafo de L\'ineas}

...
\newpage
\section{Arboles}

Un \emph{\'arbol} es un grafo conexo sin circuitos simples. Dado $G=(V,X)$ las siguientes afirmaciones son equivalentes:

\begin{enumerate}
\item $G$ es un \'arbol.
\item $G$ es un grafo sin circuitos simples, pero si se le agrega una arista $e$ a $G$ tenemos un grafo con exactamente un circuito simple, y ese circuito contiene a $e$.
\item Existe exactamente un camino simple entre todo par de nodos.
\item $G$ es conexo, pero si se le quita una cualquier arista queda disconexo.
\end{enumerate}

Sea $G=(V,X)$ un grafo conexo y $e \in X$, $G-e$ es conexo si y s\'olo si $e$ pertenece a un circuito simple de $G$. Dentro de un \'arbol, definimos como \emph{hoja} a un nodo de grado 1. Todo \'arbol no trivial tiene al menos 2 hojas.

\subsection{Propiedades de los \'arboles}

\begin{itemize}
\item Si $G$ un \'arbol, entonces $m=n-1$.
\item Si $G = (V,X)$ con $c$ componentes conexas, entonces $m \geq n-c$.
\item Si $G = (V,X)$ con $c$ componentes conexas y sin circuitos simples, entonces $m=n-c$.
\end{itemize}

\subsection{Arboles enraizados}

En un \'arbol no dirigido podemos definir un nodo cualquiera como ra\'iz. El \emph{nivel} de un v\'ertice de un \'arbol es la distancia de ese v\'ertice a la raiz. La \emph{altura} $h$ de un \'arbol es la longitud desde la raiz hasta el v\'ertice m\'as lejano. 

Un \'arbol se dice \emph{``$m$-ario''} (con $m \geq 2$) si todos sus v\'ertices salvo las hojas y la raiz tienen grado a lo sumo $m+1$ y la raiz a lo sumo $m$.

Un \'arbol se dice \emph{``balanceado''} si todas sus hojas est\'an a nivel $h$ o $h-1$. Si todas est\'an a nivel $h$, se dice que es \emph{``balanceado completo''}.

Los nodos \emph{``internos''} de un \'arbol son aquellos que no son hojas ni ra\'iz.

\begin{itemize}
\item Un \'arbol $m$-ario de altura $h$ tiene a lo sumo $m^h$ hojas. 
\item Un \'arbol $m$-ario de altura $h \geq 1$ y balanceado completo tiene exactamente $m^h$ hojas.
\item Un \'arbol $m$-ario con $l$ hojas tiene $h \geq \lceil log_{m}l \rceil$.
\item Un \'arbol exactamente $m$-ario balanceado con $l$ hojas tiene $h = \lceil log_{m}l \rceil$.
\end{itemize}

\subsection{Recorrido de \'arboles}
\subsubsection{DFS}

\emph{``Depth First Search''} es un algoritmo de b\'usqueda en grafos. Lo que hace es arrancar desde un v\'ertice $n$ cualquiera y fijarse si dicho v\'ertice es el que estamos buscando. Si no lo es, marcarlo como visitado, pedir el listado de todos sus vecinos y llamarse recursivamente ($DFSr$) en cada uno de ellos.
\vspace{8px}

\begin{algorithm}
\begin{algorithmic}[1]
\Function{DFS}{$G = (V, E)$}
  \ForAll{$v \in V$}
    \State $visitV_v \gets false$
  \EndFor
  \ForAll{$e \in E$}
    \State $visitE_e \gets false$
  \EndFor
  \ForAll{$v \in V$}
    \If{$visitV_v = false$}
      \State $DFS(v)$
    \EndIf
  \EndFor
\EndFunction
\end{algorithmic}
\end{algorithm}

\begin{algorithm}
\begin{algorithmic}[1]
\Function{DFS}{$v$}
  \State $visited_v \gets true$
  \ForAll{$e = (w,u) \in E$ \textbf{where} $w = v$ \textbf{and} $visitE_e = false$}
    \If{$visitV_w = false$}
      \State $DFS(w)$
    \EndIf
  \EndFor
\EndFunction
\end{algorithmic}
\end{algorithm}

Cuando el algoritmo llega a una hoja hace backtracking hasta el \'ultimo v\'ertice revisado con hijos todav\'ia sin revisar.

Como todos los algoritmos de grafos, la complejidad depende de ciertos factores de la implementaci\'on. En la funci\'on $DFS(v)$ estoy recorriendo todos los ejes de $E$, buscando los que salen de $v$. Si hago este llamado para cada v\'ertice $v$, la complejidad del algoritmo me va a quedar $O(n * m)$.

No obstante, no queremos recorrer ejes que ya sabemos que ya revisamos. Es verdad que, como mucho, podr\'iamos iterar $m$ veces en la linea $3:$, pero en una buena implementaci\'on de DFS no tendr\'iamos que nunca pasar por el mismo eje dos veces.

Es decir, por cada v\'ertice, solo vamos a ver sus ejes no visitados. Si queremos revisar todos los v\'ertices y todos los ejes de cada uno, nos temina quedando una complejidad de $O(n + m)$.

\newpage
\subsubsection{BFS}

\emph{``Bradth First Search''} es muy similar a DFS pero, en lugar de realizar backtracking poniendo todos los vecinos de cada v\'ertice en un stack, pone a todos los vecinos de cada v\'ertice en una queue.
\vspace{8px}

\begin{center}
\begin{minipage}{0.78\textwidth}
\begin{lstlisting}[frame=lrtb]
void BFS(Graph &g, int n) {
  queue<int> q;
  bool checked[g.n];
  memset(checked, 0, sizeof(checked));
  q.push(0);

  while(!q.empty()) {
    int t = q.front();
    q.pop();

    if(checked[t]) continue;
    if(t == n) {
      cout << "found!" << endl;
      return;
    }

    checked[t] = 1;
    vector<int> neigh = g.getAdyacents(t); 
    for(int i = 0; i < neigh.size(); ++i) {
      if(!checked[neigh[i]]) {
        q.push(neigh[i]);
      }
    }
  }

}
\end{lstlisting}
\end{minipage}
\end{center}

Al igual que con DFS, BFS s\'olo tiene sentido en grafos conexos y tiene complejidad $O(n)$.

\newpage
\subsection{Arbol generador}

Dado un grafo conexo $G$, un \emph{``\'arbol generador''} de $G$ es un subgrafo de $G$ que es un \'arbol y tiene el mismo conjunto de v\'ertices. Si los ejes del grafo tienen peso, entonces cada \'arbol generador va a tener una determinada \emph{``longitud''}, que se define como la suma de todos los pesos de sus ejes. Cuando dicha longitud es m\'inima, decimos que tenemos un \emph{``\'arbol generador m\'inimo''}.

\raggedright
  \bigskip
  \begin{center}
\begin{tikzpicture}[shorten >=1pt, auto, node distance=3cm,
   node_style/.style={circle},
   edge_style/.style={draw=black},
   edge_styly/.style={draw=green, ultra thick}]




\node[vertex] (n2) at (3,4)  {B};
\node[vertex] (n3) at (5,4)  {C};
\node[vertex] (n4) at (7,4)  {D};

\node[vertex] (n9) at (5,2)  {I};
\node[vertex] (n1) at (1,2)  {A};
\node[vertex] (n5) at (9,2)  {E};

\node[vertex] (n6) at (7,0)  {F};
\node[vertex] (n7) at (5,0)  {G};
\node[vertex] (n8) at (3,0)  {H};
    
\draw[edge_styly]  (n1) edge node{4} (n2);
\draw[edge_style]  (n1) edge node{8} (n8);
\draw[edge_style]  (n2) edge node{12} (n8);
\draw[edge_styly]  (n2) edge node{8} (n3);
\draw[edge_style]  (n8) edge node{6} (n9);
\draw[edge_styly]  (n8) edge node{1} (n7);
\draw[edge_styly]  (n7) edge node{3} (n6);
\draw[edge_style]  (n7) edge node{5} (n9);
\draw[edge_styly]  (n9) edge node{3} (n3);
\draw[edge_styly]  (n3) edge node{6} (n4);
\draw[edge_styly]  (n3) edge node{4} (n6);
\draw[edge_style]  (n4) edge node{13} (n6);
\draw[edge_styly]  (n4) edge node{9} (n5);
\draw[edge_style]  (n6) edge node{10} (n5);


  %\draw(n1) -- (n2);
  %\draw(n1) -- (n8);

  %\draw(n2) -- (n8);
  %\draw(n2) -- (n3);
  
  %\foreach \from/\to in {n1/n2,n1/n3,n1/n5,n2/n1,n2/n3,n2/n4,n2/n5,n2/n6,n3/n1,n3/n2,n3/n5,n3/n6,n4/n2,n4/n5,n4/n6,n5/n1,n5/n2,n5/n3,n5/n4,n5/n6,n6/n2,n6/n3,n6/n4,n6/n5}
  %\draw (\from) -- (\to);

  \end{tikzpicture}
  \end{center}

\newpage
\subsubsection{Algoritmo de Kruskal}

El algoritmo de Kruskal es un algoritmo greedy que lo que hace es ir seleccionando las aristas que van a conformar el AGM una por una, arrancando de la de menor peso y subiendo progresivamente. Si el grafo no es conexo, forma un \emph{``bosque generador m\'inimo''}.

\begin{algorithm}
\caption{Algoritmo de Kruskal}
\begin{algorithmic}[1]
\State $AGM \gets \emptyset$
\ForAll{$v \in V$}
  \State makeDisjointSet($V$)
\EndFor
\State sortByWeightASC($E$)
\ForAll{$(v_1, v_2)  \in E$}
  \If {find($v_1$) \neq find($v_2$)}
    \State $AGM \gets AGM \cup (v_1, v_2)$
    \State union($v_1, v_2$)
  \EndIf
\EndFor
\Return $AGM$
\end{algorithmic}
\end{algorithm}

\subsubsection*{Explicaci\'on}

\begin{enumerate}
\item [1:] Empieza declarando un $AGM$ vac\'io. Este va a ser un conjunto de ejes, que es lo que va a devolver el algoritmo.
\item [2: y 3:] Crea un conjunto disjunto para cada v\'ertice de $V$.
\item [4:] Ordena ascendentemente a todos los ejes por peso.
\item [5:] Empieza a iterar por todos los ejes $(v_1, v_2)$ de $E$
  \begin{enumerate}
  \item [6:] Para cada eje, se fija que $v_1$ y $v_2$ est\'en en diferentes conjuntos del disjoint set previamente creado.
    \begin{enumerate}
    \item [7:] Si no lo est\'an, agrega a $(v_1, v_2)$ al $AGM$.
    \item [8:] Y los une adentro del disjoint set.
    \end{enumerate}
  \end{enumerate}
\end{enumerate}

\subsubsection*{Complejidad}

En cada iteraci\'on del algoritmo (5:) necesitamos poder obtener el eje no todav\'ia insertado en el AGM en O(1). Esto se puede lograr f\'acilmente ordenando todos los $m$ ejes por peso ascendentemente usando alg\'un sort por comparaci\'on, lo cual se puede hacer en O(m log m) (4:).

Los pasos 6: y 8: (find y union) se pueden hacer en O(log m). Como los estoy haciendo en cada iteraci\'on del algoritmo, y como voy a iterar n-1 veces, la complejidad del mismo termina siendo O(n (2 log m)) = O(n log m).

\newpage
\subsubsection{Algoritmo de Prim}

Al igual que el algoritmo de Kruskal, el algoritmo de Prim nos permite encontrar un \'arbol generador m\'inimo para un grafo. Este es un pseudoc\'odigo del mismo:

\begin{algorithm}
\caption{Algoritmo de Prim}
\begin{algorithmic}[1]
\State $AGM \gets \emptyset$
\State $V \gets \{ v \}$
\While{$|V| \neq n$}
  \State $(v_1, v_2) \gets$ buscarEje($V$)
  \State $AGM \gets AGM \cup \{ (v_1, v_2) \}$
  \State $V \gets V \cup \{ v_2 \}$
\EndWhile
\Return $AGM$
\end{algorithmic}
\end{algorithm}

\subsubsection*{Explicaci\'on}

A diferencia del anterior, el algoritmo de Pimm hace crecer al AGM desde un v\'ertice raiz arbitrario, y agrega un eje nuevo en cada iteraci\'on. El algoritmo termina cuando todos los v\'ertices est\'an en el AGM. El pseudoc\'odigo funciona de la siguiente manera:

\begin{enumerate}
\item [1:] Declaramos el AGM que vamos a devolver como vac\'io. Al igual que en el algorimto de Kruskal, el AGM devuelto es un conjunto de ejes.
\item [2:] Creamos un conjunto $V$ de v\'ertices que s\'olo tiene a un v\'ertice arbitrario $v$, que puede ser cualquiera.
\item [3:] Vamos a iterar hasta que $V$ tenga a todos los v\'ertices del grafo.
\item [4:] Obtenemos un eje $(v_1, v_2)$ usando una funci\'on llamada buscarEje. El eje devuelto tiene que cumplir las siguientes propiedades:
  \begin{itemize}
  \item $v_1 \in V$
  \item $v_2 \not \in V$
  \item El peso de $(v_1, v_2)$ tiene que ser m\'inimo.
  \end{itemize}
\item [5:] Agregamos a $(v_1, v_2)$ a nuestro AGM.
\item [6:] Agregamos a $v_2$ a $V$.
\end{enumerate}

\subsubsection*{Complejidad}

L\'ogicamente toda la complejidad del algoritmo est\'a en la funci\'on buscarEje, la cual hace que el algoritmo sea greddy. La complejidad del mismo depende de las estructura utilizada para implementar la funci\'on buscarEje.

Si utilizo una cola de prioridad implementada sobre un min-heap binario esta funci\'on cuesta O(log m). Como este paso se va a tener que realizar n veces ya que est\'a adentro del while, el costo total del algoritmo es de O(n log m).
\newpage
\section{Camino M\'inimo}

Sea $G = (V, E)$ un grafo y $l: E \rightarrow {\rm I\!R}$ una funci\'on de peso para los ejes de $G$, definimos como el \emph{``peso''} de un camino $C$ entre dos nodos $v$ y $w$ como la suma de los pesos de los ejes del camino:

\begin{equation}
l(C) = \sum_{e \in C} l(e)
\end{equation}

Un \emph{``camino minimo''} $C^0$ entre $v$ y $w$ es un camino tal que $l(C^0) = min \{ l(C), \forall \textrm{$ C$ camino entre $v$ y $w$}  \}$. Esto significa que no necesariamente tiene que ser \'unico. Dado un grafo $G$, se pueden definir 3 variantes de problemas sobre caminos m\'inimos:

\begin{enumerate}
\item \textbf{Unico origen - \'unico destino}: determinar un camino m\'inimo entre dos v\'ertices $v$ y $w$.
\item \textbf{Unico origen - m\'ultiples destinos}: determinar un camino m\'inimo desde un v\'ertice $v$ al resto de los v\'ertices de $G$
\item \textbf{M\'ultiples or\'igenes - m\'ultiples destinos}: Determinar un camino m\'inimo entre todo par de v\'ertices de $G$.
\end{enumerate}

Si el grafo $G$ no contiene ciclos con peso negativo (o contiene alguno pero no es alcanzable desde $v$) entonces el problema sigue estando bien definido, aunque algunos caminos pueden tener longitud negativa. Sin embargo, si si $G$ tiene alg\'un ciclo con peso negativo alcanzable desde $v$, el concepto de camino m\'inimo deja de estar bien definido.

Un camino m\'inimo no puede contener circuitos. Tambi\'en es importante notar que un camino m\'inimo exhibe la propiedad de subestructura \'optima, ya que dados dos puntos $v'$ y $w'$ que est\'an adentro del camino m\'inimo entre $v$ y $w$, el subcamino entre estos dos puntos tambi\'en es un camino m\'inimo entre ambos.

\newpage
\subsection{Algoritmo de Dijkstra (1 a n)}

Dado $G = (V, E)$ y grafo, $l: E \rightarrow {\rm I\!R}$ una funci\'on que asigna a cada eje un cierto peso y $v$ un v\'ertice de $G$, calcular los caminos m\'inimos desde $v$ al resto de los v\'ertices. El algoritmo de Dijkstra asume que los pesos de los ejes son positivos.

\begin{algorithm}
\begin{algorithmic}[1]
\Function{Dijkstra}{$G = (V, E)$}
  \State $prev[A] \gets -1$
  \State $dist[A] \gets 0$
  \State $pq.add(<dist[A], A>)$
  \ForAll{$v \in V, v \neq A$}
    \State $prev[v] \gets -1$
    \State $dist[v] \gets \infty$
    \State $pq.add(<dist[v], v>)$
  \EndFor
  \While{$!pq.empty()$}
    \State $t \gets pq.pop()$
    \ForAll{$u \in t.second.neigh()$}
      \State $alt \gets dist[u] + length(t.first, u)$
      \If{$alt < dist[u]$}
        \State $prev[u] \gets w$
        \State $dist[u] \gets alt$
        \State $pq.decreaseKey(<dist[u], u>)$
      \EndIf
    \EndFor
  \EndWhile
  \State \textbf{return} $prev$
\EndFunction
\end{algorithmic}
\end{algorithm}

\subsubsection*{Analisis}

Este algoritmo toma un grafo $G$ y devuelve un array de v\'ertices llamado $prev$, que consiste en los caminos m\'inimos en el grafo desde el v\'ertice $A$.

\begin{enumerate}
\item [2:] Seteamos al v\'ertice previo de $A$ como $-1$, ya que es el v\'ertice por el que comienzan todos los caminos m\'inimos.
\item [3:] L\'ogicamente la distancia m\'inima de $A$ hacia $A$ es 0. Esto se asume porque $G$ no tiene ciclos negativos.
\item [4:] Declaramos como $pq$ a una cola de prioridad. Ac\'a vamos a guardar tuplas $<d, w>$, donde $d$ es la distancia m\'inima desde el v\'ertice $A$ hasta el v\'ertice $w$. Hacemos esto para poder obtener el v\'ertice de menor distancia a $A$ en O(log n). Ac\'a hay informaci\'on redundante (las distancias m\'inimas ya se est\'an guardando en $dist$) pero es beneficiosa para la complejidad del algoritmo.
\item [5 a 8:] Luego vamos a empezar a iterar todos los v\'ertices de $G$ seteando la informaci\'on necesaria para cada uno. Esto significa setear su v\'ertice previo en $-1$ y su distancia m\'inima hacia $A$ en infinito. Este proceso es O(n).
\item [9:] Ahora comienza el ciclo principal del algoritmo. En cada iteraci\'on del while vamos a visitar el v\'ertice de menor distancia hacia $A$ que encontremos en $pq$. El algoritmo termina cuando $pq$ est\'a vac\'ia, es decir, visitamos a todos los v\'ertices. Vale destacar que el primer v\'ertice que visitamos es $A$.
    \begin{enumerate}
    \item [10:] Guardamos en una tupla $t$ al v\'ertice que sacamos de $pq$. Hay que tener en cuenta que en $t.first$ vamos a tener la distancia m\'inima del v\'ertice que est\'a en $t.second$ hasta $A$.
    \item [11:] Vamos a iterar a todos los vecinos $u$ de $t.second$.
        \begin{enumerate}
        \item [12:] Definimos a la variable $alt$ como la suma de la distancia m\'inima de $u$ a $A$ m\'as la distancia de $u$ a $t.first$ (el v\'ertice que estamos visitando).
        \item [13:] \¿Es esta nueva distancia menor a la distancia que ten\'iamos calculada antes?. Es decir, \¿si pasamos por $t.first$ para llegar desde $A$ a $u$, llegamos m\'as r\'apido que si no pasamos?. Si la respuesta es \textbf{s\'i}, entonces mejoramos la distancia previamente calculada para $u$ y tenemos que actualizar todo:
            \begin{enumerate}
            \item [14:] Entonces para llegar a $u$ vamos a querer pasar por $t.first$.
            \item [15:] Actualizamos la distancia m\'inima de $A$ hasta $u$.
            \item [16:] Hacemos lo mismo, pero en $pq$.
            \end{enumerate}
        \end{enumerate}
    \end{enumerate}
\end{enumerate}

\subsubsection*{Complejidad}

La complejidad del algoritmo est\'a dada por el ciclo principal del mismo que empieza en la l\'inea 9. Vamos a tener que iterar tantas veces como elementos haya en $pq$, donde inicialmente van a haber $n$ elementos.

En cada iteraci\'on vamos a tener que obtener el v\'ertice m\'as cercano a $A$ (10:), lo cual puede hacerse en O(log n) usando una cola de prioridad basada en un min-heap.


\newpage
\subsection{Algoritmo de Bellman-Ford (1 a n)}

Es un poco m\'as lento que el algoritmo de Dijkstra, pero admite ejes de pesos negativos. La idea general es similar a la del algoritmo de Dijkstra, pero un poco m\'as simple. 

Mientras Dijkstra obtiene el v\'ertice no visitado de menor peso en cada iteraci\'on, Bellman-Ford simplemente relaja todos los ejes $n-1$ veces. En cada iteraci\'on el n\'umero de v\'ertices con caminos m\'inimos bien calculados se va incrementando, hasta que ya no se pueden mejorar.

\begin{algorithm}
\begin{algorithmic}[1]
\Function{BellmanFord}{$G = (V, E)$}
  \ForAll{$v \in V, v \neq A$}
    \State $dist[v] \gets \infty$
    \State $prev[v] \gets -1$
  \EndFor
  \State $dist[A] \gets 0$
  \For{$i \gets 1 \textrm{ \textbf{hasta} } n-1$}
    \ForAll{$(u, v) \textrm{ con peso } w \in E$}
        \If{$dist[u] + w < dist[v]$}
            \State $dist[v] \gets dist[u] + w$
            \State $prev[v] \gets u$
        \EndIf
    \EndFor
  \EndFor
  \ForAll{$(u, v) \textrm{ con peso } w \in E$}  \Comment{\¿Hay ciclos negativos? \textbf{O(E)}}
    \If{$dist[u] + w < dist[v]$}
        \State \textbf{return} null
    \EndIf
  \EndFor
  \State \textbf{return} $prev$
\EndFunction
\end{algorithmic}
\end{algorithm}

\subsubsection*{Analisis}

\begin{enumerate}
\item [\textbf{2 a 4:}] Seteamos, para todos los v\'ertices que no son $A$, la distancia m\'inima en $\infty$ y a su v\'ertice previo como $-1$.
\item [\textbf{5:}] Seteamos la distancia m\'inima hacia $A$ (el v\'ertice inicial) en $0$.
\item [\textbf{6:}] Comenzamos la iteraci\'on principal del algoritmo. Vamos a ciclar, como mucho $n-1$ veces.
    \begin{enumerate}
    \item [\textbf{7:}] Vamos a iterar por todos los ejes del grafo.
        \begin{enumerate}
        \item [\textbf{8:}] Para cada eje $(u, v)$, checkeamos si nos conviene pasar por $u$ para llegar hasta $v$.
            \begin{enumerate}
            \item [\textbf{9 y 10:}] Si este es el caso, actualizamos los valores.
            \end{enumerate}
        \end{enumerate}
    \end{enumerate}
\item [\textbf{11 a 13:}] Checkeamos si tenemos un ciclo negativo. Si este es el caso, el problema deja de estar bien definido y devolvemos $null$.
\end{enumerate}

\subsubsection*{Complejidad}

En este caso es bastante evidente que la complejidad es O(n * m), ya que vamos a ciclar a lo sumo $n-1$ veces y, en cada uno de estos ciclos, vamos a relajar, como mucho, los $m$ ejes de $G$.

\newpage
\subsection{Algoritmo de Floyd-Warshall (n a m)}

El algoritmo de Floyd-Warshall compara todos los caminos posibles entre cada par de v\'ertices. Su tiempo de ejecuci\'on es $O(n^3)$.

Consideremos un grafo $G = (V, E)$ donde los v\'ertices de $G$ est\'an numerados de $1$ a $n$. Consideremos luego la funci\'on $shortestPath(i, j, k)$ que devuelve el costo del camino m\'inimo de $i$ hasta $j$, pero usando a los v\'ertices $\{1, 2, ..., k\}$ como puntos intermedios del camino.

Dada esa funci\'on, nuestro objetivo es el de calcular el costo del camino m\'inimo de cada $i$ hasta cada $j$ usando s\'olo los v\'ertices en $\{1, 2, ..., k+1\}$. Para cada par $i,j$, el camino m\'inimo debe ser, o:

\begin{enumerate}
\item Un camino que s\'olo usa los v\'ertices de $\{1, 2, ..., k\}$
\item Un camino que va desde $1$ hasta $k+1$, y de $k+1$ hasta $j$
\end{enumerate}

Sabemos que el camino m\'inimo de $i$ a $j$ que \'unicamente usa los v\'ertices $\{1, 2, ..., k\}$ est\'a definido por $shortestPath(i, j, k)$, y es claro que si existiera un mejor camino de $i$ hacia $k+1$ hasta $j$, entonces la longitud de ese camino ser\'ia la concatenaci\'on del camino m\'inimo de $i$ a $k+1$ (pasando por $\{1, 2, ..., k\}$) y el camino m\'inimo desde $\{k+1\}$ a $j$ (pasando por $\{1, 2, ..., k\}$).

Si $w(i, j)$ es el peso de eje entre los v\'ertices $i$ y $j$, podemos definir a $shortestPath(i, j, k)$ con la siguiente recursividad:


\textbf{Caso base:}\\
$shortestPath(i, j, 0) = w(i, j)$\\
\textbf{Paso recursivo:}
\[
shortestPath(i, j, k+1) = min \left\{\begin{array}{lr}
    shortestPath(i, j, k), \\
    shortestPath(i, k+1, k) + shortestPath(k+1, j, k)
    \end{array}\right\}
\]



\newpage
\subsection{Algoritmo de Dantzig}
\newpage
\section{Grafos Eulerianos y Hamiltonianos}
\newpage
\section{Planaridad}

Una \emph{``representaci\'on planar''} de un grafo $G$ es un conjunto de puntos en el plano que se corresponden con los v\'ertices de $G$, unidos por curvas que se corresponden con los ejes de $G$, sin que estas se crucen entre s\'i. Si un grafo admite una representaci\'on planar, decimos que el mismo es planar.

Dada una representaci\'on planar de un grafo, llamamos \emph{``regi\'on''} al conjunto de todos los puntos alcanzables desde un punto que no sea un v\'ertice o un eje, sin atravesar ning\'un v\'ertice o eje. Es por eso que toda representaci\'on planar de un grafo tiene exactamente una regi\'on de \'area infinita: la regi\'on exterior.

La \emph{``frontera''} de una regi\'on es el circuito que rodea a la regi\'on, y el \emph{``grado''} de la regi\'on es el n\'umero de ejes que tiene su frontera.

\begin{badidea}
\textbf{Propiedad:} $K_5$ es el grafo no-planar con el menor n\'umero de v\'ertices, y $K_{33}$ el que tiene el menor n\'umero de ejes.
\end{badidea}

\begin{badidea}
\textbf{Propiedad:} Si un grafo $G$ contiene sun subgrafo $G'$ no planar, entonces $G$ tampoco es planar.
\end{badidea}

\subsubsection{Teorema de Euler}

Si $G$ es un grafo conexo planar entonces cualquier representaci\'on planar de $G$ determina $r = m - n + 2$ regiones en el plano.

\begin{badidea}
\textbf{Colorario 1:} Si $G$ es conexo y planar con $|V| \geq 3$ $\Rightarrow |E| \leq 3 * |V| - 6$.
\end{badidea}

\begin{badidea}
\textbf{Colorario 2:} Si $G$ es conexo, planar y bipartito con $|V| \geq 3$ \\ $\Rightarrow$ $|E| \leq 2 * |V| - 4$.
\end{badidea}

\subsection{Subdivisiones y Homeomorfismo}

Subdividir un eje $e = (v,w)$ de un grafo $G$ consiste en agregar un v\'ertice nuevo $u \not\in V$ a $G$ y reemplazar al eje $e$ por dos ejes $e' = (v, u)$ y $e'' = (u, w)$. Un grafo $G'$ es una subdivisi\'on de otro grafo $G$ si $G'$ se puede obtener mediante sucesivas operaciones de subdivisi\'on sobre $G$. Dos grafos $G$ y $H$ se dicen \emph{``homeomorfos''} si hay un isomorfismo entre una subdivisi\'on de $G$ y una de $H$.

\begin{badidea}
\textbf{Propiedad 1:} Si $G'$ es una subdivisi\'on de $G$, entonces $G$ es planar si y s\'olo si $G'$ es planar.
\end{badidea}

\begin{badidea}
\textbf{Propiedad 2:} La planaridad es invariate bajo homeomorfismo.
\end{badidea}

\begin{badidea}
\textbf{Propiedad 3:} Si un grafo $G$ tiene un subgrafo que es homeomorfo a un grafo no-planar, entonces $G$ es no-planar.
\end{badidea}

\subsubsection{Teorema de Kuratowski}

Un grafo es planar si y s\'olo si no contiene ning\'un subgrafo homemomorfo a $K_5$ o a $K_{33}$.

\subsection{Contracciones}

La operaci\'on de \emph{``contracci\'on''} de un eje $e = (v, w)$ consiste en eliminar el eje del grafo y considerar sus extremos como un s\'olo v\'ertice $u \not\in V$, quedando como ejes incidentes a $u$ todos los ejes que eran incidentes a $v$ y a $w$.

Un grafo $G'$ es una contracci\'on de otro grafo $G$ si se puede obtener a partir de $G$ por sucesivas operaciones de contracci\'on. Se dice que \emph{``$G$ es contraible a $G'$''}.

\subsubsection{Teorema de Whitney}

$G$ es planar si y s\'olo si no contiene ning\'un subgrafo contraible a $K_{5}$ o a $K_{33}$.

\subsection{Algoritmo de Demoucron}

El siguiente algoritmo encuentra una represetaci\'on planar si existe, y si $G$ es no planar lo reconoce correctamente. El tiempo de ejecuci\'on es de $O(n^2)$, pero existen algoritmos que pueden determinar planaridad en tiempo lineal. Esto significa que el problema de decidir si un grafo es o no planar est\'a en P.

La idea del algoritmo es partir de una representaci\'on planar $R$ de un subgrafo $S$ de $G$, y empezar a expandirala iterativamente hasta obtener una representaci\'on planar de todo $G$ o hasta concluir que dicha representaci\'on no existe.

\subsubsection{Definiciones\footnote{A partir de ahora van a empezar a aparecer unas operaciones que pierden completamente el hilo de coherencia de las diapositivas. Especialmente la operaci\'on $G - R$, que vendr\'ia a ser algo como \emph{``restarle una representaci\'on planar a un grafo''}, lo cual adolece completamente de cualquier significado formal. Quien haya llegado hasta ac\'a va a tener que valerse m\'as de la intuici\'on para poder comprender las intenciones de quien hizo su mejor esfuerzo por armar las diapositivas. Tener especial cuidado con el concepto de \emph{``v\'ertice de contacto''}, el cual pertenece a una representaci\'on planar $R$ a veces y a una parte $p$ otras veces.}}


\begin{itemize}
\item Llamamos \textbf{\emph{``parte $p$ de $G$ relativa a $R$''}} a:
    \begin{itemize}
    \item Una componente conexa de $G - R$ junto con los ejes que la conectan a v\'ertices de $R$ (ejes colgantes). Notar que $R$ es una representaci\'on de un subgrafo $S$ de $G$.
    \item Un eje $e = (v,w)$ de $G - R$, con $v,w \in R$. Es decir, un eje suelto sin v\'ertices. 
    \end{itemize}

    Es decir, una parte es \emph{lo que queda} de un grafo al que le sacamos la representaci\'on planar de un subgrafo. Esta parte puede tener ejes colgantes que no est\'an conectados con ning\'un v\'ertice, e incluso una parte puede ser un eje suelto volando por ah\'i. Sep.

\item Dada una parte $p$ de $G$ relativa a $R$, llamamos \textbf{\emph{``v\'ertice de contacto''}} a un v\'ertice de $R$ incidente a un eje colgante de $p$.
\item Decimos que la representaci\'on $R$ es \textbf{\emph{``extensible''}} a una representaci\'on planar de $G$ si se puede obtener una representaci\'on planar de $G$ a partir de $R$. \emph{Extender} ac\'a es el proceso de ir agregandole cosas a una representaci\'on planar.
\item Una parte $p$ es \textbf{\emph{``dibujable''}} en una regi\'on $f$ de $R$ si existe una extensi\'on planar de $R$ en la que $p$ queda en $f$.
\item Una parte $p$ es \textbf{\emph{``potencialmente dibujable''}} en una regi\'on $f$ de $R$ si todo v\'ertice de contacto de $p$ pertenece a la frontera de $f$.
\item Llamamos $F(p)$ al conjunto de regiones de $R$ donde $p$ es potencialmente dibujable.
\end{itemize}

\subsubsection{Pseudoc\'odigo}














\begin{algorithm}
\begin{algorithmic}[1]
\Function{Demoucron}{$G = (V, E)$}
  \State \textbf{return} $orden$
\EndFunction
\end{algorithmic}
\end{algorithm}
\newpage
\section{Coloreo}
\newpage
\section{Matchings y Conjuntos Independientes}
\newpage
\section{Flujo en Redes}

Una \emph{``red''} $G = (V, E)$ es un grafo orientado conexo que tiene dos v\'ertices distinguidos: una \emph{``fuente''} $s$, con un grado de salida positivo y un \emph{``sumidero''} $t$ con un grado de entrada positivo.

Una \emph{``funci\'on de capacidades en la red''} es una funci\'on $c : E \rightarrow {\rm I\!R^{\geq 0}}$. La funci\'on de capacidad determina, para cada eje del grafo, la capacidad de transportar flujo que posee.

Un \emph{``flujo factible''} en una red $G = (V, E)$ con funci\'on de capacidad $c$ es una funci\'on $f : E \rightarrow {\rm I\!R^{\geq 0}}$ que verifica:

\begin{itemize}
\item $0 \leq f(e) \leq c(e)$ para todo eje $e \in E$. Es decir, el flujo factible de un eje no puede ser superior a su capacidad.
\item La \emph{``Ley de conservaci\'on de Flujo''}, que dice que dado un v\'ertice $v$, la suma de los flujos de los ejes que llegan a \'el es la misma que la suma de los flujos de los v\'ertices que salen de \'el. Formalmente:

\[
\forall v \in V - \{ s, t \} \textrm{ se cumple que} \sum_{e \in in(v)} f(e) = \sum_{e \in out(v)} f(e)
\]

donde:

\[
in(v) = \{ e \in E, e = (w \rightarrow v), w \in V \}
\]
\[
out(v) = \{ e \in E, e = (v \rightarrow w), w \in V \}
\]


\end{itemize}

El \emph{``valor del flujo''} es:

\[
F = \sum_{e \in in(t)} f(e) - \sum_{e \in out(s)} f(e)
\]

%\textbf{PREGUNTAR SI ESTO ESTA BIEN}.

El problema m\'as com\'un cuando hablamos de flujo es el de encontrar un flujo m\'aximo. Esto es, encontrar un F m\'aximo en una red con una \'unica fuente y un \'unico sumidero.


\raggedright
  \bigskip
  \begin{center}
\begin{tikzpicture}[shorten >=1pt, auto, node distance=3cm,
   node_style/.style={circle},
   edge_style/.style={draw=black}]

\node[vertex] (S) at (0,0)  {s};
\node[vertex] (n1) at (3,2)  {1};
\node[vertex] (n2) at (6,2)  {2};
\node[vertex] (T) at (9,0)  {t};
\node[vertex] (n3) at (3,-2)  {3};
\node[vertex] (n4) at (6,-2)  {4};

    
\draw[edge_style]  (S) edge node{3/\textbf{3}} (n1);

\draw[edge_style]  (S) edge node{2/\textbf{3}} (n3);

\draw[edge_style]  (n1) edge node{0/\textbf{2}} (n3);

\draw[edge_style]  (n1) edge node{3/\textbf{3}} (n2);
\draw[edge_style]  (n3) edge node{2/\textbf{2}} (n4);
\draw[edge_style]  (n2) edge node{1/\textbf{4}} (n4);

\draw[edge_style]  (n2) edge node{2/\textbf{2}} (T);
\draw[edge_style]  (n4) edge node{3/\textbf{3}} (T);



\end{tikzpicture}
\end{center}

Este problema est\'a en P. Notar que un flujo m\'aximo no requiere que todos los ejes est\'en transportando su m\'axima capacidad (ver $(s,3)$). No obstante, la suma de los flujos de $t$ es m\'axima.

\newpage
\subsection{Cortes}

Un \emph{``corte''} en una red $G = (V, E)$ es un subconjunto $S \subseteq V - {t}$ tal que $s \in S$. Es decir, un subconjunto de v\'ertices de la red en los que est\'a la fuente pero no el sumidero.

Dados dos cortes $S$ y $T$, $ST = \{ (v \rightarrow w) \in E \textrm{ tales que } v \in S \textrm{ y } w \in T \}$. Es decir, $ST$ es el conjunto de ejes para los cuales el v\'ertice de salida est\'a en $S$ y el de llegada est\'a en $T$.

Sea $f$ un flujo definido en una red $N = (V, E)$, sea $S$ un corte y sea $\bar{S} = V - S$; entonces:

\[
F = \sum_{e \in S\bar{S}} f(e) - \sum_{e \in \bar{S}S} f(e)
\]

%Es decir, el valor del flujo de la red va a ser igual al fujo saliente de un corte menos el flujo entrante al corte?

\subsection{Red Residual}

La \emph{``capacidad residual''} de un eje $e$ respecto de un flujo factible $f$ es la diferencia entre la capacidad de $e$ y su flujo. Es decir: $c_f(e) = c(e) - f(e)$.

Dada una red $G = (V, E)$ con una funci\'on de capacidad $c$ y un flujo factible $f$, definimos a la red que modela la capacidad disponible de $G$ como la \emph{``red residual''} $G_f = (V, E_{R})$, donde para todo $(v \rightarrow w) \in E$:

\begin{itemize}
\item $(v \rightarrow w) \in E_R \Longleftrightarrow f((v \rightarrow w)) < c((v \rightarrow w))$. Es decir, el eje $(v \rightarrow w)$ va a estar en la red residual si su flujo es menor a su capacidad.
\item $(w \rightarrow v) \in E_R \Longleftrightarrow f((v \rightarrow w)) > 0$.
\end{itemize}

De esta manera es trivial ver que en $G_f$ pueden haber ejes que no est\'an en $G$. Por ejemplo, miremos el siguiente grafo y su red residual:

\raggedright
  \bigskip
  \begin{center}
  \begin{tikzpicture}

  \node[vertex] (n1) at (-9,0)  {s};
  \node[vertex] (n2) at (-6,1)  {1};
  \node[vertex] (n3) at (-6,-1)  {2};
  \node[vertex] (n5) at (-3,0)  {t};
  

  \draw(n1) edge node{3/4} (n2);
  
  
  \end{tikzpicture}
  \end{center}


  \raggedright
  \bigskip


\subsection{Camino de Aumento}

Un \emph{``camino de aumento''} es un camino orientado $P$ de $s$ a $t$ en $G_f$. Dada una red $G$, esta se encuentra en su flujo m\'aximo si y s\'olo si no hay camino de aumento en $G_f$.

\subsection{Algoritmo de Ford-Fulkerson}

La idea del algoritmo es sencilla: mientras exista un camino en $G$ desde $s$ hasta $t$ que pueda transportar m\'as flujo, mandale m\'as flujo. Este camino existe si y s\'olo si hay un camino de aumento en $G_f$.

\begin{algorithm}
\begin{algorithmic}[1]
\Function{FordFulkerson}{$G = (V, E), c: E \rightarrow {\rm I\!R^{\geq 0}}$}
  \ForAll{$e \in E$}
    \State $f(e) \gets 0$
  \EndFor
  \State $G_f \gets calcularRedResidual(G, c, f)$
  \While{$p \gets obtenerCaminoDeAumento(G_f)$}
    \State $c_{min} \gets min\{c_f(e) \textrm{ tal que } e \in p\}$
    \ForAll{$(v,w) \in p$}
        \State $f(v,w) \gets f(v,w) + c_{min}$
        \State $f(w,v) \gets f(w,v) - c_{min}$ 
    \EndFor
  \EndWhile
\EndFunction
\end{algorithmic}
\end{algorithm}

\subsubsection*{Explicaci\'on}

El algoritmo utiliza fuertemente la propiedad de que, mientras exista un camino de aumento en $G_f$, $f$ no va a ser un flujo m\'aximo. 

\begin{enumerate}
\item [2: y 3:] Inicialmente el algoritmo genera un flujo factible $f$, donde todos los flujos para todos los ejes son $0$.
\item [4:] Luego calcula la red residual $G_f$, utilizando la red original $G$, el flujo factible $f$ y la funci\'on de capacidad $c$.
\item [5:] Ac\'a comienza la iteraci\'on principal del algoritmo. Mientras pueda encontrar un camino de aumento para asign\'arselo a $p$, el flujo $f$ no es m\'aximo y va a poder ser mejorado haciendo lo siguiente:
  \begin{enumerate}
  \item [6:] Dado el camino $p$ que encontr\'e, busco el eje de menor capacidad residual del mismo. Esta capacidad residual se llama $c_{min}$.
  \item [7:] Empiezo a recorrer todos los ejes del camino $p$, y a cada eje:
    \begin{enumerate}
      \item [8:] Le sumo a su flujo la capacidad residual m\'inima del camino. Esto va a aumentar el flujo total que se mueve por el camino $p$.
      \item [9:] Pero a su vez
    \end{enumerate}
  \end{enumerate}
\end{enumerate}

\subsubsection{Algoritmo de Edmonds-Karp}

Este algoritmo es en realidad una implementaci\'on de Ford Fulkerson que usa espec\'ificamente BFS para encontrar el camino de aumento $p$ en la funci\'on $obtenerCaminoDeAumento(G_f)$. Esto mejora la complejidad, dej\'andola en $\mathcal{O}(|V|^2 * |E|)$.

\newpage
\section{Teor\'ia de Complejidad}

Denominamos como \emph{``problema de decisi\'on''} a los problemas cuya respuesta es \emph{``s\'i''} o \emph{``no''}. El objetivo de esta teor\'ia es el de clasificar a este tipo de problemas seg\'un su complejidad.

Un algoritmo \emph{``eficiente''} es un algoritmo de complejidad polinomial, y decimos que un problema est\'a \emph{``bien resuelto''} si se conocen algoritmos eficientes para resolverlo. Estos problemas pertenecen a la clase $P$. Por ejemplo, el problema $\Pi$ encontrar un v\'ertice $v$ en un grafo de $n$ v\'ertices puede resolverse en tiempo $O(n)$ usando DFS, por lo que $\Pi \in P$.

\subsection{La clase NP}

Un problema de decisi\'on $\Pi$ pertenece a la clase $NP$ (no-det\'erministico polinomial) si dada cualquier instancia del mismo para las cuales la respuesta es \emph{s\'i} y un \emph{``certificado''}, podemos chequear que dicho certificado es correcto usando un algoritmo de tiempo polinomial.

Veamos por ejemplo el problema $\Pi$ que consiste en: \emph{``dado un array $A$ de $n$ enteros, existe una subsecuencia del mismo que sume 0?''}. Una instancia del problema puede ser el array $A = \{-7, -3, -2, 8, 5\}$ y un certificado puede ser $c = \{-3, -2, 5\}$. Como es trivial ver que la suma de todos los elementos de $c$ es $0$ en tiempo polinomial, entonces tenemos que $\Pi \in NP$.

Es trivial darse cuenta que $P$ est\'a contenido en $NP$, pero la gran inc\'ognita en la teor\'ia de la complejidad es determinar si $P = NP$. Es decir, si para cada problema en $NP$ existe una soluci\'on polinomial.

\subsection{La clase NP-Complete}

NP-Complete es una subclase de NP, pero con la particularidad de que todos los problemas en NP pueden ser reducible en tiempo polinomial a cualquier problema de NP-Complete.

\subsection{La clase NP-Hard}




\newpage
\section{Preguntas}
\begin{enumerate}
\item [Camino m\'inimo] C\'omo justificar la complejidad del algoritmo de Dijkstra?
\item [Camino m\'inimo] C\'omo se aplica el principio del optimalidad en el algoritmo de Floyd?
\item [Camino m\'inimo] Si usando un Fibonacci heap puedo hacer que el algoritmo de Dijkstra resuelva el problema de camino m\'inimo en $O(m + n log n)$, que es mejor que $O(n^2)$, no es m\'as r\'apido aplicar $n$ veces el algoritmo de Dijkstra en lugar de Floyd? Me terminar\'ia quedando una complejidad de $O(n(m + n log n)) = $$O(n*m + n^2 log n)$, que es mejor que $O(n^3)$.
\item [Camino m\'inimo] C\'omo funciona el algoritmo de Dantzig?
\item [Planaridad] Definimos a la \textbf{\emph{``parte $p$ de $G$ relativa a $R$''}} como un eje $e = (u, v)$ de $G - R$ con $u,v \in R$, pero entonces ni $u$ ni $v$ van a estar en $G - R$, por lo que $e = (u, v)$ no estar\'ia bien definido.
\item [Planaridad] Si $R$ es una regi\'on, qu\'e significa hacer $G - R$?
\end{enumerate}

\end{document}

