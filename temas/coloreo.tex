\newpage
\section{Coloreo}

Un \emph{``coloreo''} de los v\'ertices de un grafo $G = (V, E)$ es una asignaci\'on $f : V \rightarrow C$, tal que $f(v) \neq f(w)$ para todo $(v,w) \in E$. Para todo entero positivo $k$, un \emph{``$k$-coloreo''} de $G$ es un coloreo que usa exactamente $k$ colores.

Un grafo se dice \emph{``$k$-coloreable''} si existe un $k$-coloreo de $G$. De esta manera definimos al n\'umero crom\'atico de $G$, $\chi(G)$, como el menor n\'umero de colores necesarios para colorear los v\'ertices de $G$. Un grafo se dice \emph{``$k$-crom\'atico''} si $\chi(G) = k$.

Algunos ejemplos:

\begin{itemize}
\item $\chi(K_n) = n$
\item Si $G$ es bipartito $\Rightarrow \chi(G) = 2$
\item Si $C_{2k}$ es un circuito simple par $\Rightarrow \chi(C_{2k}) = 2$
\item Si $C_{2k+1}$ es un circuito simple impar $\Rightarrow \chi(C_{2k+1}) = 3$
\end{itemize}

\textbf{Propiedad de la cota superior de $\chi(G)$}: Si $\Delta(G)$ es el grado m\'aximo de $G$, entonces $\chi(G) \leq \Delta(G) + 1$.

\subsection{Teorema de Brooks}

Sea $G$ un grafo conexo que no es un circuito impar ni un grafo completo, entonces $\chi(G) \leq \Delta(G)$.

\subsection{Teorema de Heawood (1890)}

Si $G$ es un grafo planar, entonces $\chi(G) \leq 5$.

\subsection{Teorema de los Cuatro Colores (1976)}

Si $G$ es un grafo planar, entonces $\chi(G) \leq 4$.

\subsection{Algoritmos para coloreo de grafos}

No se conocen algoritmos polinomiales para calcular $\chi(G)$ dado un grafo general $G$. Dado un grafo $G$ de $n$ v\'ertices y un entero $k$, el problema de determinar si existe un $k$-coloreo para $G$ es NP-completo, y su complejidad es $O(2^n * n)$. En particular, encontrar el n\'umero crom\'atico es $NP$-hard.

\newpage
\subsubsection{Algoritmo secuencial (S)}

Dado un orden $v_1, v_2, ..., v_n$ de $V$, asignar en el paso $i$ el menor color posible a $v_i$. Es decir, el color m\'as chico que no est\'a siendo usado por los vecinos de $v_i$ (greedy). Puede devolver resultados bastante malos en algunos grafos.

\subsubsection{Algoritmo secuencial (LFS)}

Consiste en ordenar los v\'ertices por su grado de mayor a menor.

\subsubsection{Algoritmo secuencial (SLS)}

\subsubsection{Algoritmo secuencial con backtracking (exacto)}

La idea es ir asignado colores uno por uno a los diferentes v\'ertices, arrancando desde el v\'ertice 0. Antes de asignar un color, miramos los colores de los vecinos de dicho v\'ertice. Si encontramos un color apropiado, se lo ponemos al v\'ertice en cuesti\'on. Si no es posible encontrar un color apropiado, volvemos hacia atr\'as y probamos con otro color en alg\'un v\'ertice anterior.

\subsection{Cotas inferiores para $\chi(G)$}

Si $H$ es un subgrafo de $G$, entonces $\chi(H) \leq \chi(G)$.

Una \emph{``clique''} es un subgrafo completo maximal de un grafo. El \emph{``n\'umero clique''} de un grafo es el m\'aximo n\'umero de v\'ertices de una clique de $G$, y se denota por $\omega(G)$. Para cualquier grafo $G$, $\chi(G) \geq \omega(G)$.

\subsection{Grafos perfectos}

Un grafo $G$ es \emph{``perfecto''} si $\chi(H) = \omega(H)$ para todo subgrafo inducido $H$ de $G$.

\subsubsection{Teorema de los grafos perfectos}

Un grafo es perfecto si y s\'olo si su complemento es perfecto.

\subsubsection{Teorema de Grotschel, Lov\'asz y Schrijver}

Existe un algoritmo polinomial para determinar $\chi(G)$ si $G$ es perfecto.

\subsubsection{Teorema Fuerte de los Grafos Perfectos}

Un grafo es perfecto si y s\'olo si no tiene ciclos impares ni complementos de ciclos impares como subgrafos inducidos.


\newpage
\subsection{Coloreo de ejes}

Un \emph{``coloreo de ejes''} de un grafo $G$ es una asignaci\'on de colores a los mismos en el cual dos ejes que tienen un v\'ertice en com\'un no pueden tener el mismo color.

El \emph{``\'indice crom\'atico''} $\chi'(G)$ de un grafo $G$ es el menor n\'umero de colores con el que se pueden colorear los ejes de $G$.

\subsubsection{Teorema de Vizing}

Para todo grafo $G$ se cumple que $\Delta(G) \leq \chi'(G) \leq \Delta(G) + 1$