\newpage
\section{Camino M\'inimo}

Sea $G = (V, E)$ un grafo y $l: E \rightarrow {\rm I\!R}$ una funci\'on de peso para los ejes de $G$, definimos como el \emph{``peso''} de un camino $C$ entre dos nodos $v$ y $w$ como la suma de los pesos de los ejes del camino:

\begin{equation}
l(C) = \sum_{e \in C} l(e)
\end{equation}

Un \emph{``camino minimo''} $C^0$ entre $v$ y $w$ es un camino tal que $l(C^0) = min \{ l(C), \forall \textrm{$ C$ camino entre $v$ y $w$}  \}$. Esto significa que no necesariamente tiene que ser \'unico. Dado un grafo $G$, se pueden definir 3 variantes de problemas sobre caminos m\'inimos:

\begin{enumerate}
\item \textbf{Unico origen - \'unico destino}: determinar un camino m\'inimo entre dos v\'ertices $v$ y $w$.
\item \textbf{Unico origen - m\'ultiples destinos}: determinar un camino m\'inimo desde un v\'ertice $v$ al resto de los v\'ertices de $G$
\item \textbf{M\'ultiples or\'igenes - m\'ultiples destinos}: Determinar un camino m\'inimo entre todo par de v\'ertices de $G$.
\end{enumerate}

Si el grafo $G$ no contiene ciclos con peso negativo (o contiene alguno pero no es alcanzable desde $v$) entonces el problema sigue estando bien definido, aunque algunos caminos pueden tener longitud negativa. Sin embargo, si si $G$ tiene alg\'un ciclo con peso negativo alcanzable desde $v$, el concepto de camino m\'inimo deja de estar bien definido.

Un camino m\'inimo no puede contener circuitos. Tambi\'en es importante notar que un camino m\'inimo exhibe la propiedad de subestructura \'optima, ya que dados dos puntos $v'$ y $w'$ que est\'an adentro del camino m\'inimo entre $v$ y $w$, el subcamino entre estos dos puntos tambi\'en es un camino m\'inimo entre ambos.

\newpage
\subsection{Algoritmo de Dijkstra}

Dado $G = (V, E)$ y grafo, $l: E \rightarrow {\rm I\!R}$ una funci\'on que asigna a cada eje un cierto peso y $v$ un v\'ertice de $G$, calcular los caminos m\'inimos desde $v$ al resto de los v\'ertices. El algoritmo de Dijkstra asume que los pesos de los ejes son positivos.


\newpage
\subsection{Algoritmo de Ford}
\newpage
\subsection{Algoritmo de Floyd}
\newpage
\subsection{Algoritmo de Dantzig}