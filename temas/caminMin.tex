\newpage
\section{Camino M\'inimo}

Sea $G = (V, E)$ un grafo y $l: E \rightarrow {\rm I\!R}$ una funci\'on de peso para los ejes de $G$, definimos como el \emph{``peso''} de un camino $C$ entre dos nodos $v$ y $w$ como la suma de los pesos de los ejes del camino:

\begin{equation}
l(C) = \sum_{e \in C} l(e)
\end{equation}

Un \emph{``camino minimo''} $C^0$ entre $v$ y $w$ es un camino tal que $l(C^0) = min \{ l(C), \forall \textrm{$ C$ camino entre $v$ y $w$}  \}$. Esto significa que no necesariamente tiene que ser \'unico. Dado un grafo $G$, se pueden definir 3 variantes de problemas sobre caminos m\'inimos:

\begin{enumerate}
\item \textbf{Unico origen - \'unico destino}: determinar un camino m\'inimo entre dos v\'ertices $v$ y $w$.
\item \textbf{Unico origen - m\'ultiples destinos}: determinar un camino m\'inimo desde un v\'ertice $v$ al resto de los v\'ertices de $G$
\item \textbf{M\'ultiples or\'igenes - m\'ultiples destinos}: Determinar un camino m\'inimo entre todo par de v\'ertices de $G$.
\end{enumerate}

Si el grafo $G$ no contiene ciclos con peso negativo (o contiene alguno pero no es alcanzable desde $v$) entonces el problema sigue estando bien definido, aunque algunos caminos pueden tener longitud negativa. Sin embargo, si si $G$ tiene alg\'un ciclo con peso negativo alcanzable desde $v$, el concepto de camino m\'inimo deja de estar bien definido.

Un camino m\'inimo no puede contener circuitos. Tambi\'en es importante notar que un camino m\'inimo exhibe la propiedad de subestructura \'optima, ya que dados dos puntos $v'$ y $w'$ que est\'an adentro del camino m\'inimo entre $v$ y $w$, el subcamino entre estos dos puntos tambi\'en es un camino m\'inimo entre ambos.

\subsection{BFS}

Cuando el grafo no tiene pesos.

\newpage
\subsection{Algoritmo de Dijkstra (1 a n)}

Dado $G = (V, E)$ y grafo, $l: E \rightarrow {\rm I\!R}$ una funci\'on que asigna a cada eje un cierto peso y $v$ un v\'ertice de $G$, calcular los caminos m\'inimos desde $v$ al resto de los v\'ertices. El algoritmo de Dijkstra asume que los pesos de los ejes son positivos.

\begin{algorithm}
\begin{algorithmic}[1]
\Function{Dijkstra}{$G = (V, E)$}
  \State $prev[A] \gets -1$
  \State $dist[A] \gets 0$
  \State $pq.add(<dist[A], A>)$
  \ForAll{$v \in V - \{ A \}$}
    \State $prev[v] \gets -1$
    \State $dist[v] \gets \infty$
    \State $pq.add(<dist[v], v>)$
  \EndFor
  \While{$!pq.empty()$}
    \State $t \gets pq.pop()$
    \ForAll{$w \in t.second.neigh()$}
      \State $alt \gets dist[t.second] + length(t.second, w)$
      \If{$alt < dist[w]$}
        \State $prev[w] \gets t.second$
        \State $dist[w] \gets alt$
        \State $pq.decreaseKey(<dist[w], u>)$
      \EndIf
    \EndFor
  \EndWhile
  \State \textbf{return} $prev$
\EndFunction
\end{algorithmic}
\end{algorithm}

\subsubsection*{Analisis}

Este algoritmo toma un grafo $G$ y devuelve un array de v\'ertices llamado $prev$, que consiste en los caminos m\'inimos en el grafo desde el v\'ertice $A$.

\begin{enumerate}
\item [2:] Seteamos al v\'ertice previo de $A$ como $-1$, ya que es el v\'ertice por el que comienzan todos los caminos m\'inimos.
\item [3:] L\'ogicamente la distancia m\'inima de $A$ hacia $A$ es 0. Esto se asume porque $G$ no tiene ciclos negativos.
\item [4:] Declaramos como $pq$ a una cola de prioridad. Ac\'a vamos a guardar tuplas $<d, w>$, donde $d$ es la distancia m\'inima desde el v\'ertice $A$ hasta el v\'ertice $w$. Hacemos esto para poder obtener el v\'ertice de menor distancia a $A$ en O(log n). Ac\'a hay informaci\'on redundante (las distancias m\'inimas ya se est\'an guardando en $dist$) pero es beneficiosa para la complejidad del algoritmo.
\item [5 a 8:] Luego vamos a empezar a iterar todos los v\'ertices de $G$ seteando la informaci\'on necesaria para cada uno. Esto significa setear su v\'ertice previo en $-1$ y su distancia m\'inima hacia $A$ en infinito. Este proceso es O(n).
\item [9:] Ahora comienza el ciclo principal del algoritmo. En cada iteraci\'on del while vamos a visitar el v\'ertice de menor distancia hacia $A$ que encontremos en $pq$. El algoritmo termina cuando $pq$ est\'a vac\'ia, es decir, visitamos a todos los v\'ertices. Vale destacar que el primer v\'ertice que visitamos es $A$.
    \begin{enumerate}
    \item [10:] Guardamos en una tupla $t$ al v\'ertice que sacamos de $pq$. Hay que tener en cuenta que en $t.first$ vamos a tener la distancia m\'inima del v\'ertice que est\'a en $t.second$ hasta $A$.
    \item [11:] Vamos a iterar a todos los vecinos $w$ de $t.second$.
        \begin{enumerate}
        \item [12:] Definimos a la variable $alt$ como la longitud que tendr\'ia el camino hacia $w$ si pas\'aramos antes por $t.second$.
        \item [13:] \¿Es esta nueva distancia menor a la distancia que ten\'iamos calculada antes?. Es decir, \¿si pasamos por $t.second$ para llegar desde $A$ a $w$, llegamos m\'as r\'apido que si no pasamos?. Si la respuesta es \textbf{s\'i}, entonces mejoramos la distancia previamente calculada para $w$ y tenemos que actualizar todo:
            \begin{enumerate}
            \item [14:] Entonces para llegar a $w$ vamos a querer pasar por $t.second$.
            \item [15:] Actualizamos la distancia m\'inima de $A$ hasta $w$.
            \item [16:] Hacemos lo mismo, pero en $pq$.
            \end{enumerate}
        \end{enumerate}
    \end{enumerate}
\end{enumerate}

\subsubsection*{Complejidad}

La complejidad del algoritmo est\'a dada por el ciclo principal del mismo que empieza en la l\'inea 9. Vamos a tener que iterar tantas veces como elementos haya en $pq$, donde inicialmente van a haber $n$ elementos.

En cada iteraci\'on vamos a tener que obtener el v\'ertice m\'as cercano a $A$ (10:), lo cual puede hacerse en O(log n) usando una cola de prioridad basada en un min-heap.


\newpage
\subsection{Algoritmo de Bellman-Ford (1 a n)}

Es un poco m\'as lento que el algoritmo de Dijkstra, pero admite ejes de pesos negativos. La idea general es similar a la del algoritmo de Dijkstra, pero un poco m\'as simple. 

Mientras Dijkstra obtiene el v\'ertice no visitado de menor peso en cada iteraci\'on, Bellman-Ford simplemente relaja todos los ejes $n-1$ veces. En cada iteraci\'on el n\'umero de v\'ertices con caminos m\'inimos bien calculados se va incrementando, hasta que ya no se pueden mejorar.

\begin{algorithm}
\begin{algorithmic}[1]
\Function{BellmanFord}{$G = (V, E)$}
  \ForAll{$v \in V, v \neq A$}
    \State $dist[v] \gets \infty$
    \State $prev[v] \gets -1$
  \EndFor
  \State $dist[A] \gets 0$
  \For{$i \gets 1 \textrm{ \textbf{hasta} } n-1$}
    \ForAll{$(u, v) \textrm{ con peso } w \in E$}
        \If{$dist[u] + w < dist[v]$}
            \State $dist[v] \gets dist[u] + w$
            \State $prev[v] \gets u$
        \EndIf
    \EndFor
  \EndFor
  \ForAll{$(u, v) \textrm{ con peso } w \in E$}  \Comment{\¿Hay ciclos negativos? \textbf{O(E)}}
    \If{$dist[u] + w < dist[v]$}
        \State \textbf{return} null
    \EndIf
  \EndFor
  \State \textbf{return} $prev$
\EndFunction
\end{algorithmic}
\end{algorithm}

\subsubsection*{Analisis}

\begin{enumerate}
\item [\textbf{2 a 4:}] Seteamos, para todos los v\'ertices que no son $A$, la distancia m\'inima en $\infty$ y a su v\'ertice previo como $-1$.
\item [\textbf{5:}] Seteamos la distancia m\'inima hacia $A$ (el v\'ertice inicial) en $0$.
\item [\textbf{6:}] Comenzamos la iteraci\'on principal del algoritmo. Vamos a ciclar, como mucho $n-1$ veces.
    \begin{enumerate}
    \item [\textbf{7:}] Vamos a iterar por todos los ejes del grafo.
        \begin{enumerate}
        \item [\textbf{8:}] Para cada eje $(u, v)$, checkeamos si nos conviene pasar por $u$ para llegar hasta $v$.
            \begin{enumerate}
            \item [\textbf{9 y 10:}] Si este es el caso, actualizamos los valores.
            \end{enumerate}
        \end{enumerate}
    \end{enumerate}
\item [\textbf{11 a 13:}] Checkeamos si tenemos un ciclo negativo. Si este es el caso, el problema deja de estar bien definido y devolvemos $null$.
\end{enumerate}

\subsubsection*{Complejidad}

En este caso es bastante evidente que la complejidad es O(n * m), ya que vamos a ciclar a lo sumo $n-1$ veces y, en cada uno de estos ciclos, vamos a relajar, como mucho, los $m$ ejes de $G$.

\newpage
\subsection{Algoritmo de Floyd-Warshall (n a m)}

Dado un grafo dirigido $G = (V, E)$ sin pesos negativos, el algoritmo de Floyd sirve para calcular el camino m\'inimo entre cada par de v\'ertices de $G$ en tiempo $O(n^3)$.

%Supongamos primero que los v\'ertices de $G$ est\'an numerados de $1$ a $n$, y definamos a la matriz $L$ como el peso de cada eje de $E$; siendo $L[i][i] = 0$, para todo $i \in V$. $L[i][j] \geq 0$ para todo $i \neq j$ y $L[i][j] = \infty$ para todo par de v\'ertices que no est\'an conectados.

Supongamos que los v\'ertices de $V$ est\'an numerados de $1$ a $n$, y definamos a la funci\'on $caminoMin(i,j,k)$ que devuelve el camino m\'inimo desde el v\'ertice $i$ hasta el $j$, usando los v\'ertices que est\'an en $\{1, 2, ..., k\}$ como posibles puntos intermedios del camino.

Ahora bien, a partir de esa funci\'on, nuestro objetivo es el de encontrar el camino m\'inimo de $i$ hasta $j$ usando s\'olamente los v\'ertices que est\'an en $\{1, 2, ..., k, k+1\}$. Para cada par de v\'ertices $i,j$ el camino m\'inimo podr\'ia ser:

\begin{itemize}
\item Un camino que s\'olamente usa los v\'ertices del conjunto $\{1, 2, ..., k\}$
\item Un camino que va de $i$ a $k+1$, y de $k+1$ hasta $j$
\end{itemize}

Pero nosotros sabemos que el mejor camino de $i$ a $j$ que s\'olo usa los v\'ertices del conjunto $\{1, 2, ..., k\}$ est\'a dado por la funci\'on $caminoMin(i,j,k)$, y es claro que si hubiera un mejor camino de $i$ a $j$ que pasara por $k+1$, entonces la longitud de dicho camino deber\'ia ser la concatenaci\'on del camino m\'inimo de $i$ a $k+1$ (usando los v\'ertices en $\{1, 2, ...., k\}$) y el camino m\'inimo de $k+1$ a $j$ (tambi\'en usando los v\'ertices en $\{1, 2, ...., k\}$)

Si $w(i,j)$ es el peso del eje entre los v\'ertices $i$ y $j$, podemos definir a la funci\'on $caminoMin(i,j,k+1)$ con la siguiente recursividad:

\[
caminoMin(i,j,0) = w(i,j)
\]
\[
caminoMin(i,j,k+1) = min\left\{\begin{array}{lr}
    caminoMin(i,j,k), \\
    caminoMin(i,k+1,k) + caminoMin(k+1, j, k)
    \end{array}\right\}
\]


%Se aplica el principio de optimalidad: Si $k$ es un v\'ertice en el camino m\'inimo desde $i$ hasta $j$, entonces los subcaminos que van desde $i$ hasta $k$ y desde $k$ hasta $j$ tambi\'en tienen que ser m\'inimos.


%\[
%D[i][j] = min\{ D[i][j], D[i][k] + D[k][j]\}
%\]

\begin{algorithm}
\begin{algorithmic}[1]
\Function{FloydWarshall}{$G = (V, E)$}
  \State $dist \gets \infty$
  \State $next \gets -1$
  \ForAll{$v \in V$}
    \State $dist[v][v] \gets 0$
  \EndFor
  \ForAll{$e = (v, w)\in E$}
    \State $dist[v][w] \gets w(e)$
    \State $next[v][w] \gets v$
  \EndFor

  \ForAll{$k \textrm{ \textbf{from} } 1 \textrm{ \textbf{to} } n$}
    \ForAll{$i \textrm{ \textbf{from} } 1 \textrm{ \textbf{to} } n$}
      \ForAll{$j \textrm{ \textbf{from} } 1 \textrm{ \textbf{to} } n$}
        \State $alt \gets dist[i][k] + dist[k][j]$
        \If{$alt < dist[i][j]$}
          \State $dist[i][j] \gets alt$
          \State $next[i][j] \gets next[i][k]$
        \EndIf
      \EndFor
    \EndFor
  \EndFor
  \State \textbf{return} $next$
\EndFunction
\end{algorithmic}
\end{algorithm}

Nos construimos una matriz $D$ que contiene las longitudes de los caminos m\'inimos entre cada par de v\'ertices, y el algoritmo la inicializa con los valores de $L$.

El algoritmo empieza a iterar. Despu\'es de la iteraci\'on $k$, $D$ contiene las longitudes m\'inimas que s\'olo usan los v\'ertices en el conjunto $\{1, 2, ..., k\}$ como v\'ertices intermedios. Despu\'es de $n$ iteraciones $D$ contiene las longitudes m\'inimas que usan cualquiera de los $n$ v\'ertices, lo cual es el resultado que quer\'iamos.

En la iteraci\'on $k$, el algoritmo debe checkear para cada par de v\'ertices $i$ y $j$ si existe un camino desde $i$ hasta $j$ que pasa por el v\'ertice $k$ que tiene longitud menor que la que ya ten\'iamos calculada antes, la cual correspond\'ia al camino de $i$ a $j$ que s\'olo usaba los v\'ertices en $\{1, 2, ..., k-1\}$.



\newpage
\subsection{Algoritmo de Dantzig (n a m)}

Al finalizar la iteraci\'on $k-1$, el algoritmo de Dantzig gener\'o una matriz $k x k$ de caminos m\'inimos en el subgrafo inducido por los v\'ertices $\{v_1, ..., v_k\}$. Asumimos que el grafo es orientado. El algoritmo detecta si hay circuitos de longitud negativa.

\begin{algorithm}
\begin{algorithmic}[1]
\Function{Dantzig}{$G = (V, E)$}
  \ForAll{$k \textrm{ \textbf{from} } 1 \textrm{ \textbf{to} } n-1$}
    \ForAll{$k \textrm{ \textbf{from} } 1 \textrm{ \textbf{to} } n-1$}
      \State $L_{i,k+1} \gets min_{1 \leq j \leq k}\{ L_{i,j} + L_{j,k+1} \}$
      \State $L_{k+1,i} \gets min_{1 \leq j \leq k}\{ L_{k+1,j} + L_{j,i} \}$
    \EndFor
    \State $t \gets min_{1 \leq j \leq k}\{ L_{k+1,i} + L_{i,k+1} \}$
    \If{$t < 0$}
      \State \textbf{return} null
    \EndIf
    \ForAll{$i \textrm{ \textbf{from} } 1 \textrm{ \textbf{to} } k$}
      \ForAll{$j \textrm{ \textbf{from} } 1 \textrm{ \textbf{to} } k$}
        \State $L_{i,j} \gets min\{L_{i,j}, L_{i,k+1} + L_{k+1, j}\}$
      \EndFor
    \EndFor
  \EndFor
  \State \textbf{return} $L$
\EndFunction
\end{algorithmic}
\end{algorithm}