\newpage
\section{Planaridad}

Una \emph{``representaci\'on planar''} de un grafo $G$ es un conjunto de puntos en el plano que se corresponden con los v\'ertices de $G$, unidos por curvas que se corresponden con los ejes de $G$, sin que estas se crucen entre s\'i. Si un grafo admite una representaci\'on planar, decimos que el mismo es planar.

Dada una representaci\'on planar de un grafo, llamamos \emph{``regi\'on''} al conjunto de todos los puntos alcanzables desde un punto que no sea un v\'ertice o un eje, sin atravesar ning\'un v\'ertice o eje. Es por eso que toda representaci\'on planar de un grafo tiene exactamente una regi\'on de \'area infinita: la regi\'on exterior.

La \emph{``frontera''} de una regi\'on es el circuito que rodea a la regi\'on, y el \emph{``grado''} de la regi\'on es el n\'umero de ejes que tiene su frontera.

\begin{badidea}
\textbf{Propiedad:} $K_5$ es el grafo no-planar con el menor n\'umero de v\'ertices, y $K_{33}$ el que tiene el menor n\'umero de ejes.
\end{badidea}

\begin{badidea}
\textbf{Propiedad:} Si un grafo $G$ contiene sun subgrafo $G'$ no planar, entonces $G$ tampoco es planar.
\end{badidea}

\subsection{Subdivisiones y Homeomorfismo}

Subdividir un eje $e = (v,w)$ de un grafo $G$ consiste en agregar un v\'ertice nuevo $u \not\in V$ a $G$ y reemplazar al eje $e$ por dos ejes $e' = (v, u)$ y $e'' = (u, w)$. Un grafo $G'$ es una subdivisi\'on de otro grafo $G$ si $G'$ se puede obtener mediante sucesivas operaciones de subdivisi\'on sobre $G$. Dos grafos $G$ y $H$ se dicen \emph{``homeomorfos''} si hay un isomorfismo entre una subdivisi\'on de $G$ y una de $H$.

\begin{badidea}
\textbf{Propiedad 1:} Si $G'$ es una subdivisi\'on de $G$, entonces $G$ es planar si y s\'olo si $G'$ es planar.
\end{badidea}

\begin{badidea}
\textbf{Propiedad 2:} La planaridad es invariate bajo homeomorfismo.
\end{badidea}

\begin{badidea}
\textbf{Propiedad 3:} Si un grafo $G$ tiene un subgrafo que es homeomorfo a un grafo no-planar, entonces $G$ es no-planar.
\end{badidea}

\subsubsection{Teorema de Kuratowski}

Un grafo es planar si y s\'olo si no contiene ning\'un subgrafo homemomorfo a $K_5$ o a $K_{33}$.

\subsection{Contracciones}

La operaci\'on de \emph{``contracci\'on''} de un eje $e = (v, w)$ consiste en eliminar el eje del grafo y considerar sus extremos como un s\'olo v\'ertice $u \not\in V$, quedando como ejes incidentes a $u$ todos los ejes que eran incidentes a $v$ y a $w$.

Un grafo $G'$ es una contracci\'on de otro grafo $G$ si se puede obtener a partir de $G$ por sucesivas operaciones de contracci\'on. Se dice que \emph{``$G$ es contraible a $G'$''}.

\newpage
\subsubsection{Teorema de Whitney}

$G$ es planar si y s\'olo si no contiene ning\'un subgrafo contraible a $K_{5}$ o a $K_{33}$.

\subsubsection{Teorema de Euler}

Si $G$ es un grafo conexo planar entonces cualquier representaci\'on planar de $G$ determina $r = m - n + 2$ regiones en el plano.

\begin{badidea}
\textbf{Colorario 1:} Si $G$ es conexo y planar con $n \geq 3$, entonces $m \leq 3n - 6$.
\end{badidea}

\begin{badidea}
\textbf{Colorario 2:} Si $G$ es conexo, planar y bipartito con $n \geq 3$, entonces $m \leq 2n - 4$.
\end{badidea}

\subsection{Algoritmo de Demoucron}

El siguiente algoritmo encuentra una represetaci\'on planar si existe, y si $G$ es no planar lo reconoce correctamente. El tiempo de ejecuci\'on es de $O(n^2)$, pero existen algoritmos que pueden determinar planaridad en tiempo lineal. Esto significa que el problema de decidir si un grafo es o no planar est\'a en $P$. Pero antes de exponer el algoritmo hay que hacer las siguientes definiciones:

\begin{itemize}
\item Una \textbf{\emph{``parte $p$ de $G$ relativa a $R$''}} puede ser alguna de las siguientes cosas:
    \begin{itemize}
    \item una componente conexa de $G - R$ junto con los ejes que la conectan a todos los nodos de $R$ (ejes colgantes).
    \item un eje $e = (v, w)$ de $G - R$ con $v,w \in R$.
    \end{itemize}
\item Dada una parte $p$ de $G$ relativa a $R$, un \textbf{\emph{``nodo de contacto''}} es un nodo $R$ incidente a un eje colgante de $p$.
\item Decimos que $R$ \textbf{\emph{``es extensible a una representaci\'on planar de $G$''}} si se puede obtener una representaci\'on planar de $G$ a partir de $R$.
\item Una parte $p$ es \textbf{\empg{``dibujable en una regi\'on $f$ de $R$''}} si existe una extensi\'on planar de $R$ en la que $p$ queda en $f$.
\item Una parte $p$ es \textbf{\empg{``potencialmente dibujable en una regi\'on $f$ de $R$''}} si todo nodo de contacto de $p$ pertenece a la frontera de $f$.
\item Llamamos \textbf{$F(p)$} al conjunto de regiones de $R$ donde $p$ es potencialmente dibujable.
\end{itemize}

\begin{algorithm}
\begin{algorithmic}[1]
\Function{Demoucron}{$G = (V, E)$}
  \State \textbf{return} $orden$
\EndFunction
\end{algorithmic}
\end{algorithm}