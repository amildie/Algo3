\newpage
\section{Planaridad}

Una \emph{``representaci\'on planar''} de un grafo $G$ es un conjunto de puntos en el plano que se corresponden con los v\'ertices de $G$, unidos por curvas que se corresponden con los ejes de $G$, sin que estas se crucen entre s\'i. Si un grafo admite una representaci\'on planar, decimos que el mismo es planar.

Dada una representaci\'on planar de un grafo, llamamos \emph{``regi\'on''} al conjunto de todos los puntos alcanzables desde un punto que no sea un v\'ertice o un eje, sin atravesar ning\'un v\'ertice o eje. Es por eso que toda representaci\'on planar de un grafo tiene exactamente una regi\'on de \'area infinita: la regi\'on exterior.

La \emph{``frontera''} de una regi\'on es el circuito que rodea a la regi\'on, y el \emph{``grado''} de la regi\'on es el n\'umero de ejes que tiene su frontera.

\begin{badidea}
\textbf{Propiedad:} $K_5$ es el grafo no-planar con el menor n\'umero de v\'ertices, y $K_{33}$ el que tiene el menor n\'umero de ejes.
\end{badidea}

\begin{badidea}
\textbf{Propiedad:} Si un grafo $G$ contiene sun subgrafo $G'$ no planar, entonces $G$ tampoco es planar.
\end{badidea}

\subsubsection{Teorema de Euler}

Si $G$ es un grafo conexo planar entonces cualquier representaci\'on planar de $G$ determina $r = m - n + 2$ regiones en el plano.

\begin{badidea}
\textbf{Colorario 1:} Si $G$ es conexo y planar con $|V| \geq 3$ $\Rightarrow |E| \leq 3 * |V| - 6$.
\end{badidea}

\begin{badidea}
\textbf{Colorario 2:} Si $G$ es conexo, planar y bipartito con $|V| \geq 3$ \\ $\Rightarrow$ $|E| \leq 2 * |V| - 4$.
\end{badidea}

\subsection{Subdivisiones y Homeomorfismo}

Subdividir un eje $e = (v,w)$ de un grafo $G$ consiste en agregar un v\'ertice nuevo $u \not\in V$ a $G$ y reemplazar al eje $e$ por dos ejes $e' = (v, u)$ y $e'' = (u, w)$. Un grafo $G'$ es una subdivisi\'on de otro grafo $G$ si $G'$ se puede obtener mediante sucesivas operaciones de subdivisi\'on sobre $G$. Dos grafos $G$ y $H$ se dicen \emph{``homeomorfos''} si hay un isomorfismo entre una subdivisi\'on de $G$ y una de $H$.

\begin{badidea}
\textbf{Propiedad 1:} Si $G'$ es una subdivisi\'on de $G$, entonces $G$ es planar si y s\'olo si $G'$ es planar.
\end{badidea}

\begin{badidea}
\textbf{Propiedad 2:} La planaridad es invariate bajo homeomorfismo.
\end{badidea}

\begin{badidea}
\textbf{Propiedad 3:} Si un grafo $G$ tiene un subgrafo que es homeomorfo a un grafo no-planar, entonces $G$ es no-planar.
\end{badidea}

\subsubsection{Teorema de Kuratowski}

Un grafo es planar si y s\'olo si no contiene ning\'un subgrafo homemomorfo a $K_5$ o a $K_{33}$.

\subsection{Contracciones}

La operaci\'on de \emph{``contracci\'on''} de un eje $e = (v, w)$ consiste en eliminar el eje del grafo y considerar sus extremos como un s\'olo v\'ertice $u \not\in V$, quedando como ejes incidentes a $u$ todos los ejes que eran incidentes a $v$ y a $w$.

Un grafo $G'$ es una contracci\'on de otro grafo $G$ si se puede obtener a partir de $G$ por sucesivas operaciones de contracci\'on. Se dice que \emph{``$G$ es contraible a $G'$''}.

\subsubsection{Teorema de Whitney}

$G$ es planar si y s\'olo si no contiene ning\'un subgrafo contraible a $K_{5}$ o a $K_{33}$.

\subsection{Algoritmo de Demoucron}

El siguiente algoritmo encuentra una represetaci\'on planar si existe, y si $G$ es no planar lo reconoce correctamente. El tiempo de ejecuci\'on es de $O(n^2)$, pero existen algoritmos que pueden determinar planaridad en tiempo lineal. Esto significa que el problema de decidir si un grafo es o no planar est\'a en P.

La idea del algoritmo es partir de una representaci\'on planar $R$ de un subgrafo $S$ de $G$, y empezar a expandirala iterativamente hasta obtener una representaci\'on planar de todo $G$ o hasta concluir que dicha representaci\'on no existe.

\subsubsection{Definiciones\footnote{A partir de ahora van a empezar a aparecer unas operaciones que pierden completamente el hilo de coherencia de las diapositivas. Especialmente la operaci\'on $G - R$, que vendr\'ia a ser algo como \emph{``restarle una representaci\'on planar a un grafo''}, lo cual adolece completamente de cualquier significado formal. Quien haya llegado hasta ac\'a va a tener que valerse m\'as de la intuici\'on para poder comprender las intenciones de quien hizo su mejor esfuerzo por armar las diapositivas. Tener especial cuidado con el concepto de \emph{``v\'ertice de contacto''}, el cual pertenece a una representaci\'on planar $R$ a veces y a una parte $p$ otras veces.}}


\begin{itemize}
\item Llamamos \textbf{\emph{``parte $p$ de $G$ relativa a $R$''}} a:
    \begin{itemize}
    \item Una componente conexa de $G - R$ junto con los ejes que la conectan a v\'ertices de $R$ (ejes colgantes). Notar que $R$ es una representaci\'on de un subgrafo $S$ de $G$.
    \item Un eje $e = (v,w)$ de $G - R$, con $v,w \in R$. Es decir, un eje suelto sin v\'ertices. 
    \end{itemize}

    Es decir, una parte es \emph{lo que queda} de un grafo al que le sacamos la representaci\'on planar de un subgrafo. Esta parte puede tener ejes colgantes que no est\'an conectados con ning\'un v\'ertice, e incluso una parte puede ser un eje suelto volando por ah\'i.

\item Dada una parte $p$ de $G$ relativa a $R$, llamamos \textbf{\emph{``v\'ertice de contacto''}} a un v\'ertice de $R$ incidente a un eje colgante de $p$.
\item Decimos que la representaci\'on $R$ es \textbf{\emph{``extensible''}} a una representaci\'on planar de $G$ si se puede obtener una representaci\'on planar de $G$ a partir de $R$. \emph{Extender} ac\'a es el proceso de ir agregandole cosas a una representaci\'on planar.
\item Una parte $p$ es \textbf{\emph{``dibujable''}} en una regi\'on $f$ de $R$ si existe una extensi\'on planar de $R$ en la que $p$ queda en $f$.
\item Una parte $p$ es \textbf{\emph{``potencialmente dibujable''}} en una regi\'on $f$ de $R$ si todo v\'ertice de contacto de $p$ pertenece a la frontera de $f$.
\item Llamamos $F(p)$ al conjunto de regiones de $R$ donde $p$ es potencialmente dibujable.
\end{itemize}

\subsubsection{Pseudoc\'odigo}

\begin{algorithm}
\begin{algorithmic}[1]
\Function{Demoucron}{$G = (V, E)$}
    \State $R \gets obtenerRepresentacionPlanarCiclo(G)$
    \While{$R$ no sea una representaci\'on planar de $G$}
        \State calculamos $F(p)$ para cada parte $p$ de $G$ relativa a $R$
        \If{existe un $p$ tal que $F(p) = \emptyset$}
            \State \textbf{return} false
        \ElsIf{existe un $p$ \textbf{tal que} $F(p) = \{ f \}$}
            \State elegir ese $p$ y ese $f$
        \Else
            \State elegir cualquier $p$ y un $f \in F(p)$ 
        \EndIf
        \State Buscar camino o circuito $q$ en $p$ que empieza y termina en dos
        \State ejes colgantes diferentes si es posible. Caso contrario, $q$ es
        \State el camino formado por el \'unico eje colgante.
    
        \State $R \gets R \cup q$
    \EndWhile
    \State \textbf{return} $R$
\EndFunction
\end{algorithmic}
\end{algorithm}

\subsubsection*{Explicaci\'on}

\begin{enumerate}
\item [\textbf{2:}] El algoritmo empieza definiendo una representaci\'on base $R$, que inicialmente puede ser de cualquier ciclo de $G$. 
\item [\textbf{3:}] Luego va a ciclar hasta que $R$ sea una representaci\'on de todo $G$ o hasta determinar que $G$ no es planar.
    \begin{enumerate}
    \item [\textbf{4:}] Empezamos calculanedo $F(p)$ para todas las partes $p$ de $G$ relativas a $R$.
    \item [\textbf{5 y 6:}] Si existe un $p$ tal que $F(p)$ es vac\'io, significa que no existen regiones de $R$ donde $p$ sea potencialmente dibujable, por lo que $G$ no es planar.
    \item [\textbf{7 y 8:}] Si existe un $p$ tal que $F(p)$ es igual a una \'unica region $f$, entonces no me queda m\'as opci\'on que seleccionar esa regi\'on $f$ para dibujar a $p$.
    \item [\textbf{9 y 10:}] Si no se cumpli\'o ninguna de las dos condiciones, elijo cualquier $p$ y cualquier region $f$ en la que $p$ sea potencialmente dibujable.
    \end{enumerate}
\end{enumerate}
