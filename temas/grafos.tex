\newpage
\section{Grafos}

Un \emph{``grafo''} $G = (V, E)$ es un par de conjuntos. $V$ es un conjunto de \emph{``v\'ertices''} y $X$ es un conjunto de \emph{``ejes''}, que a su vez son pares no ordenados de los elementos de $V$.

Dados dos v\'ertices $v, w \in V$ se dice que $v$ y $w$ son \emph{``adyacentes''} si $\exists$  $e \in E$ tal que $e = (v, w)$. Un \emph{``multigrafo''} es un grafo en el que pueden haber varios ejes entre el mismo par de v\'ertices. Un \emph{``seudografo''} es un multigrafo donde pueden haber ejes que unan a un v\'ertice con si mismo.

El \emph{``grado''} de un v\'ertice $v$ es la cantidad de ejes incidentes a \'el. Se nota con $d(v)$ y da lugar a la siguiente propiedad: 

\begin{figure}[h]
\[ \sum_{i=1}^{|V|} d(v_i) = 2 * |E| \]
\caption{La suma de los grados de todos los v\'ertices de un grafo es igual a dos veces el n\'umero de aristas.}
\end{figure}

Un grafo se dice \emph{``completo''} si todos los v\'ertices son adyacentes entre si. Al grafo completo de $n$ v\'ertices se lo nota $K_n$.

Dado un grafo $G$, se denomina como el grafo \emph{``complemento''} de $G$ al grafo $\overline{G}$ que tiene los mismos v\'ertices que $G$, pero donde dichos v\'ertices s\'olo son adyacentes en $\overline{G}$ si no lo son en $G$.

\subsection{Caminos y distancia}

Un \emph{``camino''} en un grafo es una sucesi\'on de ejes $e_1, e_2, ..., e_k$ tal que los extremos de $e_i$ coinciden con uno de $e_{i-1}$ y con uno de $e_{i+1}$, para todo $i \in \[2, ..., k-1 \]$.

Cuando un camino no pasa dos veces por el mismo v\'ertice, se lo denomina \emph{``camino simple''}. Un \emph{``circuito''} es un camino que empieza y termina en el mismo v\'ertice. Cuando un circuito tiene 3 o m\'as v\'ertices se lo denomina \emph{``circuito simple''}.

La \emph{``longitud''} de un camino es la cantidad de v\'ertices por los que pasa. La \emph{``distancia''} entre dos v\'ertices $v$ y $w$ de un grafo se define como la longitud del camino m\'as corto entre ambos, y se nota con  $d(v,w)$. Para todo v\'ertice $v$, $d(v,v) = 0$. Si no existe camino entre $v$ y $w$, se dice que la distancia entre ambos es infinita.

Si un camino $P$ entre $v$ y $w$ tiene longitud $d(v, w)$ entonces $P$ es un camino simple. Es decir, la distancia m\'as corta entre dos v\'ertices no va a dar vueltas en ning\'un lado.

Notar que si $P$ es un camino entre $u$ y $v$ de longitud $d(u,v)$ y tenemos los puntos $z$ y $w$ que est\'an adentro de $P$, entonces $P_{zw}$ es un camino entre $z$ y $w$ de longitud $d(z,w)$, donde $P_{zw}$ es el subcamino interno de $P$ entre $z$ y $w$.

\subsection{Subgrafos y biparticidad}

Dado un grafo $G = (V,E)$, un \emph{``subgrafo''} del mismo es un grafo $H = (V',E')$ que cumple que $V' \subseteq V$ y que $E' \subseteq E \cap (V' \times V')$. Si se cumple que para todo $u,v \in V', (u,v) \in E \Longleftrightarrow (u,v) \in E'$ entonces $H$ es un subgrafo \emph{``inducido''}. Es decir, para cada par de v\'ertices de $H$, se preservan los mismos ejes que ten\'ian en $G$.


Un grafo se dice \emph{``conexo''} si existe un camino que conecta cada par de v\'ertices. Una \emph{``componente conexa''} de un grafo $G$ es un subgrafo conexo maximal de $G$.

Un grafo $G = (V, E)$ se dice \emph{``bipartito''} si existe una partici\'on $V_1,V_2$ de $V$ tal que todos los ejes de $G$ tienen un extremo en $V_1$ y otro en $V_2$. Si todo v\'ertice de $V_1$ es adyacente a todo v\'ertice de $V_2$ entonces el $G$ es \emph{``bipartito completo''}. Un grafo es bipartito si y s\'olo si no tiene circuitos simples de longitud impar.

\subsection{Isomorfismo}

Dos grafos $G = (V,E)$ y $G' = (V',E')$ se dicen \emph{``isomorfos''} si existe una funci\'on biyectiva $f:V\rightarrow V'$ tal que para todo $v,w \in V$:

\begin{figure}[h]
\[ (v,w) \in E \Longleftrightarrow (f(v), f(w)) \in E' \]
\end{figure}

Si dos grafos son isomorfos, entonces:

\begin{itemize}
\item tienen el mismo n\'umero de v\'ertices
\item tienen el mismo n\'umero de ejes
\item tienen el mismo n\'umero de componentes conexas
\item $\forall$ $k, 0 \leq k \leq n-1$, tienen el mismo n\'umero de v\'ertices de grado $k$
\item $\forall$ $k, 1 \leq k \leq n-1$, tienen el mismo n\'umero de caminos simples de longitud $k$
\end{itemize}

No se conoce un algoritmo de tiempo polinomial para detectar si dos grafos son isomorfos, ni tampoco es NP-completo, pero en la pr\'actica hay maneras de resolverlo eficientemente.

\subsection{Pesos}

  Sea $G = (V, E)$ un grafo, decimos que los ejes de $G$ tienen \emph{``peso''}, si existe una funci\'on de peso $l: E \rightarrow {\rm I\!R}$ para los ejes de $G$.

\newpage
\subsection{Digrafos}

Un \emph{``grafo orientado''} (o \emph{``digrafo''}) es un grafo $G=(V,E)$ en el cual los ejes de $E$ tienen una direcci\'on. Para cada v\'ertice $v$ se define a su \emph{``grado de entrada''} ($d_{in}(v)$) como a la cantidad de ejes en $E$ que llegan a $v$. Es decir, que lo tienen como segundo elemento. El \emph{``grado de salida''} es lo mismo, pero con las aristas que salen del mismo.

Un \emph{``camino orientado''} es un digrafo es una sucesi\'on de ejes $e_1, e_2, ..., e_k$ ta que el primer elemento del par $e_i$ coincide con el segundo de $e_{i-1}$ y el segundo elemento de $e_i$ coincide con el primero de $e_{i+1}$, para todo $i = 2, ..., k-1$.

Un \emph{``circuito orientado''} en un digrafo es un camino orientado que comienza y termina en el mismo v\'ertice. Un digrafo se dice \emph{``fuertemente conexo''} si para todo par de v\'ertices $v,w$ existe un camino orientado de $v$ a $w$ y otro de $w$ a $v$.

\subsection{Grafo de L\'ineas}

Dado un grafo $G$, su \emph{``grafo de l\'ineas''} $L(G)$ es un grafo que cumple que:

\begin{enumerate}
\item Cada v\'ertice de $L(G)$ representa un eje de $G$
\item Dos v\'ertices de $L(G)$ son adyacentes $\Longleftrightarrow$ sus ejes correspondientes en $G$ son incidentes
\end{enumerate}

\raggedright
  \bigskip
  \begin{center}
  \begin{tikzpicture}[shorten >=1pt, auto, node distance=3cm]
  \node[vertex] (n1) at (-7,1)  {1};
  \node[vertex] (n2) at (-3,1)  {2};
  \node[vertex] (n3) at (-7,-1)  {3};
  \node[vertex] (n4) at (-3,-1)  {4};
  \node[vertex] (n5) at (-5,0)  {5};
  %\node[vertex] (n2) at (4,3)  {1};
  

  \draw(n2) edge node[right]{E} (n4);
  \draw(n1) edge node[left]{A} (n3);
  \draw(n3) edge node{B} (n5);
  \draw(n3) edge node{C} (n4);
  \draw(n5) edge node{D} (n4);
  
  %\foreach \from/\to in {n1/n2,n1/n3,n1/n5,n2/n1,n2/n3,n2/n4,n2/n5,n2/n6,n3/n1,n3/n2,n3/n5,n3/n6,n4/n2,n4/n5,n4/n6,n5/n1,n5/n2,n5/n3,n5/n4,n5/n6,n6/n2,n6/n3,n6/n4,n6/n5}
  %\draw (\from) -- (\to);

  \node[vertex] (n6) at (-1,1)  {A};
  \node[vertex] (n7) at (0,0)  {B};
  \node[vertex] (n8) at (1,-1)  {C};
  \node[vertex] (n9) at (2,0)  {D};
  \node[vertex] (n10) at (3,1)  {E};

  \node[draw=none,fill=none] at (-5,-2) {\textbf{$G$}};
  \node[draw=none,fill=none] at (1,-2) {\textbf{$L(G)$}};


  \draw(n6) -- (n7);
  \draw(n6) edge [bend right=60] (n8);
  \draw(n7) -- (n9);
  \draw(n7) -- (n8);
  \draw(n8) -- (n9);
  \draw(n9) -- (n10);
  \draw(n8) edge [bend right=60] (n10);

  \end{tikzpicture}
  \end{center}


  \raggedright
  \bigskip

