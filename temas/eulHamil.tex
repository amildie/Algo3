\newpage
\section{Grafos Eulerianos y Hamiltonianos}

\subsection{Grafos Eulerianos}

Un circuito $C$ en un grafo $G$ es un \emph{``circuito euleriano''} si $C$ pasa por todos los ejes de $G$ una y s\'olo una vez. Un grafo es euleriano si tiene un circuito euleriano. Un circuito euleriano puede pasar varias veces por el mismo v\'ertice. Un grafo conexo es euleriano $\Leftrightarrow$ todos sus nodos son de grado par. Podemos encontrar un circuito euleriano utilizando el algoritmo de Hierholzer en $O(m)$: 

\begin{algorithm}
\begin{algorithmic}[1]
\Function{Hierholzer}{$G = (V, E)$}
  \State $v \gets A$
  \State $Z \gets construirCiclo(v)$
  \While{$\exists\ e \in E, e \not\in Z$}
    \State $w \gets encontrarEje()$
    \State $D \gets construirCicloQueNoPasePor(Z, w)$
    \State $Z \gets Z \cup D$
  \EndWhile
  \State \textbf{return} $Z$
\EndFunction
\end{algorithmic}
\end{algorithm}

\subsubsection*{An\'alisis}

\begin{enumerate}
  \item [\textbf{2:}] Elegimos un nodo random de $G$. Puede ser $A$, o puede ser cualquier otro.
  \item [\textbf{3:}] La funci\'on $construirCiclo(v)$ nos devuelve un ciclo euleriano que pasa por $v$.
  \item [\textbf{4:}] Quiz\'as el ciclo construido en \textbf{3:} es un ciclo euleriano (por ejemplo, si el grafo es un pol\'igono). Si no lo es, entonces vamos a tener que iterar hasta que dejen de haber ejes que no est\'an en $Z$.
  \begin{enumerate}
    \item [\textbf{5:}] La funci\'on $encontrarEje()$ devuelve un eje $w$ tal que uno de los v\'ertices de $e$ est\'a en $Z$ y el otro no.
    \item [\textbf{6:}] La funci\'on $construirCicloQueNoPasePor(Z, w)$ devuelve un ciclo $D$, cuyos ejes no est\'an en $Z$.
    \item [\textbf{7:}] Ahora tenemos dos ciclos eulerianos que comparten como \'unico v\'ertice a $w$. Al unirlos, es trivial ver que todav\'ia voy a seguir teniendo un ciclo euleriano.
  \end{enumerate}
\end{enumerate}

\subsubsection{Problema del Cartero Chino}

El \emph{``Problema del Cartero Chino''} es el problema de encontrar, dado un grafo $G$ con pesos en sus ejes, un circuito euleriano de longitud m\'inima que pase por cada eje al menos una vez. Si $G$ es euleriano, la soluci\'on del problema es su circuito euleriano. Hay algoritmos polinomiales para resolverlo.


\newpage
\subsection{Grafos Hamiltonianos}

Un circuito en un grafo $G$ es un \emph{``circuito hamiltoniano''} si pasa por cada v\'ertice de $G$ s\'olo una vez. Un grafo es hamiltoniano cuando tiene dicho circuito. No se conocen caracterizaciones para grafos hamiltonianos ni tampoco un algoritmo polinomial para decidir si un grafo es hamiltoniano o no.

\newenvironment{badidea}
  {\par\leftskip=1cm}
  {\par}

\begin{badidea}
\textbf{Teorema (condici\'on necesaria):} Sea $G$ un grafo conexo, si existe $W \subset V$ tal que $G - W$ tiene $c$ componentes conexas con $c > |W|$, entonces $G$ no es hamiltoniano.
\end{badidea}

\begin{badidea}
\textbf{Teorema de Dirac (condici\'on suficiente):} Sea $G$ un grafo con $n \geq 3$ tal que para todo $v \in V$ se verifica que $d(v) \geq n/2$, entonces $G$ es hamiltoniano.
\end{badidea}

\subsubsection{The Travelling Salesman Problem}

Dado un grafo completo $G = (V, E)$ con longitudes asignadas a sus ejes, queremos determinar un circuito hamiltoniano de longitud m\'inima. No se conocen algoritmos polinomiales para resolver este problema, pero existen varias heur\'isticas para atacarlo:

\subsubsection*{Heur\'istica del vecino m\'as cercano}

Esta heur\'istica greedy es bastante r\'apida, pero no siempre obtiene el circuito hamiltoniano de menor peso. B\'asicamente lo que hace es partir de un v\'ertice cualquiera e ir movi\'endose por el camino de menor peso que encuentre, y as\'i sucesivamente hasta haber recorrido todos los v\'ertices.

\begin{algorithm}
\begin{algorithmic}[1]
\Function{NNA}{$G = (V, E)$}
  \State $v \gets getRandomNode()$
  \State $orden[v] \gets 0$
  \State $S \gets \{v\}$
  \State $i \gets 0$
  \While{$S \neq V$}
    \State $i++$
    \State $w \gets aristaMasBarata()$
    \State $orden[w] \gets i$
    \State $S \gets S \cup \{ w \}$
    \State $v \gets w$
  \EndWhile
  \State \textbf{return} $orden$
\EndFunction
\end{algorithmic}
\end{algorithm}

La funci\'on $aristaMasBarata()$ devuelve siempre una arista $(v,w)$ de peso m\'inimo, con $w \not\in S$. Este algoritmo puede no siempre devolver un ciclo, y si lo devuelve, puede no ser el de peso m\'inimo. Su complejidad es $O(n^2)$. Tambi\'en existe la posibilidad de correr este algoritmo una vez en cada v\'ertice, obteniendo un circuito hamiltoniano en cada ejecuci\'on, y qued\'andome con el de peso m\'inimo.

\newpage
\subsubsection*{Heur\'istica de inserci\'on}

Las heur\'isticas de inserci\'on son las que ejecutan el siguiente procedimiento:

\begin{algorithm}
\begin{algorithmic}[1]
\State $C \gets getCircuitoLong3()$
\State $S \gets \{ \textrm{v\'ertices de C} \}$
\While{$S \neq V$}
  \State \textbf{\emph{elegir}} un v\'ertice $v$ tal que $v \not\in S$
  \State $S \gets S \cup \{ v \}$
  \State \textbf{\emph{insertar}} $v$ en $C$
\EndWhile
\State \textbf{return} $C$
\end{algorithmic}
\end{algorithm}

L\'ogicamente la particularidad de cada heur\'istica de inserci\'on depende de c\'omo se est\'a eligiendo el v\'ertice y c\'omo se lo est\'a insertando en $C$. Hay diferentes maneras de elegir el v\'ertice:

\begin{itemize}
\item Elegir $v$ al azar.
\item Elegir el $v$ m\'as lejano a un v\'ertice que ya est\'a en $C$.
\item Elegir el $v$ m\'as cercano a un v\'ertice que ya est\'a en $C$.
\item Elegir el $v$ m\'as barato, es decir, que hace crecer menos la longitud de $C$.
\end{itemize}

Para insertar el nodo $v$ elegido, lo que hacemos es:

\begin{enumerate}
\item Sea $c(v_i,v_j)$ el costo de la longitud del eje $(v_i,v_j)$.
\item Elegimos dos v\'ertices $v_i$, $v_{i+1}$ consecutivos en $C$, tales que $C(v_i$, $v_{i+1})$ sea m\'inimo.
\item Insertamos a $v$ entre $v_i$ y $v_{i+1}$.
\end{enumerate}

\subsubsection*{Heur\'istica del \'arbol generador}

Se basa en el siguiente procedimiento:

\begin{enumerate}
\item Encontrar un AGM $T$ de $G$.
\item Duplicar los ejes de $T$.
\item Armar un circuito euleriano $E$ con los ejes de $T$ y sus duplicados.
\item Recorrer $E$ usando DFS y armar un circuito hamiltoniano de $G$.
\end{enumerate}