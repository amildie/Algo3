\newpage
\section{Programaci\'on Din\'amica}

\subsection{Subestructura Optima}

La \emph{``subestructura \'optima''} es una propiedad que pueden exhibir algunos problemas. Se dice que un problema tiene subestructura \'optima si el mismo cumple que una soluci\'on \'optima del mismo puede ser construida a partir de soluciones \'optimas de sus subproblemas.

Un ejemplo de esto es el problema del camino m\'as corto. Supongamos dos puntos $A$ y $B$, y un camino $w$ que es el m\'as corto entre ellos. Para cualquier par de puntos $A'$  y $B'$ dentro de $w$, el camino m\'as corto $w'$ entre ellos est\'a necesariamente contenido adentro de $w$.

\subsection{Soluciones Sobrepuestas}

Al igual que la subestructura \'optima, un problema tiene esta caracter\'istica cuando sub subproblemas comparten soluciones entre ellos. Un ejemplo cl\'asico de este fen\'omeno es el problema de calcular el $n$-\'esimo n\'umero de Fibonacci.

La ecuaci\'on recursiva para calcularlo es $f(n) = f(n-1) + f(n-2)$. Supongamos entonces que queremos calcular $f(5)$. Voy a tener que calcular $f(4)$ y $f(3)$. Pero para calcular $f(4)$ voy a tener que calcular $f(3)$ y $f(2)$. Es decir, voy a tener que calcular $f(3)$ m\'as de una vez.

Uno podr\'ia pensar que ser\'ia una buena idea cachear cada n\'umero de fibonacci calculado para no tener que recalcularlo m\'as de una vez. Esta t\'ecnica se llama \emph{``memoization''}\footnote{S\'i, \emph{memoization}, sin r.}.  

\subsection{Programaci\'on Din\'amica}

Cuando un problema exhibe tanto subestructura \'optima como soluciones sobrepuestas, es candidato a poseer un algoritmo para solucionarlo que emplee una t\'ecnica de desarrollo de algoritmos llamada \emph{``programaci\'on din\'amica''}.

Si un problema puede ser solucionado combinando soluciones no sobrepuestas de sus subproblemas, esta estrategia se llama \emph{``divide \& conquer''}. Es por eso que mergesort, por ejemplo, no es un problema de programaci\'on din\'amica.

\newpage
\subsection{Ejemplos}

\subsubsection{Subsecuencia continua de suma m\'axima}


\noindent\fbox{%
    \parbox{\textwidth}{%
Dado un arreglo $A = \{-6, 2, -4, 1, 3, -1, 5, -1\}$, dicho arreglo tiene varias subsecuencias $s$ continuas, por ejemplo $s_1 = \{1, 3, -1\}$, $s_2 = \{-6\}$, $s_3 = \{-1, 5, -1\}$, etc. Cada una de estas subsecuencias tiene un valor $sum(s_i)$ que representa la suma de todos los elementos de la misma. Encontrar el valor de la subsecuencia de suma m\'axima.
    }%
}

Notar que este problema s\'olo es interesante cuando hay tanto n\'umeros negativos como positivos en el arreglo; ya que si fuesen todos positivos, la soluci\'on es simplemente devolver $sum(A)$, y si fuesen todos negativos, la soluci\'on es devolver el elemento m\'as chico de $A$.

Un ejemplo elemental de esto es, por ejemplo teniendo el arreglo $A = \{ 1, 2, 3, -100, 4, 5, 6\}$. Como la suma de cualquier secuencia continua que no tenga al $-100$ es bastante mayor a la suma de cualquier secuencia que lo tenga, es l\'ogico asumir que la soluci\'on no va a tener al $-100$. Hay dos secuencias continuas de que no lo tienen: $S_1 = \{ 1, 2, 3\}$ y $S_2 = \{ 4, 5, 6\}$. Ac\'a es trivial ver que $sum(S_2)$ es la respuesta al problema.

Pero en el arreglo $A = \{-2, -3, 4, -1, -2, 1, 5, -3\}$ deja de ser tan evidente que el valor buscado es $7$, la suma de la subsecuencia $\{4, -1, -2, 1, 5\}$.

Definimos el array $S$, donde $S_i$ representa a la suma m\'axima de todas las subsecuencias continuas de $A$ que tienen a $A_i$ como \'ultimo elemento. Es decir, $A_i$ tiene que estar, por lo que en cada paso nos interesa saber si vamos a preservar la suma que ven\'iamos armando desde antes o a arrancar con $A_i$ como el inicio de una una subsecuencia nueva. Es decir:

\[
S(i) = \left\{\begin{array}{lr}
    A[i] & \text{si } i = 0\\
    max\{S(i-1) + A[i], A[i] \} & \text{si } i > 0
    \end{array}\right
\]

Entonces el problema se reduce a armar a $S$ mientras vamos buscando el m\'aximo:

\begin{center}
\begin{minipage}{0.78\textwidth}
\begin{lstlisting}[frame=lrtb]
int sum(int* a, unsigned int n) {
  int max = numeric_limits<int>::min();
  int s[n];
  memset(s, -1, sizeof(s));
  for(int i = 0; i < n; ++i) {
    if (i == 0) {
      s[i] = a[i];
    } else {
      s[i] = std::max(a[i], a[i] + s[i-1]);
    }
    if (s[i] > max) {
      max = s[i];
    }    
  }
  return max;
}
\end{lstlisting}
\end{minipage}
\end{center}



\subsubsection{Subsecuencia no-continua estrictamente creciente m\'as larga}

\noindent\fbox{%
    \parbox{\textwidth}{%
Dado un arreglo $A = \{3, 2, 6, 4, 5, 1\}$, encontrar una subsecuencia del mismo estrictamente creciente de longitud m\'axima.
    }%
}

Por ejemplo, para el array del enunciado la respuesta es $A = \{2, 4, 5\}$.

Similarmente al problema anterior, vamos a definir un vector de vectores $S$, donde $S_i$ es la subsecuencia de $A$ que termina en $A_i$. 

\[
S_i = \left\{\begin{array}{lr}
    \{ A_i \} & \text{si } i = 0\\
    max\{S_j \textnormal{ tal que } j < i \textnormal{ y } A_j < A_i\} + A_i & \text{si } i > 0
    \end{array}\right
\]

%Es decir, para calcular la subsecuencia que va a estar en $S_i$, lo que hacemos es fijarnos en todas las subsecuencias que calculamos antes. De todas esas, buscamos las de longitudes m\'aximas. Y de esas, seleccionamos alguna que tenga como \'ultimo elemento a algo m\'as chico que $A_i$ y se lo appendeamos.

\begin{center}
\begin{minipage}{1.15\textwidth}
\begin{lstlisting}[frame=lrtb]
vector<int> lis(int* a, unsigned int n) {
  std::vector< std::vector<int> > L(n);
  L[0].push_back(a[0]);
  vector<int> res;

  for(int i = 1; i < n; ++i) {
    int maxLength = numeric_limits<int>::min();
    int maxIndex = 0;

    for(int j = i-1; j >= 0; --j) {
      if((int)L[j].size() > maxLength && L[j].back() < a[i]) {
        maxLength = L[j].size();
        maxIndex = j;
      }
    }
    
    std::vector<int> v;
    if(maxLength != numeric_limits<int>::min()) {
      v = L[maxIndex];      
    }
    v.push_back(a[i]);
    L[i] = v;
  }

  unsigned int maxLength = numeric_limits<unsigned int>::min();
  for(int i = 0; i < L.size(); ++i) {
    if(L[i].size() > maxLength) {
      res = L[i];
      maxLength = L[i].size();
    }
  }

  return res;
}
\end{lstlisting}
\end{minipage}
\end{center}


\newpage
\subsubsection{El problema de la mochila}