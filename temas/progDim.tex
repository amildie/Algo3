\newpage
\section{Programaci\'on Din\'amica}

\subsection{Subestructura Optima}

La \emph{``subestructura \'optima''} es una propiedad que pueden exhibir algunos problemas. Se dice que un problema tiene subestructura \'optima si el mismo cumple que una soluci\'on \'optima del mismo puede ser construida a partir de soluciones \'optimas de sus subproblemas.

Un ejemplo de esto es el problema del camino m\'as corto. Supongamos dos puntos $A$ y $B$, y un camino $w$ que es el m\'as corto entre ellos. Para cualquier par de puntos $A'$  y $B'$ dentro de $w$, el camino m\'as corto $w'$ entre ellos est\'a necesariamente contenido adentro de $w$.

\subsection{Soluciones Sobrepuestas}

Al igual que la subestructura \'optima, un problema tiene esta caracter\'istica cuando sub subproblemas comparten soluciones entre ellos. Un ejemplo cl\'asico de este fen\'omeno es el problema de calcular el $n$-\'esimo n\'umero de Fibonacci.

La ecuaci\'on recursiva para calcularlo es $f(n) = f(n-1) + f(n-2)$. Supongamos entonces que queremos calcular $f(5)$. Voy a tener que calcular $f(4)$ y $f(3)$. Pero para calcular $f(4)$ voy a tener que calcular $f(3)$ y $f(2)$. Es decir, voy a tener que calcular $f(3)$ m\'as de una vez.

Uno podr\'ia pensar que ser\'ia una buena idea cachear cada n\'umero de fibonacci calculado para no tener que recalcularlo m\'as de una vez. Esta t\'ecnica se llama \emph{``memoization''}\footnote{S\'i, \emph{memoization}, sin r.}.  

\subsection{Programaci\'on Din\'amica}

Cuando un problema exhibe tanto subestructura \'optima como soluciones sobrepuestas, es candidato a poseer un algoritmo para solucionarlo que emplee una t\'ecnica de desarrollo de algoritmos llamada \emph{``programaci\'on din\'amica''}.

Si un problema puede ser solucionado combinando soluciones no sobrepuestas de sus subproblemas, esta estrategia se llama \emph{``divide \& conquer''}. Es por eso que mergesort, por ejemplo, no es un problema de programaci\'on din\'amica.

\newpage
\subsection{Ejemplos}

\subsubsection{Subsecuencia continua de suma m\'axima}


\noindent\fbox{%
    \parbox{\textwidth}{%
Dado un arreglo $A = \{-6, 2, -4, 1, 3, -1, 5, -1\}$, dicho arreglo tiene varias subsecuencias $s$ continuas, por ejemplo $s_1 = \{1, 3, -1\}$, $s_2 = \{-6\}$, $s_3 = \{-1, 5, -1\}$, etc. Cada una de estas subsecuencias tiene un valor $sum(s_i)$ que representa la suma de todos los elementos de la misma. Encontrar el valor de la subsecuencia de suma m\'axima.
    }%
}

Notar que este problema s\'olo es interesante cuando hay tanto n\'umeros negativos como positivos en el arreglo; ya que si fuesen todos positivos, la soluci\'on es simplemente devolver $sum(A)$, y si fuesen todos negativos, la soluci\'on es devolver el elemento m\'as chico de $A$.

Un ejemplo elemental de esto es, por ejemplo teniendo el arreglo $A = \{ 1, 2, 3, -100, 4, 5, 6\}$. Como la suma de cualquier secuencia continua que no tenga al $-100$ es bastante mayor a la suma de cualquier secuencia que lo tenga, es l\'ogico asumir que la soluci\'on no va a tener al $-100$. Hay dos secuencias continuas de que no lo tienen: $S_1 = \{ 1, 2, 3\}$ y $S_2 = \{ 4, 5, 6\}$. Ac\'a es trivial ver que $sum(S_2)$ es la respuesta al problema.

Pero en el arreglo $A = \{-2, -3, 4, -1, -2, 1, 5, -3\}$ deja de ser tan evidente que el valor buscado es $7$, la suma de la subsecuencia $\{4, -1, -2, 1, 5\}$.

Definimos el array $S$, donde $S_i$ representa a la suma m\'axima de todas las subsecuencias continuas de $A$ que tienen a $A_i$ como \'ultimo elemento. Es decir, $A_i$ tiene que estar, por lo que en cada paso nos interesa saber si vamos a preservar la suma que ven\'iamos armando desde antes o a arrancar con $A_i$ como el inicio de una una subsecuencia nueva. Es decir:

\[
S(i) = \left\{\begin{array}{lr}
    A[i] & \text{si } i = 0\\
    max\{S(i-1) + A[i], A[i] \} & \text{si } i > 0
    \end{array}\right
\]

Entonces el problema se reduce a armar a $S$ mientras vamos buscando el m\'aximo:

\begin{center}
\begin{minipage}{0.78\textwidth}
\begin{lstlisting}[frame=lrtb]
int sum(int* a, unsigned int n) {
  int max = numeric_limits<int>::min();
  int s[n];
  memset(s, -1, sizeof(s));
  for(int i = 0; i < n; ++i) {
    if (i == 0) {
      s[i] = a[i];
    } else {
      s[i] = std::max(a[i], a[i] + s[i-1]);
    }
    if (s[i] > max) {
      max = s[i];
    }    
  }
  return max;
}
\end{lstlisting}
\end{minipage}
\end{center}



\subsubsection{Subsecuencia no-continua estrictamente creciente m\'as larga}

\noindent\fbox{%
    \parbox{\textwidth}{%
Dado un arreglo $A = \{3, 2, 6, 4, 5, 1\}$, encontrar una subsecuencia del mismo estrictamente creciente de longitud m\'axima.
    }%
}

Por ejemplo, para el array del enunciado la respuesta es $A = \{2, 4, 5\}$.

Similarmente al problema anterior, vamos a definir un vector de vectores $S$, donde $S_i$ es la subsecuencia de $A$ que termina en $A_i$. 

\[
S_i = \left\{\begin{array}{lr}
    \{ A_i \} & \text{si } i = 0\\
    max\{S_j \textnormal{ tal que } j < i \textnormal{ y } A_j < A_i\} + A_i & \text{si } i > 0
    \end{array}\right
\]

%Es decir, para calcular la subsecuencia que va a estar en $S_i$, lo que hacemos es fijarnos en todas las subsecuencias que calculamos antes. De todas esas, buscamos las de longitudes m\'aximas. Y de esas, seleccionamos alguna que tenga como \'ultimo elemento a algo m\'as chico que $A_i$ y se lo appendeamos.

\begin{center}
\begin{minipage}{1.15\textwidth}
\begin{lstlisting}[frame=lrtb]
vector<int> lis(int* a, unsigned int n) {
  std::vector< std::vector<int> > L(n);
  L[0].push_back(a[0]);
  vector<int> res;

  for(int i = 1; i < n; ++i) {
    int maxLength = numeric_limits<int>::min();
    int maxIndex = 0;

    for(int j = i-1; j >= 0; --j) {
      if((int)L[j].size() > maxLength && L[j].back() < a[i]) {
        maxLength = L[j].size();
        maxIndex = j;
      }
    }
    
    std::vector<int> v;
    if(maxLength != numeric_limits<int>::min()) {
      v = L[maxIndex];      
    }
    v.push_back(a[i]);
    L[i] = v;
  }

  unsigned int maxLength = numeric_limits<unsigned int>::min();
  for(int i = 0; i < L.size(); ++i) {
    if(L[i].size() > maxLength) {
      res = L[i];
      maxLength = L[i].size();
    }
  }

  return res;
}
\end{lstlisting}
\end{minipage}
\end{center}


\newpage
\subsubsection{El problema de la mochila}

El \emph{``problema de la mochila''} es un problema c\'asico de optimizaci\'on combinatoria. Se trata de, dado un conjunto de items, cada uno con un determinado peso y valor, y una mochila que puede soportar hasta cierto peso, encontrar cuantas veces tengo que poner cada \'item para garantizar que la mochila contiene el m\'aximo valor posible.

Este problema tiene diferentes variantes, pero cuando s\'olo tengo un \'item de cada peso, se lo conoce como el \emph{``0-1 knapsack problem''}, que se define formalmente como:

\noindent\fbox{
\parbox{\textwidth}{

Dado un conjunto de $n$ items numerados de $1$ hasta $n$, cada uno con un peso $w_i$ y un valor $v_i$, y una m\'axima capacidad para la mochila $W$:
\[
\textrm{maximizar }\sum_{n=1}^{n} v_ix_i
\textrm{ cumpliendo que } \sum_{n=1}^{n} v_ix_i \leq W \textrm{ y con } x_i \in \{0, 1\}
\]

}
}

La siguiente soluci\'on utilizar programaci\'on din\'amica y corre en tiempo pseudopolinomial. Asumimos que los pesos $w[1], w[2], ..., w[n]$ y el peso m\'aximo de la mochila $W$ son enteros positivos.

Primero definimos una matriz $m[i, w]$ que representa el m\'aximo valor que podemos alcanzar usando hasta el \'item del \'indice $i$ teniendo un peso disponible de $w$. Claramente $m[0,w] = 0$, ya que el valor m\'aximo sin poner ning\'un \'item en la mochila es $0$.

Luego definimos la din\'amica para calcular $m[i,w]$ de la siguiente manera:

\[
m[i,w] = \left\{\begin{array}{lr}
    m[i-1, w] \textnormal{ si } w_i > w\\ 
    max\{ m[i-1,w], m[i-1, w-w_i] + v_i\} \textnormal{ si } w_i \leq w
    \end{array}\right
\]

Por ejemplo, si estamos viendo quinto elemento pueden pasar dos cosas:

\begin{itemize}
\item Que $w[5]$ sea m\'as grande que $w$. Esto significa que ya no nos queda espacio para meter a $w_5$, por lo que no lo metemos, dejando el valor de $m[5, w]$ igual al que ten\'iamos en $m[4,w]$.
\item Que $w[5]$ sea menor o igual a $w$. Esto significa que todav\'ia tenemos espacio suficiente para meter al quinto elemento en la mochila.

Pero que \emph{podamos} meter al quinto elemento no significa que \emph{debamos} hacerlo. Tenemos que elegir el m\'aximo entre \textbf{no} poner al quinto elemento ($m[i-1,w]$) y \textbf{s\'i} ponerlo ($m[i-1, w-w_5] + v_i$). 
\end{itemize}

Es decir, esta din\'amica va reduciendo continuamente la capacidad restante en la mochila llam\'andose recursivamente cada vez con menos espacio disponible, hasta que eventualmente se queda sin lugar.

Para un peso m\'aximo $W$, el valor de la soluci\'on est\'a dado en $m[n,W]$.

\newpage

La siguiente es una implementaci\'on en C++ de la din\'amica anteriormente enunciada. Notar que los arrays son indexados desde $0$ en lugar de $1$.

\begin{center}
\begin{minipage}{1.02\textwidth}
\begin{lstlisting}[frame=lrtb]
void knapsack(int v[], int w[], int n, int W)
{
  int m[2000][2000];
  
  for(int j = 0; j <= W; ++j)
  {
    m[0][j] = 0;
  }

  for(int i = 0; i <= n; ++i)
  {
    for(int j = 0; j <= W; ++j)
    {
      if(w[i] > j)
      {
        m[i][j] = m[i-1][j];
      }
      else
      {
        m[i][j] = max(m[i-1][j], m[i-1][j-w[i]] + v[i]);
      }
    }
  }

  std::cout << "Rta: " << m[n-1][W] << std::endl;
}
\end{lstlisting}
\end{minipage}
\end{center}

Ac\'a queda evidente que el tiempo de ejecuci\'on de la funci\'on depende de los dos ciclos anidados, donde uno va a iterar $n$ veces, siendo $n$ la cantidad de elementos y el otro va a iterar $W$ veces, siendo $W$ el peso m\'aximo de la mochila.

Esto nos deja a la complejidad temporal del algoritmo como $O(nW)$. Notar que la matriz construida tiene que ser de $n$ filas por $W$ columnas, por lo que la complejidad espacial tambi\'en es de $O(nW)$. Es decir, esta soluci\'on es pseudopolinomial en tiempo y espacio.

\newpage
\subsubsection{Subconjunto de suma 0}

Este problema es en realidad un caso especial del problema de la mochila:

\noindent\fbox{
\parbox{\textwidth}{

Dado un array $A = \{a_1, a_2, ..., a_n\}$ de enteros, determinar si existe una subsecuencia no necesariamente continua no vac\'ia de $A$ tal que la suma de todos sus elementos sea 0.

}
}
