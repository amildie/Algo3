\newpage
\section{Flujo en Redes}

Una \emph{``red''} $G = (V, E)$ es un grafo orientado conexo que tiene dos v\'ertices distinguidos: una \emph{``fuente''} $s$, con un grado de salida positivo y un \emph{``sumidero''} $t$ con un grado de entrada positivo.

Una \emph{``funci\'on de capacidades en la red''} es una funci\'on $c : E \rightarrow {\rm I\!R^{\geq 0}}$. La funci\'on de capacidad determina, para cada eje del grafo, la capacidad de transportar flujo que posee.

Un \emph{``flujo factible''} en una red $G = (V, E)$ con funci\'on de capacidad $c$ es una funci\'on $f : E \rightarrow {\rm I\!R^{\geq 0}}$ que verifica:

\begin{itemize}
\item $0 \leq f(e) \leq c(e)$ para todo eje $e \in E$. Es decir, el flujo factible de un eje no puede ser superior a su capacidad.
\item La \emph{``Ley de conservaci\'on de Flujo''}, que dice que dado un v\'ertice $v$, la suma de los flujos de los ejes que llegan a \'el es la misma que la suma de los flujos de los v\'ertices que salen de \'el. Formalmente:

\[
\forall v \in V - \{ s, t \} \textrm{ se cumple que} \sum_{e \in in(v)} f(e) = \sum_{e \in out(v)} f(e)
\]

donde:

\[
in(v) = \{ e \in E, e = (w \rightarrow v), w \in V \}
\]
\[
out(v) = \{ e \in E, e = (v \rightarrow w), w \in V \}
\]


\end{itemize}

El \emph{``valor del flujo''} es:

\[
F = \sum_{e \in in(t)} f(e) - \sum_{e \in out(s)} f(e)
\]

%\textbf{PREGUNTAR SI ESTO ESTA BIEN}.

El problema m\'as com\'un cuando hablamos de flujo es el de encontrar un flujo m\'aximo. Esto es, encontrar un F m\'aximo en una red con una \'unica fuente y un \'unico sumidero.


\raggedright
  \bigskip
  \begin{center}
\begin{tikzpicture}[shorten >=1pt, auto, node distance=3cm,
   node_style/.style={circle},
   edge_style/.style={draw=black}]

\node[vertex] (S) at (0,0)  {s};
\node[vertex] (n1) at (3,2)  {1};
\node[vertex] (n2) at (6,2)  {2};
\node[vertex] (T) at (9,0)  {t};
\node[vertex] (n3) at (3,-2)  {3};
\node[vertex] (n4) at (6,-2)  {4};

    
\draw[edge_style]  (S) edge node{3/\textbf{3}} (n1);

\draw[edge_style]  (S) edge node{2/\textbf{3}} (n3);

\draw[edge_style]  (n1) edge node{0/\textbf{2}} (n3);

\draw[edge_style]  (n1) edge node{3/\textbf{3}} (n2);
\draw[edge_style]  (n3) edge node{2/\textbf{2}} (n4);
\draw[edge_style]  (n2) edge node{1/\textbf{4}} (n4);

\draw[edge_style]  (n2) edge node{2/\textbf{2}} (T);
\draw[edge_style]  (n4) edge node{3/\textbf{3}} (T);



\end{tikzpicture}
\end{center}

Este problema est\'a en P. Notar que un flujo m\'aximo no requiere que todos los ejes est\'en transportando su m\'axima capacidad (ver $(s,3)$). No obstante, la suma de los flujos de $t$ es m\'axima.

\newpage
\subsection{Cortes}

Un \emph{``corte''} en una red $G = (V, E)$ es un subconjunto $S \subseteq V - {t}$ tal que $s \in S$. Es decir, un subconjunto de v\'ertices de la red en los que est\'a la fuente pero no el sumidero.

Dados dos cortes $S$ y $T$, $ST = \{ (v \rightarrow w) \in E \textrm{ tales que } v \in S \textrm{ y } w \in T \}$. Es decir, $ST$ es el conjunto de ejes para los cuales el v\'ertice de salida est\'a en $S$ y el de llegada est\'a en $T$. Notar que $t \in T$. 

Sea $f$ un flujo definido en una red $G = (V, E)$, sea $S$ un corte y sea $\bar{S} = V - S$; entonces:

\[
F = \sum_{e \in S\bar{S}} f(e) - \sum_{e \in \bar{S}S} f(e)
\]

Es decir, el valor del flujo de la red va a ser igual al fujo saliente de un corte menos el flujo entrante al corte. Veamos esto con un ejemplo sencillo, supongamos que tenemos la siguiente red:

\begin{figure}[htb]
    \centering
    \input{cort1.pdf_tex}
\end{figure}

Es trivial ver que $F = 5$. Pero ahora miremos que pasa cuando seleccionamos los v\'ertices $s$ y $1$ para nuestro corte:

\begin{figure}[htb]
    \centering
    \input{cort2.pdf_tex}
\end{figure}

Vemos que la suma de los flujos que van de $S$ a $\bar{S}$ es de $5 + 2$, mientras que los flujos que van de $\bar{S}$ a $S$ suman $2$. Luego $F = 5 + 2 - 2 = 5$, que era lo que quer\'iamos ver.

Notar tambi\'en que esto pasa cuando $S = {s, 1, 2}$, ya que en ese caso $\sum_{e \in S\bar{S}} f(e) = 5$ y $\sum_{e \in \bar{S}S} f(e) = 0$, teniendo como resultado el mismo $F = 5$ de antes.

Tambi\'en se puede ver que, en el caso de que $S$ fuera simplemente el v\'ertice $s$, $F = 3+2$.

\subsubsection{Capacidad de un corte}

Cuando hablamos de un corte $S$, llamamos \emph{``capacidad''} del mismo a la capacidad de todos los ejes que salen de un v\'ertice de $S$ y terminan en un v\'ertice de $\bar{S}$. Dicho formalmente:

\[
c(S) = \sum_{e \in S\bar{S}} c(e)
\]

El \emph{``problema del corte m\'inimo''} consiste en encontrar un corte $S$ que cumple que $c(s)$ es m\'inima.

\subsection{Red Residual}

La \emph{``capacidad residual''} de un eje $e$ respecto de un flujo factible $f$ es la diferencia entre la capacidad de $e$ y su flujo. Es decir: $c_f(e) = c(e) - f(e)$.

Dada una red $G = (V, E)$ con una funci\'on de capacidad $c$ y un flujo factible $f$, definimos a la red que modela la capacidad disponible de $G$ como la \emph{``red residual''} $G_f = (V, E_{R})$, donde para todo $(v \rightarrow w) \in E$:

\begin{enumerate}
\item $(v \rightarrow w) \in E_R \Longleftrightarrow f((v \rightarrow w)) < c((v \rightarrow w))$. Es decir, el eje $e$ va a estar en la red residual si su flujo es menor a su capacidad. El peso de $e$ es $c(e) - f(e)$.
\item $(w \rightarrow v) \in E_R \Longleftrightarrow f((v \rightarrow w)) > 0$. Es decir, todos los ejes con flujo mayor a $0$ van a tener un eje contrario, y el peso de $e$ va a ser igual a $f(e)$.
\end{enumerate}



De esta manera es trivial ver que en $G_f$ pueden haber ejes que no est\'an en $G$. Por ejemplo, miremos el siguiente grafo y su red residual:

\begin{figure}[htb]
    \centering
    \input{fluj1.pdf_tex}
\end{figure}

Los ejes que ten\'ian su m\'axima capacidad saturada desaparecieron por no cumplir con $1.$, mientras que el resto de los ejes se preservaron, y ahora tienen como peso el valor de su capacidad previa. Notar tambi\'en que todos los ejes con un flujo $>0$ generaron ejes inversos (en rojo).


\newpage
\subsection{Camino de Aumento}

Un \emph{``camino de aumento''} es un camino orientado $P$ de $s$ a $t$ en $G_f$. Dada una red $G$, esta se encuentra en su flujo m\'aximo si y s\'olo si no hay camino de aumento en $G_f$. Para el ejemplo anterior tenemos el siguiente camino de aumento:

\begin{figure}[htb]
    \centering
    \input{fluj2.pdf_tex}
\end{figure}

Por lo que es evidente que el flujo $f$ no es m\'aximo. Para encontrar el flujo m\'aximo podemos utilizar el...

\subsection{Algoritmo de Ford-Fulkerson}

La idea del algoritmo es sencilla: mientras exista un camino de aumento, mejorarlo. Viendo el grafo anterior, podemos ver que el camino de aumento $s,1,2,t$ tiene un par de cuellos de botella:

\begin{figure}[htb]
    \centering
    \input{fluj3.pdf_tex}
\end{figure}

Notar que interpretamos como \emph{``camino''} un conjunto de ejes dirigidos que pueden no tener las direcciones necesarias para llegar de $s$ a $t$. 

Luego buscamos la magnitud m\'inima ($1$), se la sumamos a todos los ejes que van en nuestra direcci\'on y se la restamos a los que van en contra:

%Lo que hacemos es buscar la magnitud m\'inima en la que puedo incrementar a alguno de los ejes que van hacia $t$, que en este caso es $1$. Luego le sumo $1$ a todos los ejes que van para mi lado y le resto $1$ a todos los ejes que van en contramano, por lo que el flujo me queda as\'i:

\begin{figure}[htb]
    \centering
    \input{fluj4.pdf_tex}
\end{figure}

\newpage
\subsubsection*{Pseudoc\'odigo}
\begin{algorithm}
\begin{algorithmic}[1]
\Function{FordFulkerson}{$G = (V, E), c: E \rightarrow {\rm I\!R^{\geq 0}}$}
  \ForAll{$e \in E$}
    \State $f(e) \gets 0$
  \EndFor
  \State $G_f \gets calcularRedResidual(G, c, f)$
  \While{$p \gets obtenerCaminoDeAumento(G_f)$}
    \State $c_{min} \gets min\{c_f(e) \textrm{ tal que } e \in p\}$
    \ForAll{$(v,w) \in p$}
        \State $f(v,w) \gets f(v,w) + c_{min}$
        \State $f(w,v) \gets f(w,v) - c_{min}$ 
    \EndFor
  \EndWhile
\EndFunction
\end{algorithmic}
\end{algorithm}

\subsubsection*{Explicaci\'on}

El algoritmo utiliza fuertemente la propiedad de que, mientras exista un camino de aumento en $G_f$, $f$ no va a ser un flujo m\'aximo. 

\begin{enumerate}
\item [2: y 3:] Inicialmente el algoritmo genera un flujo factible $f$, donde todos los flujos para todos los ejes son $0$.
\item [4:] Luego calcula la red residual $G_f$, utilizando la red original $G$, el flujo factible $f$ y la funci\'on de capacidad $c$.
\item [5:] Ac\'a comienza la iteraci\'on principal del algoritmo. Mientras pueda encontrar un camino de aumento para asign\'arselo a $p$, el flujo $f$ no es m\'aximo y va a poder ser mejorado haciendo lo siguiente:
  \begin{enumerate}
  \item [6:] Dado el camino $p$ que encontr\'e, busco el eje de menor capacidad residual del mismo. Esta capacidad residual se llama $c_{min}$.
  \item [7:] Empiezo a recorrer todos los ejes del camino $p$, y:
    \begin{enumerate}
      \item [8:] A los ejes que van para mi lado, le sumo $c_{min}$.
      \item [9:] A los que van para el otro lado, les resto $c_{min}$.
    \end{enumerate}
  \end{enumerate}
\end{enumerate}

Si los valores del flujo inicial y las capacidades de los ejes son enteros este algoritmo realiza a lo sumo $|V| * U$ iteraciones, siendo entonces su complejidad del orden de $\mathcal{O}(|E| * |V| * U)$, con $U$ siendo una cota superior finita para el valor de las capacidades.

\subsubsection{Algoritmo de Edmonds-Karp}

Este algoritmo es en realidad una implementaci\'on de Ford Fulkerson que usa espec\'ificamente BFS para encontrar el camino de aumento $p$ en la funci\'on $obtenerCaminoDeAumento(G_f)$. Esto mejora la complejidad, dej\'andola en $\mathcal{O}(|V|^2 * |E|)$.
