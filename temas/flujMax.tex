\newpage
\section{Flujo en Redes}

Una \emph{``red''} $G = (V, E)$ es un grafo orientado conexo que tiene dos v\'ertices distinguidos: una \emph{``fuente''} $s$, con un grado de salida positivo y un \emph{``sumidero''} $t$ con un grado de entrada positivo.

Una \emph{``funci\'on de capacidades en la red''} es una funci\'on $c : E \rightarrow {\rm I\!R^{\geq 0}}$. La funci\'on de capacidad determina, para cada eje del grafo, la capacidad de transportar flujo que posee.

Un \emph{``flujo factible''} en una red $G = (V, E)$ con funci\'on de capacidad $c$ es una funci\'on $f : E \rightarrow {\rm I\!R^{\geq 0}}$ que verifica:

\begin{itemize}
\item $0 \leq f(e) \leq c(e)$ para todo eje $e \in E$. Es decir, el flujo factible de un eje no puede ser superior a su capacidad.
\item La \emph{``Ley de conservaci\'on de Flujo''}, que dice que dado un v\'ertice $v$, la suma de los flujos de los ejes que llegan a \'el es la misma que la suma de los flujos de los v\'ertices que salen de \'el. Formalmente:

\[
\forall v \in V - \{ s, t \} \textrm{ se cumple que} \sum_{e \in in(v)} f(e) = \sum_{e \in out(v)} f(e)
\]

donde:

\[
in(v) = \{ e \in E, e = (w \rightarrow v), w \in V \}
\]
\[
out(v) = \{ e \in E, e = (v \rightarrow w), w \in V \}
\]


\end{itemize}

El \emph{``valor del flujo''} es:

\[
F = \sum_{e \in in(t)} f(e) - \sum_{e \in out(s)} f(e)
\]

%\textbf{PREGUNTAR SI ESTO ESTA BIEN}.

El problema m\'as com\'un cuando hablamos de flujo es el de encontrar un flujo m\'aximo. Esto es, encontrar un F m\'aximo en una red con una \'unica fuente y un \'unico sumidero.


\raggedright
  \bigskip
  \begin{center}
\begin{tikzpicture}[shorten >=1pt, auto, node distance=3cm,
   node_style/.style={circle},
   edge_style/.style={draw=black}]

\node[vertex] (S) at (0,0)  {s};
\node[vertex] (n1) at (3,2)  {1};
\node[vertex] (n2) at (6,2)  {2};
\node[vertex] (T) at (9,0)  {t};
\node[vertex] (n3) at (3,-2)  {3};
\node[vertex] (n4) at (6,-2)  {4};

    
\draw[edge_style]  (S) edge node{3/\textbf{3}} (n1);

\draw[edge_style]  (S) edge node{2/\textbf{3}} (n3);

\draw[edge_style]  (n1) edge node{0/\textbf{2}} (n3);

\draw[edge_style]  (n1) edge node{3/\textbf{3}} (n2);
\draw[edge_style]  (n3) edge node{2/\textbf{2}} (n4);
\draw[edge_style]  (n2) edge node{1/\textbf{4}} (n4);

\draw[edge_style]  (n2) edge node{2/\textbf{2}} (T);
\draw[edge_style]  (n4) edge node{3/\textbf{3}} (T);



\end{tikzpicture}
\end{center}

Este problema est\'a en P. Notar que un flujo m\'aximo no requiere que todos los ejes est\'en transportando su m\'axima capacidad (ver $(s,3)$). No obstante, la suma de los flujos de $t$ es m\'axima.

\newpage
\subsection{Cortes}

Un \emph{``corte''} en una red $G = (V, E)$ es un subconjunto $S \subseteq V - {t}$ tal que $s \in S$. Es decir, un subconjunto de v\'ertices de la red en los que est\'a la fuente pero no el sumidero.

Dados dos cortes $S$ y $T$, $ST = \{ (v \rightarrow w) \in E \textrm{ tales que } v \in S \textrm{ y } w \in T \}$. Es decir, $ST$ es el conjunto de ejes para los cuales el v\'ertice de salida est\'a en $S$ y el de llegada est\'a en $T$.

Sea $f$ un flujo definido en una red $N = (V, E)$, sea $S$ un corte y sea $\bar{S} = V - S$; entonces:

\[
F = \sum_{e \in S\bar{S}} f(e) - \sum_{e \in \bar{S}S} f(e)
\]

%Es decir, el valor del flujo de la red va a ser igual al fujo saliente de un corte menos el flujo entrante al corte?

\subsection{Red Residual}

La \emph{``capacidad residual''} de un eje $e$ respecto de un flujo factible $f$ es la diferencia entre la capacidad de $e$ y su flujo. Es decir: $c_f(e) = c(e) - f(e)$.

Dada una red $G = (V, E)$ con una funci\'on de capacidad $c$ y un flujo factible $f$, definimos a la \emph{``red residual''} $G_f = (V, E_{R})$, como la red que modela la capacidad disponible de $G$. Es decir, $e \in E_{R} \Longleftrightarrow c_f(e) > 0$. 

\subsection{Camino de Aumento}

Un \emph{``camino de aumento''} es un camino orientado $P$ de $s$ a $t$ en $G_f$. Dada una red $G$, esta se encuentra en su flujo m\'aximo si y s\'olo si no hay camino de aumento en $G_f$.

\subsection{Algoritmo de Ford-Fulkerson}

\begin{algorithm}
\begin{algorithmic}[1]
\Function{FordFulkerson}{$G = (V, E), c: E \rightarrow {\rm I\!R^{\geq 0}}$}
  \ForAll{$e \in E$}
    \State $f(e) \gets 0$
  \EndFor
  \State $G_f \gets calcularRedResidual(G)$
  \While{$p \gets obtenerCaminoDeAumento(G_f)$}
    \State $c_{min} \gets min\{c_f(e) \textrm{ tal que } e \in p\}$
    \ForAll{$(v,w) \in p$}
        \State $f(v,w) \gets f(v,w) + c_{min}$
        \State $f(w,v) \gets f(w,v) - c_{min}$ 
    \EndFor
  \EndWhile
\EndFunction
\end{algorithmic}
\end{algorithm}

\subsubsection{Algoritmo de Edmonds-Karp}

Este algoritmo es en realidad una implementaci\'on de Ford Fulkerson que usa espec\'ificamente BFS para encontrar el camino de aumento $p$ en la funci\'on $obtenerCaminoDeAumento(G_f)$. Esto mejora la complejidad, dej\'andola en $\mathcal{O}(|V|^2 * |E|)$.
