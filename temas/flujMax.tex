\newpage
\section{Flujo en Redes}

Una \emph{``red''} $N = (V, E)$ es un grafo orientado conexo que tiene dos nodos distinguidos: una \emph{``fuente''} $s$, con un grado de salida positivo y un \emph{``sumidero''} $t$ con un grado de entrada positivo.

Una \emph{``funci\'on de capacidades en la red''} es una funci\'on $c : E \rightarrow {\rm I\!R^{\geq 0}}$. La funci\'on de capacidad determina, para cada eje del grafo, la capacidad de transportar flujo que posee.

Un \emph{``flujo factible''} en una red $N = (V, E)$ con funci\'on de capacidad $c$ es una funci\'on $f : E \rightarrow {\rm I\!R^{\geq 0}}$ que verifica:

\begin{itemize}
\item $0 \leq f(e) \leq c(e)$ para todo eje $e \in E$. Es decir, el flujo factible de un eje no puede ser superior a su capacidad.
\item La \emph{``Ley de conservaci\'on de Flujo''}, que dice que dado un v\'ertice $v$, la suma de los flujos de los ejes que llegan a \'el es la misma que la suma de los flujos de los v\'ertices que salen de \'el. Formalmente:

\[
\forall v \in V - \{ s, t \} \textrm{ se cumple que} \sum_{e \in in(v)} f(e) = \sum_{e \in out(v)} f(e)
\]

donde:

\[
in(v) = \{ e \in E, e = (w \rightarrow v), w \in V \}
\]
\[
out(v) = \{ e \in E, e = (v \rightarrow w), w \in V \}
\]


\end{itemize}

El \emph{``valor del flujo''} es:

\[
F = \sum_{e \in in(t)} f(e) - \sum_{e \in out(s)} f(e)
\]

%\textbf{PREGUNTAR SI ESTO ESTA BIEN}.

El problema m\'as com\'un cuando hablamos de flujo es el de encontrar un flujo m\'aximo. Esto es, encontrar un F m\'aximo en una red con una \'unica fuente y un \'unico sumidero. Este problema est\'a en P.


\raggedright
  \bigskip
  \begin{center}
\begin{tikzpicture}[shorten >=1pt, auto, node distance=3cm,
   node_style/.style={circle},
   edge_style/.style={draw=black},
   edge_styly/.style={->,> = latex'}]

\node[vertex] (S) at (0,0)  {s};
\node[vertex] (n1) at (3,2)  {1};
\node[vertex] (n2) at (6,2)  {2};
\node[vertex] (T) at (9,0)  {t};
\node[vertex] (n3) at (3,-2)  {3};
\node[vertex] (n4) at (6,-2)  {4};

    
\draw[edge_style]  (S) edge node{4} (n1);
\draw[edge_style]  (S) edge node{4} (n3);

\draw[edge_style]  (n1) edge node{4} (n3);
\draw[edge_style]  (n1) edge node{4} (n3);

\draw[edge_style]  (n1) edge node{4} (n2);
\draw[edge_style]  (n3) edge node{4} (n4);
\draw[edge_style]  (n2) edge node{4} (n4);

\draw[edge_style]  (n2) edge node{4} (T);
\draw[edge_styly]  (n4) edge node{4} (T);



\end{tikzpicture}
\end{center}

\newpage
\subsection{Cortes}


\subsection{Camino de Aumento}
\subsection{Algoritmo de Ford-Fulkerson}
\subsection{Algoritmo de Edmonds-Karp}
