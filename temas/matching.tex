\newpage
\section{Matchings y Conjuntos Independientes}

Dado un grafo $G = (V, E)$ tenemos las siguientes definiciones:

\begin{itemize}
\item Un \emph{``matching''} es un conjunto $M \subseteq E$ de ejes de $G$ tal que para todo v\'ertice $v \in V$, $v$ es incidente \textbf{a lo sumo} de un eje $e \in M$. Se dice entonces que $v$ est\'a \emph{``saturado''} por $M$.

\raggedright
\bigskip
\begin{center}
\begin{tikzpicture}[shorten >=1pt, auto, node distance=3cm,
  edge_red/.style={draw=red, ultra thick}]

\node[vertex] (n1) at (-7,1)  {1};
\node[vertex] (n2) at (-3,1)  {2};
\node[vertex] (n3) at (-7,-1)  {3};
\node[vertex] (n4) at (-3,-1)  {4};
\node[vertex] (n5) at (-5,0)  {5};


\draw[edge_red](n2) -- (n4);
\draw(n1) -- (n3);
\draw(n1) -- (n2);
\draw[edge_red](n3) -- (n5);
\draw(n3) -- (n4);
\draw(n5) -- (n4);

\end{tikzpicture}
\end{center}
\raggedright
\bigskip


\item Un \emph{``conjunto independiente''} es un conjunto de $I \subseteq V$ tal que para todo eje $e \in E$, $e$ es incidente \textbf{a lo sumo} a un v\'ertice $v$ de $I$.

\raggedright
\bigskip
\begin{center}
\begin{tikzpicture}[shorten >=1pt, auto, node distance=3cm,
  vertex_red/.style={fill=red!35, circle}]

\node[vertex_red] (n1) at (-7,1)  {1};
\node[vertex] (n2) at (-3,1)  {2};
\node[vertex] (n3) at (-7,-1)  {3};
\node[vertex_red] (n4) at (-3,-1)  {4};
\node[vertex] (n5) at (-5,0)  {5};


\draw(n2) -- (n4);
\draw(n1) -- (n3);
\draw(n1) -- (n2);
\draw(n3) -- (n5);
\draw(n3) -- (n4);
\draw(n5) -- (n4);

\end{tikzpicture}
\end{center}
\raggedright
\bigskip

\item Un \emph{``recubrimiento de los v\'ertices''} de $G$ es un conjunto $R_e$ de ejes tales que para todo $v \in V$, $v$ es incidente a \textbf{al menos} un eje $e \in R_e$:

\raggedright
\bigskip
\begin{center}
\begin{tikzpicture}[shorten >=1pt, auto, node distance=3cm,
  edge_red/.style={draw=red, ultra thick}]

\node[vertex] (n1) at (-7,1)  {1};
\node[vertex] (n2) at (-3,1)  {2};
\node[vertex] (n3) at (-7,-1)  {3};
\node[vertex] (n4) at (-3,-1)  {4};
\node[vertex] (n5) at (-5,0)  {5};


\draw(n2) -- (n4);
\draw[edge_red](n1) -- (n3);
\draw[edge_red](n1) -- (n2);
\draw(n3) -- (n5);
\draw(n3) -- (n4);
\draw[edge_red](n5) -- (n4);

\end{tikzpicture}
\end{center}
\raggedright
\bigskip

\item Un \emph{``recubrimiento de los ejes''} de $G$ es un conjunto $R_v$ de v\'ertices tal que para todo $e \in E$, $e$ es incidente a \textbf{al menos} un v\'ertice $v \in R_v$.

\raggedright
\bigskip
\begin{center}
\begin{tikzpicture}[shorten >=1pt, auto, node distance=3cm,
  vertex_red/.style={fill=red!35, circle}]

\node[vertex_red] (n1) at (-7,1)  {1};
\node[vertex] (n2) at (-3,1)  {2};
\node[vertex] (n3) at (-7,-1)  {3};
\node[vertex_red] (n4) at (-3,-1)  {4};
\node[vertex_red] (n5) at (-5,0)  {5};


\draw(n2) -- (n4);
\draw(n1) -- (n3);
\draw(n1) -- (n2);
\draw(n3) -- (n5);
\draw(n3) -- (n4);
\draw(n5) -- (n4);

\end{tikzpicture}
\end{center}
\raggedright
\bigskip


\end{itemize}

$S \subseteq V$ es un conjunto independiente si y s\'olo si $V - S$ es un recubrimiento de ejes.

\subsection{Matching maximal}

Un \emph{``matching maximal''} $M$ en un grafo $G$ es un matching que cumple la propiedad de que si se le agrega un eje cualquiera $M$ deja de ser un matching. Es decir, pasa a existir un v\'ertice $v$ que es incidente a dos ejes de $M$, el que le agregu\'e y uno que ya estaba. 

\raggedright
\bigskip
\begin{center}
\begin{tikzpicture}[shorten >=1pt, auto, node distance=3cm,
  edge_red/.style={draw=red, ultra thick}]

\node[vertex] (n1) at (0,0)  {1};
\node[vertex] (n2) at (1,1)  {2};
\node[vertex] (n3) at (1,-1)  {3};
\node[vertex] (n4) at (2.5,1)  {4};
\node[vertex] (n5) at (2.5,-1)  {5};
\node[vertex] (n6) at (4,1)  {6};


\draw(n1) -- (n2);
\draw[edge_red](n1) -- (n3);
\draw(n2) -- (n3);
\draw[edge_red](n2) -- (n4);
\draw(n4) -- (n5);
\draw(n3) -- (n5);
\draw(n4) -- (n6);

\end{tikzpicture}
\end{center}
\raggedright
\bigskip


\subsection{Matching m\'aximo}

Un \emph{``matching m\'aximo''} es un matching que contiene la mayor cantidad posible de ejes. El \emph{``n\'umero de matching''} $\nu(G)$ de un grafo $G$ es el tama\~no de su matching m\'aximo. Todo matching m\'aximo es maximal, pero no todo matching maximal es m\'aximo, como se puede ver al comparar los dibujos de cada uno.

\raggedright
\bigskip
\begin{center}
\begin{tikzpicture}[shorten >=1pt, auto, node distance=3cm,
  edge_red/.style={draw=red, ultra thick}]

\node[vertex] (n1) at (0,0)  {1};
\node[vertex] (n2) at (1,1)  {2};
\node[vertex] (n3) at (1,-1)  {3};
\node[vertex] (n4) at (2.5,1)  {4};
\node[vertex] (n5) at (2.5,-1)  {5};
\node[vertex] (n6) at (4,1)  {6};


\draw[edge_red](n1) -- (n2);
\draw(n1) -- (n3);
\draw(n2) -- (n3);
\draw(n2) -- (n4);
\draw(n4) -- (n5);
\draw[edge_red](n3) -- (n5);
\draw[edge_red](n4) -- (n6);

\end{tikzpicture}
\end{center}
\raggedright
\bigskip

El problema de encontrar un matching m\'aximo en un grafo arbitrario est\'a bien resuelto en $P$, con el algoritmo Blossom que tiene un costo de $O(V^{2}E)$. Como un matching m\'aximo tambi\'en es maximal, es posible encontrar en el matching maximal m\'as grande en tiempo polinomial.

No obstante, todav\'ia no hay ning\'un algoritmo polinomial para encontrar el matching maximal m\'inimo, esto es, un matching maximal que contenga la menor cantidad de ejes.

\newpage
\subsection{Caminos}

Dado un matching $M$ en $G$, definimos:

\begin{itemize}
\item Un \textbf{\emph{``camino alternado''}} en $G$ con respecto a $M$, como un camino simple donde se alternan ejes que est\'an en $M$ con ejes que no lo est\'an. Para el gra\'afico anterior, un ejemplo de camino alternado es $6,4,2,1,3,5$.
\item Un \textbf{\emph{``camino de aumento''}} en $G$ con respecto a $M$, como un camino alternado entre v\'ertices no saturados por $M$. Es decir, es un camino entre v\'ertices que si y que no tocan ejes de $M$. Por ejemplo, el camino $3,5,4,6$ en el siguiente matching:

\raggedright
\bigskip
\begin{center}
\begin{tikzpicture}[shorten >=1pt, auto, node distance=3cm,
  edge_red/.style={draw=red, ultra thick}]

\node[vertex] (n1) at (0,0)  {1};
\node[vertex] (n2) at (1,1)  {2};
\node[vertex] (n3) at (1,-1)  {3};
\node[vertex] (n4) at (2.5,1)  {4};
\node[vertex] (n5) at (2.5,-1)  {5};
\node[vertex] (n6) at (4,1)  {6};


\draw(n1) -- (n2);
\draw[edge_red](n1) -- (n3);
\draw(n2) -- (n3);
\draw[edge_red](n2) -- (n4);
\draw(n4) -- (n5);
\draw(n3) -- (n5);
\draw(n4) -- (n6);

\end{tikzpicture}
\end{center}
\raggedright
\bigskip

\end{itemize}

\subsection{Propiedades}

\begin{itemize}
\item $M$ es un matching m\'aximo de $G$ $\Leftrightarrow$ no existe un camino de aumento en $G$ con respecto a $M$.
\item Dado un grafo $G$ sin v\'ertices aislados, si $M$ es un matching m\'aximo de $G$ y $R_e$ un recubrimiento m\'inimo de los v\'ertices de $G$ $\Rightarrow$ $|M| + |R_e| = n$.
\item Dado un grafo $G$, si $I$ es un conjunto independiente m\'aximo de $G$ y $R_n$ un recubrimiento m\'inimo de los ejes de $G$ $\Rightarrow$ $|I| + |R_n| = n$.
\item Sean $M_0$ y $M_1$ dos matchings en $G$, y sea $G' = (V, E')$ con $E' = (M_0 - M_1) \cup (M_1 - M_0)$ $\Rightarrow$ las componentes conexas de $G'$ son de alguno de los siguientes tipos:

\begin{itemize}
\item V\'ertice aislado
\item Circuito simple con ejes alternadamente entre $M_0$ y $M_1$.
\item Camino simple con ejes alternadamente entre $M_0$ y $M_1$.
\end{itemize}

\end{itemize}


