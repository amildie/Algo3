\newpage
\section{Teor\'ia de Complejidad}

Denominamos como \emph{``problema de decisi\'on''} a los problemas cuya respuesta es \emph{``s\'i''} o \emph{``no''}. El objetivo de esta teor\'ia es el de clasificar a este tipo de problemas seg\'un su complejidad.

Un algoritmo \emph{``eficiente''} es un algoritmo de complejidad polinomial, y decimos que un problema est\'a \emph{``bien resuelto''} si se conocen algoritmos eficientes para resolverlo. Estos problemas pertenecen a la clase $P$. Por ejemplo, el problema $\Pi$ encontrar un v\'ertice $v$ en un grafo de $n$ v\'ertices puede resolverse en tiempo $O(n)$ usando DFS, por lo que $\Pi \in P$.

\subsection{La clase NP}

Un problema de decisi\'on $\Pi$ pertenece a la clase $NP$ (no-det\'erministico polinomial) si dada cualquier instancia del mismo para las cuales la respuesta es \emph{s\'i} y un \emph{``certificado''}, podemos chequear que dicho certificado es correcto usando un algoritmo de tiempo polinomial.

Veamos por ejemplo el problema $\Pi$ que consiste en: \emph{``dado un array $A$ de $n$ enteros, existe una subsecuencia del mismo que sume 0?''}. Una instancia del problema puede ser el array $A = \{-7, -3, -2, 8, 5\}$ y un certificado puede ser $c = \{-3, -2, 5\}$. Como es trivial ver que la suma de todos los elementos de $c$ es $0$ en tiempo polinomial, entonces tenemos que $\Pi \in NP$.

Es trivial darse cuenta que $P$ est\'a contenido en $NP$, pero la gran inc\'ognita en la teor\'ia de la complejidad es determinar si $P = NP$. Es decir, si para cada problema en $NP$ existe una soluci\'on polinomial.

\subsection{La clase NP-Complete}

NP-Complete es una subclase de NP, pero con la particularidad de que todos los problemas en NP pueden ser reducible en tiempo polinomial a cualquier problema de NP-Complete.

\subsection{La clase NP-Hard}


