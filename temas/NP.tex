\newpage
\section{Teor\'ia de Complejidad}

Denominamos como \emph{``problema de decisi\'on''} a los problemas cuya respuesta es \emph{``s\'i''} o \emph{``no''}. El objetivo de esta teor\'ia es el de clasificar a este tipo de problemas seg\'un su complejidad.

\subsection{La clase P}

La clase P es donde viven los problemas cuyas soluciones tienen complejidad \emph{``polinomial''}. Decimos que un algoritmo corre en tiempo polinomial si su complejidad es $\mathcal{O}(n^k)$, donde $n$ denota el n\'umero de bits de entrada en el algoritmo.

Un algoritmo \emph{``eficiente''} es un algoritmo de complejidad polinomial, y decimos que un problema est\'a \emph{``bien resuelto''} si se conocen algoritmos eficientes para resolverlo. Por ejemplo, el problema $\Pi$ de determinar si el v\'ertice $v$ se encuentra en un grafo $G$ puede resolverse en tiempo $\mathcal{O}(|V|+|E|)$ usando DFS, por lo que $\Pi \in$ P.

\subsection{La clase NP}

Un problema de decisi\'on $\Pi$ pertenece a la clase NP (no-det\'erministico polinomial) si dada cualquier instancia del mismo para las cuales la respuesta es \emph{s\'i} y un \emph{``certificado''}, podemos chequear que dicho certificado es correcto usando un algoritmo de tiempo polinomial.

Veamos por ejemplo el problema $\Pi$ que consiste en: \emph{``dado un array $A$ de $n$ enteros, existe una subsecuencia del mismo que sume 0?''}. Una instancia del problema puede ser el array $A = \{-7, -3, -2, 8, 5\}$ y un certificado puede ser $c = \{-3, -2, 5\}$. Como es trivial ver que la suma de todos los elementos de $c$ es $0$ en tiempo polinomial, entonces tenemos que $\Pi \in$ NP.

Es trivial darse cuenta que P est\'a contenido en NP, pero la gran inc\'ognita en la teor\'ia de la complejidad es determinar si P = NP. Es decir, si para cada problema en NP existe una soluci\'on polinomial.


\subsubsection{Reducciones Polinomiales}

Una \emph{``reducci\'on polinomial''} de un problema de decisi\'on $\Pi_1$ a uno $\Pi_2$ es una funci\'on polinomial que transforma una instancia  $i_1$ de $\Pi_1$ en una instancia $i_2$ de $\Pi_2$, tal que $i_1$ tiene respuesta \emph{``s\'i''} en $\Pi_1$ si y s\'olo si $i_2$ tiene respuesta \emph{``s\'i''} en $\Pi_2$.

El problema de decisi\'on $\Pi_1$ se reduce polinomialmente a otro problema de decisi\'on $\Pi_2$ ($\Pi_1 \leq_p \Pi_2$) si existe una transformaci\'on polinomial de $\Pi_1$ a $\Pi_2$.

\subsection{La clase NP-Completo}

\emph{``NP-Completo''} es una subclase de NP, pero con la particularidad de que todos los problemas en NP pueden ser reducibles en tiempo polinomial a cualquier problema de NP-Completo. Formalmente, un problema de decisi\'on $\Pi$ es NP-Completo si:

\begin{itemize}
\item $\Pi \in \textrm{NP}$
\item $\forall\ \Pi' \in \textrm{NP}, \Pi' \leq_p \Pi$.
\end{itemize} 

Es importante notar que la operaci\'on de reducci\'on es transitiva. Si $\Pi$ es un problema de decisi\'on, podemos probar que $\Pi \in$ NP-Completo encontrando otro problema $\Pi'$ que ya sabemos que es NP-Completo y demostrando que:

\begin{itemize}
\item $\Pi \in \textrm{NP}$
\item $\Pi' \leq_p \Pi$
\end{itemize} 

\subsection{El problema SAT}

El problema SAT (\emph{``problema de satisfacibilidad booleana''}) consiste en, dada una f\'ormula booleana, decidir si existe una manera de reemplazarla con valores de verdadero o falso de manera que dicha f\'ormula devuelva verdadero.

\subsubsection{Teorema de Cook}

El Teorema de Cook dice que SAT es NP-Completo. Es decir, que todo problema en NP puede ser reducido en tiempo polinomial al problema de decidir si una f\'ormula booleana es satisfacible.

Una consecuencia importante de este problema es que si existe un algoritmo de tiempo polinomial para resolver SAT, entonces existe un algoritmo polinomial para todos los problemas de NP, lo que demostrar\'ia que P = NP.


\subsection{La clase NP-Intermedio}

Esta es la clase de problemas de NP que no est\'an ni en P ni en NP-Completo. El Teorema de Ladner dice que, si P $\neq$ NP, entonces NP-i no es vac\'io. Es decir, P = NP si y s\'olo si NP-i es vac\'io.

Hasta ahora no se conoce ning\'un problema de NP-i. De existir uno, implicar\'ia que P $\neq$ NP, lo cual no se sabe, pero se sospecha ampliamente. Un ejemplo de un problema que se sospecha que podr\'ia estar en NP-i es el de la factorizaci\'on de n\'umeros naturales.

\newpage
\subsection{La clase NP-Hard}

Similarmente a la clase NP-Completo, un problema de decisi\'on $\Pi$ es \emph{``NP-hard''} si todo problema de NP se puede reducir polinomialmente a $\Pi$, \textbf{pero $\Pi$ no tiene que necesariamente estar en NP}.Es decir, la clase NP-Completo est\'a contenida dentro de la clase NP-Hard. 

Todos los problemas NP-Completos son NP-Hard, pero no todos los problemas NP-Hard son NP-Completos, por ejemplo, el Halting Problem.

Hasta el d\'ia de la fecha no se pudo demostrar que P $\neq$ NP, pero se sospecha fuertemente que este es el caso. De serlo as\'i, el siguiente diagrama ilustra las relaciones entre las clases de problemas descriptas anteriormente:

\begin{figure}[htb]
    \centering
    %LaTeX with PSTricks extensions
%%Creator: inkscape 0.48.3.1
%%Please note this file requires PSTricks extensions
\psset{xunit=.5pt,yunit=.5pt,runit=.5pt}
\begin{pspicture}(334.08355713,327.86865234)
{
\newrgbcolor{curcolor}{0 0 0}
\pscustom[linewidth=1.06299218,linecolor=curcolor]
{
\newpath
\moveto(272.93667533,106.86394161)
\curveto(272.93667533,48.13815809)(224.5861731,0.53150434)(164.94280598,0.53150434)
\curveto(105.29943886,0.53150434)(56.94893663,48.13815809)(56.94893663,106.86394161)
\curveto(56.94893663,165.58972512)(105.29943886,213.19637888)(164.94280598,213.19637888)
\curveto(224.5861731,213.19637888)(272.93667533,165.58972512)(272.93667533,106.86394161)
\closepath
}
}
{
\newrgbcolor{curcolor}{0 0 0}
\pscustom[linewidth=0.50179891,linecolor=curcolor,linestyle=dashed,dash=1.13295039 1.13295039]
{
\newpath
\moveto(216.36435028,59.90311959)
\curveto(216.36435028,32.18087112)(193.53988304,9.70754842)(165.38447584,9.70754842)
\curveto(137.22906864,9.70754842)(114.4046014,32.18087112)(114.4046014,59.90311959)
\curveto(114.4046014,87.62536806)(137.22906864,110.09869077)(165.38447584,110.09869077)
\curveto(193.53988304,110.09869077)(216.36435028,87.62536806)(216.36435028,59.90311959)
\closepath
}
}
{
\newrgbcolor{curcolor}{0 0 0}
\pscustom[linestyle=none,fillstyle=solid,fillcolor=curcolor,opacity=0.43809521]
{
\newpath
\moveto(23.74141947,316.18758769)
\curveto(46.22191018,252.94252858)(68.70265955,189.69672892)(103.72528461,158.07349247)
\curveto(138.74790966,126.45029146)(186.31065646,126.45029146)(224.17071578,157.60294128)
\curveto(262.03077511,188.75562655)(290.18706271,251.05855219)(304.2651888,282.20992643)
\lineto(318.34327945,313.36130421)
}
}
{
\newrgbcolor{curcolor}{0 0 0}
\pscustom[linewidth=0.77609275,linecolor=curcolor]
{
\newpath
\moveto(23.74141947,316.18758769)
\curveto(46.22191018,252.94252858)(68.70265955,189.69672892)(103.72528461,158.07349247)
\curveto(138.74790966,126.45029146)(186.31065646,126.45029146)(224.17071578,157.60294128)
\curveto(262.03077511,188.75562655)(290.18706271,251.05855219)(304.2651888,282.20992643)
\lineto(318.34327945,313.36130421)
}
}
{
\newrgbcolor{curcolor}{0 0 0}
\pscustom[linewidth=0.26929132,linecolor=curcolor,linestyle=dashed,dash=0 3.15968494]
{
\newpath
\moveto(160.25126557,10.326074)
\curveto(154.61200622,10.85670837)(148.49011251,12.61119274)(143.08655941,15.24533335)
\curveto(112.40212833,30.20344577)(105.21802837,69.79709868)(128.76290949,94.18714542)
\curveto(144.45847815,110.44611408)(169.07267177,114.40001093)(189.12260916,103.88301412)
\curveto(203.43454033,96.37584228)(213.22365278,82.58418611)(215.50766527,66.70970494)
\curveto(215.94694026,63.65663309)(216.01578089,57.11550812)(215.64057464,54.08019251)
\curveto(213.28562153,35.02900199)(200.05006223,18.89622083)(181.77262795,12.79867399)
\curveto(174.76509674,10.4608865)(167.58514053,9.63597088)(160.25126557,10.326074)
\closepath
}
}
{
\newrgbcolor{curcolor}{0 0 0}
\pscustom[linestyle=none,fillstyle=solid,fillcolor=curcolor,opacity=0]
{
\newpath
\moveto(150.22929687,13.16502086)
\curveto(138.63085631,17.13379584)(130.05692198,23.51605205)(123.86300014,32.79151763)
\curveto(114.05805332,47.47453942)(112.80276895,65.64850183)(120.52711266,81.08853612)
\curveto(139.11268443,118.23891404)(191.65049352,118.23891404)(210.23607467,81.08853612)
\curveto(223.92789647,53.72017377)(208.30063406,20.31908332)(178.1269811,12.45984274)
\curveto(169.83237489,10.299374)(157.70830621,10.60584275)(150.22929687,13.16502086)
\closepath
}
}
{
\newrgbcolor{curcolor}{0 0 0}
\pscustom[linestyle=none,fillstyle=solid,fillcolor=curcolor,opacity=0.18090455]
{
\newpath
\moveto(151.24441249,135.34548582)
\curveto(130.9974126,138.63979518)(113.58294707,148.40884512)(98.6862034,163.92912004)
\curveto(94.0664378,168.74226376)(85.64512847,177.59354809)(85.64512847,178.35348559)
\curveto(85.64512847,179.01443246)(94.47199093,186.55172617)(98.97668778,190.03318865)
\curveto(114.15985644,201.76751046)(133.76262509,210.38021979)(153.20106873,212.5267854)
\curveto(160.74255307,213.35957602)(176.10739048,212.15419478)(183.12879669,210.89294791)
\curveto(205.26461532,206.91673856)(224.54334647,198.0945636)(240.16784326,183.06596994)
\curveto(243.65947761,179.70749808)(245.18075886,178.16087621)(244.78256511,177.41683872)
\curveto(243.90841199,175.78346997)(229.23482144,161.99784193)(224.70076209,158.19561382)
\curveto(211.67834341,147.27514513)(195.89259662,139.48242017)(180.19200296,136.14916707)
\curveto(173.00024362,134.6223452)(157.97573746,134.25025145)(151.24441249,135.34548582)
\closepath
}
}
{
\newrgbcolor{curcolor}{0 0 0}
\pscustom[linestyle=none,fillstyle=solid,fillcolor=curcolor,opacity=0.06532665]
{
\newpath
\moveto(156.94420621,1.53177092)
\curveto(129.30941574,4.06749279)(105.75781274,15.1236646)(87.33838472,34.207827)
\curveto(77.2793379,44.62989256)(70.33169418,55.40142063)(64.92032234,68.96445806)
\curveto(57.47624425,87.62224546)(55.82467864,109.65828596)(60.34319424,130.03459522)
\curveto(63.67400047,145.05489202)(71.83444105,161.5378013)(81.93597537,173.64891374)
\curveto(83.66864411,175.7262731)(85.24152535,177.36356059)(85.4312566,177.28733247)
\curveto(85.62099722,177.21111372)(87.48832846,175.01795435)(89.5808847,172.41365437)
\curveto(105.89363774,152.11157635)(124.43338451,140.01596079)(146.25734377,135.43705144)
\curveto(154.15902498,133.77918583)(166.69275303,133.36347021)(174.46403737,134.50149207)
\curveto(196.62599349,137.74685456)(215.81490277,147.79938575)(233.67648079,165.52122941)
\curveto(236.70975265,168.53077314)(240.44950576,172.43846062)(241.98704325,174.20498248)
\lineto(244.78256511,177.41683872)
\lineto(247.77744634,173.85584811)
\curveto(257.87569004,161.84874193)(266.23623062,144.95671702)(269.59967122,129.7651671)
\curveto(273.53370245,111.99641095)(272.67033996,92.27469231)(267.21239311,75.23307052)
\curveto(254.8917588,36.76374261)(221.37834648,8.59021776)(180.61966233,2.43769592)
\curveto(175.75123423,1.70279905)(161.16350931,1.14462093)(156.94420621,1.53177092)
\closepath
\moveto(173.19272175,10.191374)
\curveto(173.92585612,10.47679588)(175.41740924,10.71699275)(176.5072811,10.72513025)
\curveto(177.62777172,10.73356775)(179.51768734,11.24603337)(180.85640921,11.90458024)
\curveto(182.1585592,12.54512711)(183.78601231,13.07115836)(184.47297481,13.07353961)
\curveto(185.15993731,13.07635211)(186.75802792,13.78858023)(188.02429042,14.6572396)
\curveto(189.29055291,15.5258896)(190.91323415,16.38243647)(191.63023415,16.56067397)
\curveto(192.34723414,16.73890209)(193.93018413,17.83936771)(195.14788413,19.00612395)
\curveto(196.3655935,20.17289895)(197.62704661,21.12752707)(197.95112161,21.12752707)
\curveto(198.71124661,21.12752707)(201.57726847,23.52969268)(201.23066534,23.87628643)
\curveto(200.92056847,24.18639268)(203.33712471,26.80963016)(203.93290596,26.80963016)
\curveto(204.58318408,26.80963016)(206.63124657,29.34004577)(206.64769032,30.16378952)
\curveto(206.65612782,30.57620514)(207.52160906,31.84406138)(208.57145905,32.9812395)
\curveto(209.62129967,34.118427)(210.59747154,35.59216761)(210.74074029,36.25621761)
\curveto(211.15489967,38.17593635)(211.95405279,40.06788634)(212.35079341,40.06788634)
\curveto(212.88374341,40.06788634)(214.30571527,43.83811757)(213.97474028,44.37364569)
\curveto(213.81998715,44.62404256)(214.14825277,45.76857068)(214.7042184,46.91704568)
\curveto(216.87785901,51.40719253)(216.89224026,69.02254556)(214.7255934,73.06509241)
\curveto(214.19743402,74.0505174)(213.76531215,75.43587052)(213.76531215,76.14365489)
\curveto(213.76531215,77.67455488)(212.89260278,79.75282362)(211.77786841,80.87656737)
\curveto(211.33585591,81.32215174)(210.73772154,82.56493923)(210.44869029,83.63832985)
\curveto(210.15966842,84.7117111)(209.2677028,86.26777359)(208.46656218,87.09624233)
\curveto(207.66542156,87.92471108)(206.73109969,89.2586142)(206.39029032,90.06044857)
\curveto(206.04948094,90.86230169)(204.58476845,92.53147355)(203.13536533,93.7697423)
\curveto(201.68597159,95.00800167)(200.60613097,96.19268291)(200.7357216,96.40235478)
\curveto(200.9877591,96.81017666)(198.02890599,99.25650789)(197.28360287,99.25650789)
\curveto(197.03853099,99.25650789)(196.01218412,100.01969851)(195.0028435,100.95249226)
\curveto(193.21489039,102.604836)(190.56336853,104.27856411)(190.56336853,103.75484849)
\curveto(190.56336853,103.43920162)(187.58392792,104.92892661)(187.2487998,105.41212348)
\curveto(186.79723417,106.0632266)(183.90425919,107.34352347)(183.47877482,107.0805641)
\curveto(183.24357482,106.93519535)(181.82476233,107.37720785)(180.32586858,108.06282034)
\curveto(178.79065609,108.76504534)(176.79249673,109.29822971)(175.74999673,109.28383909)
\curveto(174.73216236,109.26977659)(173.333928,109.47448909)(172.64282175,109.73872346)
\curveto(170.98186238,110.37375783)(161.75433118,110.62406096)(160.30351244,110.07342971)
\curveto(159.67702807,109.83567033)(157.75931558,109.54857971)(156.04192809,109.43546096)
\curveto(154.20047185,109.31417659)(152.14982186,108.83518909)(151.04362812,108.26797347)
\curveto(150.01194687,107.73897972)(148.74860001,107.3061641)(148.23619063,107.3061641)
\curveto(146.95906252,107.3061641)(144.15948128,106.25727973)(144.15948128,105.77878911)
\curveto(144.15948128,105.35356723)(142.00192817,104.34224849)(140.81503442,104.21114849)
\curveto(139.7862313,104.09750474)(137.45991882,102.5447235)(136.45261257,101.29930163)
\curveto(135.99006883,100.72741726)(135.13775008,100.11564226)(134.55857196,99.93980476)
\curveto(133.18516259,99.52284227)(130.28669386,97.3456579)(130.6377126,96.99464853)
\curveto(131.00913135,96.62322978)(128.98551886,94.52142042)(128.25648137,94.52142042)
\curveto(127.56284387,94.52142042)(125.22134388,91.85138293)(125.21406888,91.05214544)
\curveto(125.21125638,90.74659544)(124.25243764,89.45930482)(123.0832814,88.19150483)
\curveto(121.91412515,86.92371421)(120.95754703,85.69850484)(120.95754703,85.46883609)
\curveto(120.95754703,85.23915797)(120.96223453,84.94246735)(120.96720328,84.80952047)
\curveto(120.98304703,84.40965797)(119.43080017,81.51785486)(119.06357204,81.26316424)
\curveto(118.42254705,80.81859237)(117.1695408,77.99768301)(117.1695408,76.99912364)
\curveto(117.1695408,76.45104239)(116.73630331,75.05160802)(116.20680331,73.88927678)
\curveto(115.67729394,72.72694554)(115.15655019,70.55457055)(115.04958144,69.06178931)
\curveto(114.94262207,67.56899869)(114.66061269,65.7083987)(114.42291894,64.92710495)
\curveto(113.8714252,63.11446746)(114.14540957,54.07947064)(114.78829082,52.87823314)
\curveto(115.05625644,52.37753315)(115.27550019,50.93858316)(115.27550019,49.68056129)
\curveto(115.27550019,48.27511754)(115.65186894,46.69103318)(116.25168143,45.57190193)
\curveto(116.78858768,44.57016444)(117.30245018,43.1883082)(117.3936033,42.50110195)
\curveto(117.6447408,40.60781134)(118.44681892,38.79059885)(119.49320017,37.74422698)
\curveto(120.01071891,37.22669886)(120.66025641,35.96356761)(120.93661266,34.93725824)
\curveto(121.22179078,33.87819262)(122.26490953,32.22028013)(123.34874389,31.10345514)
\curveto(124.39906264,30.02118639)(125.44950326,28.63305828)(125.68307201,28.01873328)
\curveto(125.91663138,27.40440828)(127.2606595,25.97546142)(128.66979699,24.84328955)
\curveto(130.07892511,23.71112705)(131.4238251,22.38003644)(131.6584626,21.88532706)
\curveto(131.8931001,21.39060832)(133.15046572,20.42245207)(134.45261571,19.7338677)
\curveto(135.7547657,19.04528333)(137.38163757,17.92003021)(138.06787819,17.23331146)
\curveto(138.75411881,16.54660209)(140.16459693,15.75613022)(141.20225005,15.47671772)
\curveto(142.23990317,15.19730523)(143.59614691,14.58215523)(144.21610628,14.10972086)
\curveto(145.62025627,13.03971461)(148.36293126,11.91481774)(148.71332188,12.26520837)
\curveto(149.0591375,12.61102399)(151.88566249,11.63363337)(152.28398749,11.03049275)
\curveto(152.45532499,10.77103962)(154.31976873,10.45894588)(156.42718434,10.336949)
\curveto(158.53460933,10.21495213)(160.25885932,9.93215525)(160.25885932,9.70850526)
\curveto(160.25885932,9.13784901)(171.57154051,9.56018338)(173.19280612,10.191374)
\closepath
}
}
{
\newrgbcolor{curcolor}{0 0 0}
\pscustom[linewidth=0.93749996,linecolor=curcolor]
{
\newpath
\moveto(23.74141238,316.18759478)
\lineto(318.34327945,313.3613113)
}
}
{
\newrgbcolor{curcolor}{0 0 0}
\pscustom[linestyle=none,fillstyle=solid,fillcolor=curcolor,opacity=0]
{
\newpath
\moveto(78.13574414,189.72260427)
\curveto(67.85416295,204.82591669)(59.65987862,222.08402597)(48.43152555,249.57841645)
\curveto(41.03556934,267.68858197)(24.70650381,314.64923172)(25.26872255,315.21145984)
\curveto(25.42974755,315.37248484)(55.50105364,315.18970984)(92.09384406,314.80532547)
\curveto(128.68663449,314.42093172)(194.18422788,313.76413797)(237.6440464,313.3457786)
\curveto(281.10386491,312.92741923)(317.39919596,312.27676611)(317.50745846,312.18122548)
\curveto(317.86103658,311.86923486)(298.29708982,269.03423197)(290.54471486,253.14664455)
\curveto(282.2148774,236.07562277)(275.21052432,221.71352597)(267.56250248,209.42671667)
\curveto(262.45521814,201.22166984)(252.00461507,185.56406992)(248.18260571,181.58689494)
\lineto(245.42755573,178.71999183)
\lineto(239.67941514,183.89584493)
\curveto(223.56982772,198.40158548)(204.49135595,208.08861355)(183.33343419,211.62866978)
\curveto(169.97377802,213.86395102)(155.67734372,213.60057914)(141.98345004,210.80994166)
\curveto(137.59670007,209.91597916)(133.65125946,208.75276667)(129.24077199,207.20421355)
\curveto(114.40265957,201.99446046)(100.65576902,193.44758238)(90.2253222,183.53799181)
\lineto(85.4405191,178.99211996)
\closepath
}
}
{
\newrgbcolor{curcolor}{0 0 0}
\pscustom[linewidth=0.47445685,linecolor=curcolor,linestyle=dashed,dash=0 1.47292502]
{
\newpath
\moveto(0.23722795,327.63142243)
\lineto(333.84632923,327.63142243)
\lineto(333.84632923,195.25854882)
\lineto(0.23722795,195.25854882)
\closepath
}
}
{
\newrgbcolor{curcolor}{0 0 0}
\pscustom[linestyle=none,fillstyle=solid,fillcolor=curcolor]
{
\newpath
\moveto(164.36518101,64.95044983)
\lineto(164.36518101,60.15674849)
\lineto(166.53559124,60.15674849)
\curveto(167.33880751,60.15674194)(167.9597379,60.36466817)(168.39838428,60.78052781)
\curveto(168.8370157,61.1963731)(169.05633515,61.78882045)(169.05634328,62.55787162)
\curveto(169.05633515,63.32120829)(168.8370157,63.91080733)(168.39838428,64.32667051)
\curveto(167.9597379,64.74251226)(167.33880751,64.95043849)(166.53559124,64.95044983)
\lineto(164.36518101,64.95044983)
\moveto(162.63910673,66.36890691)
\lineto(166.53559124,66.36890691)
\curveto(167.96543451,66.36889415)(169.04494193,66.04418743)(169.77411675,65.39478578)
\curveto(170.50897047,64.75105717)(170.87640175,63.80542006)(170.87641171,62.55787162)
\curveto(170.87640175,61.29891206)(170.50897047,60.34757835)(169.77411675,59.70386761)
\curveto(169.04494193,59.06014469)(167.96543451,58.73828628)(166.53559124,58.73829141)
\lineto(164.36518101,58.73829141)
\lineto(164.36518101,53.6113381)
\lineto(162.63910673,53.6113381)
\lineto(162.63910673,66.36890691)
}
}
{
\newrgbcolor{curcolor}{0 0 0}
\pscustom[linestyle=none,fillstyle=solid,fillcolor=curcolor]
{
\newpath
\moveto(277.34910585,109.516143)
\lineto(279.67332469,109.516143)
\lineto(285.33006317,98.8435352)
\lineto(285.33006317,109.516143)
\lineto(287.00486791,109.516143)
\lineto(287.00486791,96.75857419)
\lineto(284.68064908,96.75857419)
\lineto(279.0239106,107.43118199)
\lineto(279.0239106,96.75857419)
\lineto(277.34910585,96.75857419)
\lineto(277.34910585,109.516143)
}
}
{
\newrgbcolor{curcolor}{0 0 0}
\pscustom[linestyle=none,fillstyle=solid,fillcolor=curcolor]
{
\newpath
\moveto(292.1660008,108.09768592)
\lineto(292.1660008,103.30398458)
\lineto(294.33641104,103.30398458)
\curveto(295.1396273,103.30397803)(295.76055769,103.51190426)(296.19920407,103.92776389)
\curveto(296.63783549,104.34360919)(296.85715494,104.93605653)(296.85716308,105.70510771)
\curveto(296.85715494,106.46844438)(296.63783549,107.05804342)(296.19920407,107.4739066)
\curveto(295.76055769,107.88974834)(295.1396273,108.09767458)(294.33641104,108.09768592)
\lineto(292.1660008,108.09768592)
\moveto(290.43992652,109.516143)
\lineto(294.33641104,109.516143)
\curveto(295.7662543,109.51613024)(296.84576172,109.19142352)(297.57493654,108.54202187)
\curveto(298.30979026,107.89829326)(298.67722154,106.95265615)(298.6772315,105.70510771)
\curveto(298.67722154,104.44614815)(298.30979026,103.49481443)(297.57493654,102.8511037)
\curveto(296.84576172,102.20738078)(295.7662543,101.88552237)(294.33641104,101.88552749)
\lineto(292.1660008,101.88552749)
\lineto(292.1660008,96.75857419)
\lineto(290.43992652,96.75857419)
\lineto(290.43992652,109.516143)
}
}
{
\newrgbcolor{curcolor}{0 0 0}
\pscustom[linestyle=none,fillstyle=solid,fillcolor=curcolor]
{
\newpath
\moveto(110.56337038,181.88214611)
\lineto(112.88758921,181.88214611)
\lineto(118.54432769,171.20953832)
\lineto(118.54432769,181.88214611)
\lineto(120.21913244,181.88214611)
\lineto(120.21913244,169.12457731)
\lineto(117.89491361,169.12457731)
\lineto(112.23817513,179.7971851)
\lineto(112.23817513,169.12457731)
\lineto(110.56337038,169.12457731)
\lineto(110.56337038,181.88214611)
}
}
{
\newrgbcolor{curcolor}{0 0 0}
\pscustom[linestyle=none,fillstyle=solid,fillcolor=curcolor]
{
\newpath
\moveto(125.38026533,180.46368903)
\lineto(125.38026533,175.66998769)
\lineto(127.55067556,175.66998769)
\curveto(128.35389183,175.66998115)(128.97482222,175.87790738)(129.4134686,176.29376701)
\curveto(129.85210002,176.70961231)(130.07141947,177.30205965)(130.0714276,178.07111083)
\curveto(130.07141947,178.8344475)(129.85210002,179.42404654)(129.4134686,179.83990972)
\curveto(128.97482222,180.25575146)(128.35389183,180.46367769)(127.55067556,180.46368903)
\lineto(125.38026533,180.46368903)
\moveto(123.65419105,181.88214611)
\lineto(127.55067556,181.88214611)
\curveto(128.98051883,181.88213336)(130.06002625,181.55742664)(130.78920107,180.90802499)
\curveto(131.52405478,180.26429638)(131.89148607,179.31865927)(131.89149603,178.07111083)
\curveto(131.89148607,176.81215127)(131.52405478,175.86081755)(130.78920107,175.21710682)
\curveto(130.06002625,174.5733839)(128.98051883,174.25152549)(127.55067556,174.25153061)
\lineto(125.38026533,174.25153061)
\lineto(125.38026533,169.12457731)
\lineto(123.65419105,169.12457731)
\lineto(123.65419105,181.88214611)
}
}
{
\newrgbcolor{curcolor}{0 0 0}
\pscustom[linestyle=none,fillstyle=solid,fillcolor=curcolor]
{
\newpath
\moveto(132.97670037,174.61896227)
\lineto(137.58241342,174.61896227)
\lineto(137.58241342,173.21759503)
\lineto(132.97670037,173.21759503)
\lineto(132.97670037,174.61896227)
}
}
{
\newrgbcolor{curcolor}{0 0 0}
\pscustom[linestyle=none,fillstyle=solid,fillcolor=curcolor]
{
\newpath
\moveto(146.98183027,178.32745849)
\lineto(146.98183027,176.85773188)
\curveto(146.53748622,177.10267833)(146.09030241,177.28496982)(145.64027748,177.4046069)
\curveto(145.19593478,177.52992402)(144.74590266,177.59258672)(144.29017978,177.59259518)
\curveto(143.27048091,177.59258672)(142.47865225,177.26788)(141.91469142,176.61847406)
\curveto(141.35072365,175.97474973)(141.0687415,175.06898889)(141.06874412,173.9011888)
\curveto(141.0687415,172.73337916)(141.35072365,171.82477002)(141.91469142,171.17535863)
\curveto(142.47865225,170.53163975)(143.27048091,170.20978134)(144.29017978,170.20978242)
\curveto(144.74590266,170.20978134)(145.19593478,170.26959573)(145.64027748,170.38922579)
\curveto(146.09030241,170.51454993)(146.53748622,170.69968972)(146.98183027,170.94464573)
\lineto(146.98183027,169.49200896)
\curveto(146.54318283,169.28693067)(146.0874541,169.13312222)(145.61464272,169.03058316)
\curveto(145.1475136,168.9280443)(144.64906031,168.87677481)(144.11928134,168.87677457)
\curveto(142.67803356,168.87677481)(141.53301514,169.32965524)(140.68422262,170.23541719)
\curveto(139.83542563,171.14117693)(139.41102825,172.36309957)(139.41102922,173.9011888)
\curveto(139.41102825,175.46205492)(139.83827393,176.68967418)(140.69276754,177.58405026)
\curveto(141.55295327,178.47840943)(142.72930305,178.92559324)(144.2218204,178.92560304)
\curveto(144.7060264,178.92559324)(145.17884495,178.87432376)(145.64027748,178.77179444)
\curveto(146.10169563,178.67494244)(146.54887944,178.52683061)(146.98183027,178.32745849)
}
}
{
\newrgbcolor{curcolor}{0 0 0}
\pscustom[linestyle=none,fillstyle=solid,fillcolor=curcolor]
{
\newpath
\moveto(153.44179071,177.59259518)
\curveto(152.59868721,177.59258672)(151.93218394,177.26218339)(151.44227892,176.60138421)
\curveto(150.95236718,175.94626669)(150.70741299,175.04620245)(150.70741561,173.9011888)
\curveto(150.70741299,172.7561656)(150.94951888,171.85325306)(151.433734,171.19244847)
\curveto(151.92363903,170.53733636)(152.5929906,170.20978134)(153.44179071,170.20978242)
\curveto(154.27918689,170.20978134)(154.94284185,170.54018467)(155.43275758,171.2009934)
\curveto(155.92265861,171.86179797)(156.1676128,172.76186221)(156.16762088,173.9011888)
\curveto(156.1676128,175.03480924)(155.92265861,175.93202517)(155.43275758,176.59283929)
\curveto(154.94284185,177.25933508)(154.27918689,177.59258672)(153.44179071,177.59259518)
\moveto(153.44179071,178.92560304)
\curveto(154.80897153,178.92559324)(155.88278235,178.48125773)(156.66322637,177.59259518)
\curveto(157.44365324,176.7039157)(157.83387096,175.47344814)(157.83388071,173.9011888)
\curveto(157.83387096,172.33461653)(157.44365324,171.10414897)(156.66322637,170.20978242)
\curveto(155.88278235,169.32111032)(154.80897153,168.87677481)(153.44179071,168.87677457)
\curveto(152.06890256,168.87677481)(150.99224344,169.32111032)(150.21181013,170.20978242)
\curveto(149.43706916,171.10414897)(149.04969975,172.33461653)(149.04970071,173.9011888)
\curveto(149.04969975,175.47344814)(149.43706916,176.7039157)(150.21181013,177.59259518)
\curveto(150.99224344,178.48125773)(152.06890256,178.92559324)(153.44179071,178.92560304)
}
}
{
\newrgbcolor{curcolor}{0 0 0}
\pscustom[linestyle=none,fillstyle=solid,fillcolor=curcolor]
{
\newpath
\moveto(167.88270783,176.85773188)
\curveto(168.27576476,177.56410367)(168.74573501,178.0853434)(169.29261999,178.42145263)
\curveto(169.83948396,178.75754327)(170.48320078,178.92559324)(171.22377241,178.92560304)
\curveto(172.22066655,178.92559324)(172.98970878,178.57525178)(173.53090139,177.87457762)
\curveto(174.07206451,177.17958256)(174.34265344,176.18837258)(174.342669,174.9009447)
\lineto(174.342669,169.12457731)
\lineto(172.7618584,169.12457731)
\lineto(172.7618584,174.84967517)
\curveto(172.76184442,175.7668235)(172.59949106,176.44756829)(172.27479783,176.89191157)
\curveto(171.95007762,177.33623931)(171.45447263,177.55840706)(170.78798137,177.55841549)
\curveto(169.97335427,177.55840706)(169.32963744,177.28781813)(168.85682896,176.74664789)
\curveto(168.38400033,176.2054624)(168.14759106,175.46775153)(168.14760042,174.53351304)
\lineto(168.14760042,169.12457731)
\lineto(166.56678982,169.12457731)
\lineto(166.56678982,174.84967517)
\curveto(166.56678204,175.77252011)(166.40442868,176.4532649)(166.07972926,176.89191157)
\curveto(165.75501524,177.33623931)(165.25371364,177.55840706)(164.57582295,177.55841549)
\curveto(163.77259528,177.55840706)(163.13457506,177.28496982)(162.66176039,176.73810297)
\curveto(162.18893795,176.19691749)(161.95252868,175.46205492)(161.95253185,174.53351304)
\lineto(161.95253185,169.12457731)
\lineto(160.37172124,169.12457731)
\lineto(160.37172124,178.69489014)
\lineto(161.95253185,178.69489014)
\lineto(161.95253185,177.20807369)
\curveto(162.31141505,177.79481634)(162.74150903,178.22775863)(163.24281509,178.50690186)
\curveto(163.74411223,178.78602632)(164.33940788,178.92559324)(165.02870383,178.92560304)
\curveto(165.72368389,178.92559324)(166.31328293,178.74899836)(166.79750272,178.39581787)
\curveto(167.28740308,178.04261883)(167.64913776,177.52992402)(167.88270783,176.85773188)
}
}
{
\newrgbcolor{curcolor}{0 0 0}
\pscustom[linestyle=none,fillstyle=solid,fillcolor=curcolor]
{
\newpath
\moveto(179.00819813,170.56012423)
\lineto(179.00819813,165.48444046)
\lineto(177.42738753,165.48444046)
\lineto(177.42738753,178.69489014)
\lineto(179.00819813,178.69489014)
\lineto(179.00819813,177.24225337)
\curveto(179.33859829,177.81190617)(179.75445075,178.23345524)(180.25575677,178.50690186)
\curveto(180.76275056,178.78602632)(181.36659112,178.92559324)(182.06728027,178.92560304)
\curveto(183.2293823,178.92559324)(184.1721711,178.46416791)(184.89564951,177.54132565)
\curveto(185.62480642,176.61846656)(185.9893894,175.40508883)(185.98939955,173.9011888)
\curveto(185.9893894,172.39727923)(185.62480642,171.18390149)(184.89564951,170.26105196)
\curveto(184.1721711,169.33820015)(183.2293823,168.87677481)(182.06728027,168.87677457)
\curveto(181.36659112,168.87677481)(180.76275056,169.01349343)(180.25575677,169.28693083)
\curveto(179.75445075,169.56606451)(179.33859829,169.99046189)(179.00819813,170.56012423)
\moveto(184.35731941,173.9011888)
\curveto(184.35731089,175.05759567)(184.11805331,175.96335652)(183.63954595,176.61847406)
\curveto(183.1667196,177.27927322)(182.51445786,177.60967654)(181.68275877,177.60968503)
\curveto(180.851048,177.60967654)(180.19593796,177.27927322)(179.71742667,176.61847406)
\curveto(179.24460424,175.96335652)(179.00819496,175.05759567)(179.00819813,173.9011888)
\curveto(179.00819496,172.74477238)(179.24460424,171.83616323)(179.71742667,171.17535863)
\curveto(180.19593796,170.52024653)(180.851048,170.19269151)(181.68275877,170.19269258)
\curveto(182.51445786,170.19269151)(183.1667196,170.52024653)(183.63954595,171.17535863)
\curveto(184.11805331,171.83616323)(184.35731089,172.74477238)(184.35731941,173.9011888)
}
}
{
\newrgbcolor{curcolor}{0 0 0}
\pscustom[linestyle=none,fillstyle=solid,fillcolor=curcolor]
{
\newpath
\moveto(188.595599,182.42047621)
\lineto(190.16786468,182.42047621)
\lineto(190.16786468,169.12457731)
\lineto(188.595599,169.12457731)
\lineto(188.595599,182.42047621)
}
}
{
\newrgbcolor{curcolor}{0 0 0}
\pscustom[linestyle=none,fillstyle=solid,fillcolor=curcolor]
{
\newpath
\moveto(201.6351504,174.30280015)
\lineto(201.6351504,173.53375715)
\lineto(194.40614624,173.53375715)
\curveto(194.47450295,172.45139701)(194.79920967,171.6253887)(195.38026737,171.05572972)
\curveto(195.96701453,170.49176349)(196.78162963,170.20978134)(197.82411511,170.20978242)
\curveto(198.42794965,170.20978134)(199.01185208,170.28383726)(199.57582416,170.4319504)
\curveto(200.14547729,170.58006093)(200.70944159,170.80222868)(201.26771875,171.09845433)
\lineto(201.26771875,169.61163787)
\curveto(200.70374498,169.3723798)(200.12553916,169.19008831)(199.53309955,169.06476285)
\curveto(198.94064447,168.93943751)(198.33965221,168.87677481)(197.73012097,168.87677457)
\curveto(196.2034238,168.87677481)(194.99289437,169.32111032)(194.09852905,170.20978242)
\curveto(193.20985573,171.09845236)(192.76552022,172.30043687)(192.76552119,173.81573958)
\curveto(192.76552022,175.38230239)(193.18706929,176.62416317)(194.03016967,177.54132565)
\curveto(194.87896219,178.46416791)(196.02113231,178.92559324)(197.45668346,178.92560304)
\curveto(198.74411146,178.92559324)(199.76095618,178.50974078)(200.50722068,177.67804441)
\curveto(201.25916437,176.85202753)(201.63514057,175.72694724)(201.6351504,174.30280015)
\moveto(200.06288472,174.76422594)
\curveto(200.05148324,175.62440828)(199.80937736,176.31084967)(199.33656634,176.82355219)
\curveto(198.86943686,177.33623931)(198.24850647,177.59258672)(197.47377331,177.59259518)
\curveto(196.59648983,177.59258672)(195.89295861,177.34478422)(195.36317753,176.84918695)
\curveto(194.83908593,176.35357424)(194.53716565,175.65573963)(194.45741578,174.75568102)
\lineto(200.06288472,174.76422594)
}
}
{
\newrgbcolor{curcolor}{0 0 0}
\pscustom[linestyle=none,fillstyle=solid,fillcolor=curcolor]
{
\newpath
\moveto(205.77089244,181.4121754)
\lineto(205.77089244,178.69489014)
\lineto(209.00941794,178.69489014)
\lineto(209.00941794,177.47296627)
\lineto(205.77089244,177.47296627)
\lineto(205.77089244,172.27765359)
\curveto(205.77088923,171.49721499)(205.8762765,170.99591339)(206.08705456,170.77374729)
\curveto(206.30352218,170.55157788)(206.73931278,170.44049401)(207.39442765,170.44049532)
\lineto(209.00941794,170.44049532)
\lineto(209.00941794,169.12457731)
\lineto(207.39442765,169.12457731)
\curveto(206.18104509,169.12457731)(205.34364355,169.34959337)(204.88222053,169.79962616)
\curveto(204.42079288,170.25535421)(204.19008021,171.08136253)(204.19008183,172.27765359)
\lineto(204.19008183,177.47296627)
\lineto(203.03651734,177.47296627)
\lineto(203.03651734,178.69489014)
\lineto(204.19008183,178.69489014)
\lineto(204.19008183,181.4121754)
\lineto(205.77089244,181.4121754)
}
}
{
\newrgbcolor{curcolor}{0 0 0}
\pscustom[linestyle=none,fillstyle=solid,fillcolor=curcolor]
{
\newpath
\moveto(214.79433096,177.59259518)
\curveto(213.95122746,177.59258672)(213.2847242,177.26218339)(212.79481917,176.60138421)
\curveto(212.30490743,175.94626669)(212.05995324,175.04620245)(212.05995587,173.9011888)
\curveto(212.05995324,172.7561656)(212.30205913,171.85325306)(212.78627425,171.19244847)
\curveto(213.27617928,170.53733636)(213.94553085,170.20978134)(214.79433096,170.20978242)
\curveto(215.63172714,170.20978134)(216.2953821,170.54018467)(216.78529783,171.2009934)
\curveto(217.27519886,171.86179797)(217.52015305,172.76186221)(217.52016114,173.9011888)
\curveto(217.52015305,175.03480924)(217.27519886,175.93202517)(216.78529783,176.59283929)
\curveto(216.2953821,177.25933508)(215.63172714,177.59258672)(214.79433096,177.59259518)
\moveto(214.79433096,178.92560304)
\curveto(216.16151178,178.92559324)(217.2353226,178.48125773)(218.01576662,177.59259518)
\curveto(218.79619349,176.7039157)(219.18641121,175.47344814)(219.18642096,173.9011888)
\curveto(219.18641121,172.33461653)(218.79619349,171.10414897)(218.01576662,170.20978242)
\curveto(217.2353226,169.32111032)(216.16151178,168.87677481)(214.79433096,168.87677457)
\curveto(213.42144281,168.87677481)(212.3447837,169.32111032)(211.56435038,170.20978242)
\curveto(210.78960942,171.10414897)(210.40224,172.33461653)(210.40224096,173.9011888)
\curveto(210.40224,175.47344814)(210.78960942,176.7039157)(211.56435038,177.59259518)
\curveto(212.3447837,178.48125773)(213.42144281,178.92559324)(214.79433096,178.92560304)
}
}
{
\newrgbcolor{curcolor}{0 0 0}
\pscustom[linestyle=none,fillstyle=solid,fillcolor=curcolor]
{
\newpath
\moveto(132.36614961,249.91784904)
\lineto(134.69036844,249.91784904)
\lineto(140.34710692,239.24524125)
\lineto(140.34710692,249.91784904)
\lineto(142.02191166,249.91784904)
\lineto(142.02191166,237.16028024)
\lineto(139.69769283,237.16028024)
\lineto(134.04095435,247.83288803)
\lineto(134.04095435,237.16028024)
\lineto(132.36614961,237.16028024)
\lineto(132.36614961,249.91784904)
}
}
{
\newrgbcolor{curcolor}{0 0 0}
\pscustom[linestyle=none,fillstyle=solid,fillcolor=curcolor]
{
\newpath
\moveto(147.18304455,248.49939196)
\lineto(147.18304455,243.70569062)
\lineto(149.35345479,243.70569062)
\curveto(150.15667105,243.70568408)(150.77760144,243.91361031)(151.21624782,244.32946994)
\curveto(151.65487924,244.74531524)(151.87419869,245.33776258)(151.87420683,246.10681375)
\curveto(151.87419869,246.87015043)(151.65487924,247.45974947)(151.21624782,247.87561264)
\curveto(150.77760144,248.29145439)(150.15667105,248.49938062)(149.35345479,248.49939196)
\lineto(147.18304455,248.49939196)
\moveto(145.45697028,249.91784904)
\lineto(149.35345479,249.91784904)
\curveto(150.78329805,249.91783629)(151.86280548,249.59312957)(152.59198029,248.94372792)
\curveto(153.32683401,248.29999931)(153.6942653,247.3543622)(153.69427525,246.10681375)
\curveto(153.6942653,244.8478542)(153.32683401,243.89652048)(152.59198029,243.25280975)
\curveto(151.86280548,242.60908683)(150.78329805,242.28722842)(149.35345479,242.28723354)
\lineto(147.18304455,242.28723354)
\lineto(147.18304455,237.16028024)
\lineto(145.45697028,237.16028024)
\lineto(145.45697028,249.91784904)
}
}
{
\newrgbcolor{curcolor}{0 0 0}
\pscustom[linestyle=none,fillstyle=solid,fillcolor=curcolor]
{
\newpath
\moveto(154.7794796,242.6546652)
\lineto(159.38519265,242.6546652)
\lineto(159.38519265,241.25329796)
\lineto(154.7794796,241.25329796)
\lineto(154.7794796,242.6546652)
}
}
{
\newrgbcolor{curcolor}{0 0 0}
\pscustom[linestyle=none,fillstyle=solid,fillcolor=curcolor]
{
\newpath
\moveto(169.85272476,242.93664763)
\lineto(169.85272476,237.16028024)
\lineto(168.28045908,237.16028024)
\lineto(168.28045908,242.88537809)
\curveto(168.28045105,243.79113321)(168.10385617,244.4690297)(167.75067391,244.91906957)
\curveto(167.39747664,245.36909393)(166.867692,245.59410999)(166.16131838,245.59411842)
\curveto(165.31251772,245.59410999)(164.64316615,245.32352106)(164.15326167,244.78235082)
\curveto(163.66334939,244.24116533)(163.4183952,243.50345446)(163.41839837,242.56921597)
\lineto(163.41839837,237.16028024)
\lineto(161.83758776,237.16028024)
\lineto(161.83758776,250.45617914)
\lineto(163.41839837,250.45617914)
\lineto(163.41839837,245.24377661)
\curveto(163.7943714,245.81912605)(164.2358586,246.24922003)(164.7428613,246.53405986)
\curveto(165.25555163,246.81888094)(165.84515067,246.96129617)(166.51166019,246.96130597)
\curveto(167.61109948,246.96129617)(168.44280441,246.61949963)(169.00677747,245.93591531)
\curveto(169.57073301,245.25801005)(169.85271516,244.25825516)(169.85272476,242.93664763)
}
}
{
\newrgbcolor{curcolor}{0 0 0}
\pscustom[linestyle=none,fillstyle=solid,fillcolor=curcolor]
{
\newpath
\moveto(177.35516462,241.97107142)
\curveto(176.0848148,241.97106661)(175.20468869,241.82580308)(174.71478367,241.53528039)
\curveto(174.22487193,241.24474895)(173.97991774,240.74914396)(173.97992036,240.04846393)
\curveto(173.97991774,239.49019335)(174.16220923,239.04585785)(174.52679538,238.71545607)
\curveto(174.8970718,238.3907478)(175.3983734,238.22839444)(176.03070168,238.22839551)
\curveto(176.9022782,238.22839444)(177.60011281,238.53601133)(178.12420762,239.1512471)
\curveto(178.65398549,239.7721755)(178.91887782,240.59533552)(178.91888538,241.62072961)
\lineto(178.91888538,241.97107142)
\lineto(177.35516462,241.97107142)
\moveto(180.49115106,242.62048551)
\lineto(180.49115106,237.16028024)
\lineto(178.91888538,237.16028024)
\lineto(178.91888538,238.61291701)
\curveto(178.55999144,238.03186143)(178.11280763,237.60176744)(177.5773326,237.32263376)
\curveto(177.04184512,237.04919636)(176.38673508,236.91247774)(175.6120005,236.91247749)
\curveto(174.63217948,236.91247774)(173.85174403,237.18591498)(173.27069182,237.73279002)
\curveto(172.69533239,238.28536053)(172.40765363,239.02307141)(172.40765468,239.94592487)
\curveto(172.40765363,241.0225812)(172.76654,241.83434799)(173.48431488,242.38122769)
\curveto(174.2077821,242.92809694)(175.28444122,243.20153417)(176.71429546,243.20154021)
\lineto(178.91888538,243.20154021)
\lineto(178.91888538,243.35534881)
\curveto(178.91887782,244.07881197)(178.67962023,244.63707966)(178.20111192,245.03015356)
\curveto(177.72828652,245.42890833)(177.06178325,245.62828964)(176.20160013,245.62829811)
\curveto(175.65472081,245.62828964)(175.12208786,245.56277864)(174.60369968,245.4317649)
\curveto(174.08530501,245.30073462)(173.58685171,245.10420161)(173.1083383,244.84216527)
\lineto(173.1083383,246.29480204)
\curveto(173.68369407,246.51696066)(174.24196176,246.68216233)(174.78314305,246.79040753)
\curveto(175.32431748,246.90433008)(175.85125382,246.96129617)(176.36395365,246.96130597)
\curveto(177.74822465,246.96129617)(178.7821592,246.6024098)(179.4657604,245.88464578)
\curveto(180.14934538,245.16686431)(180.49114192,244.07881197)(180.49115106,242.62048551)
}
}
{
\newrgbcolor{curcolor}{0 0 0}
\pscustom[linestyle=none,fillstyle=solid,fillcolor=curcolor]
{
\newpath
\moveto(189.28387515,245.26086646)
\curveto(189.10727307,245.36339732)(188.91358837,245.43745324)(188.70282044,245.48303444)
\curveto(188.4977359,245.53429559)(188.26987154,245.55993034)(188.01922667,245.55993873)
\curveto(187.13054972,245.55993034)(186.44695663,245.26940327)(185.96844535,244.68835667)
\curveto(185.49562291,244.11299163)(185.25921364,243.28413501)(185.25921681,242.20178432)
\lineto(185.25921681,237.16028024)
\lineto(183.6784062,237.16028024)
\lineto(183.6784062,246.73059307)
\lineto(185.25921681,246.73059307)
\lineto(185.25921681,245.24377661)
\curveto(185.58961696,245.82482266)(186.01971095,246.25491664)(186.54950006,246.53405986)
\curveto(187.07928024,246.81888094)(187.72299707,246.96129617)(188.48065247,246.96130597)
\curveto(188.58888165,246.96129617)(188.70851044,246.95275126)(188.8395392,246.93567121)
\curveto(188.97055446,246.92426821)(189.11581799,246.90433008)(189.27533023,246.87585675)
\lineto(189.28387515,245.26086646)
}
}
{
\newrgbcolor{curcolor}{0 0 0}
\pscustom[linestyle=none,fillstyle=solid,fillcolor=curcolor]
{
\newpath
\moveto(196.94012665,245.2779563)
\lineto(196.94012665,250.45617914)
\lineto(198.51239233,250.45617914)
\lineto(198.51239233,237.16028024)
\lineto(196.94012665,237.16028024)
\lineto(196.94012665,238.59582716)
\curveto(196.60971538,238.02616482)(196.19101461,237.60176744)(195.68402309,237.32263376)
\curveto(195.1827148,237.04919636)(194.57887424,236.91247774)(193.87249959,236.91247749)
\curveto(192.71608307,236.91247774)(191.77329427,237.37390308)(191.04413035,238.29675489)
\curveto(190.32065895,239.21960442)(189.95892427,240.43298216)(189.95892524,241.93689173)
\curveto(189.95892427,243.44079176)(190.32065895,244.65416949)(191.04413035,245.57702858)
\curveto(191.77329427,246.49987083)(192.71608307,246.96129617)(193.87249959,246.96130597)
\curveto(194.57887424,246.96129617)(195.1827148,246.82172925)(195.68402309,246.54260478)
\curveto(196.19101461,246.26915817)(196.60971538,245.84760909)(196.94012665,245.2779563)
\moveto(191.58246045,241.93689173)
\curveto(191.58245786,240.78047531)(191.81886714,239.87186616)(192.29168899,239.21106156)
\curveto(192.77020086,238.55594946)(193.4253109,238.22839444)(194.25702109,238.22839551)
\curveto(195.08872075,238.22839444)(195.7438308,238.55594946)(196.22235319,239.21106156)
\curveto(196.70086112,239.87186616)(196.94011871,240.78047531)(196.94012665,241.93689173)
\curveto(196.94011871,243.0932986)(196.70086112,243.99905945)(196.22235319,244.65417698)
\curveto(195.7438308,245.31497614)(195.08872075,245.64537947)(194.25702109,245.64538796)
\curveto(193.4253109,245.64537947)(192.77020086,245.31497614)(192.29168899,244.65417698)
\curveto(191.81886714,243.99905945)(191.58245786,243.0932986)(191.58246045,241.93689173)
}
}
\end{pspicture}

\end{figure}

\subsection{Algoritmos Pseudopolinomiales}

Decimos que un algoritmo corre en tiempo \emph{``pseudopolinomial''} si la complejidad del mismo es del orden polinomial en el valor num\'erico de la entrada, en lugar de la cantidad de bits necesaria para representarla.

Un ejemplo de esto es el algoritmo que determina si un n\'umero cualquiera $n$ es o no primo, y para esto se fija si dicho n\'umero tiene alg\'un divisor en el intervalo $[2, .., n]$.

Este algoritmo va a iterar $n$ veces, y en cada iteraci\'on va a calcular $n \textrm{ mod } i$. Si hacer la operaci\'on de m\'odulo costara $\mathcal{O}(n^3)$, el algoritmo corre en $\mathcal{O}(n^4)$.